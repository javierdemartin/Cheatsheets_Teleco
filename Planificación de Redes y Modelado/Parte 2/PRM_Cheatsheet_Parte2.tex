\documentclass[9pt,landscape]{extarticle}
\usepackage{multicol}
\usepackage{calc}
\usepackage[landscape]{geometry}
\usepackage{amsmath,amsthm,amsfonts,amssymb}
\usepackage{color,graphicx,overpic}
\graphicspath{ {images/} }
\usepackage{hyperref}
\usepackage{tikz}
\usepackage{esint}
\usepackage{bm}
\usepackage{relsize}
\usepackage{datetime}
\usepackage[utf8] {inputenc}
\usepackage[spanish, activeacute] {babel}
\usepackage{IEEEtrantools}
\usepackage{framed}
\usepackage{wrapfig}

\usetikzlibrary{arrows}

\usepackage{pgfplots}

\usepackage{pgf}
\usetikzlibrary{arrows,automata}

\usepackage{draftwatermark}
\SetWatermarkText{Javier de Martín}
\SetWatermarkScale{2}


\ifthenelse{\lengthtest { \paperwidth = 11in}}
    { \geometry{top=.5in,left=.5in,right=.5in,bottom=.5in} }
    {\ifthenelse{ \lengthtest{ \paperwidth = 297mm}}
        {\geometry{top=1cm,left=1cm,right=1cm,bottom=1cm} }
        {\geometry{top=1cm,left=1cm,right=1cm,bottom=1cm} }
    }

% Turn off header and footer
\pagestyle{empty}

% Redefine section commands to use less space
\makeatletter
\renewcommand{\section}{\@startsection{section}{1}{0mm}%
                                {-1ex plus -.5ex minus -.2ex}%
                                {0.5ex plus .2ex}%x
                                {\normalfont\large\bfseries}}
\renewcommand{\subsection}{\@startsection{subsection}{2}{0mm}%
                                {-1explus -.5ex minus -.2ex}%
                                {0.5ex plus .2ex}%
                                {\normalfont\normalsize\bfseries}}
\renewcommand{\subsubsection}{\@startsection{subsubsection}{3}{0mm}%
                                {-1ex plus -.5ex minus -.2ex}%
                                {1ex plus .2ex}%
                                {\normalfont\small\bfseries}}
\makeatother

\newcommand{\Lagr}{\mathcal{L}}

% Define BibTeX command
\def\BibTeX{{\rm B\kern-.05em{\sc i\kern-.025em b}\kern-.08em
    T\kern-.1667em\lower.7ex\hbox{E}\kern-.125emX}}

% Don't print section numbers
\setcounter{secnumdepth}{0}


\setlength{\parindent}{0pt}
\setlength{\parskip}{0pt plus 0.5ex}

%My Environments
\newtheorem{example}[section]{Example}
% ---------------------------------------------------------------


\tikzset{global scale/.style={
    scale=#1,
    every node/.style={scale=#1}
  }
}

\begin{document}
\raggedright
\footnotesize
\begin{multicols}{3}


% multicol parameters
% These lengths are set only within the two main columns
%\setlength{\columnseprule}{0.25pt}
\setlength{\premulticols}{1pt}
\setlength{\postmulticols}{1pt}
\setlength{\multicolsep}{1pt}
\setlength{\columnsep}{2pt}

\begin{framed}
	\begin{center}
    	\Large{\underline{PRM: ``Modelado de Tráfico''}} \\
    	\scriptsize{3º Ingeniería de Telecomunicaciones | UPV/EHU}\\
     	"\textsl{Under-promise and over-deliver}." \\
     	\small{\textbf{Javier de Martín -- 2016/17}}
	\end{center}
\end{framed}

%
% Cheatsheet code below 
%                             

%\section{Variables}
%
%$\lambda$: Tasa de llegada de usuarios (nº usuarios llegados / tiempo) \\
%	\quad $1 / \lambda$: Tiempo medio entre llegadas consecutivas\\
%$\mu$: Tasa de servicio\\
%	\quad $1 / \mu$: Tiempo medio de servicio\\
%$C$: Capacidad del canal\\
%$\rho = \frac{\lambda}{\mu}$: Factor de utilización (intensidad de tráfico)\\
%$N$: Estado del sistema, número de clientes en el sistema\\
%$P_{B}$: Probabilidad de bloqueo.\\
%$\gamma$: Throughput\\
%$u(n)$: Tasa de servicio dependiente del estado\\
%$u(n,M)$: Tasa de servicio dependiente del estado de $M$ etapas\\
%$E(n)$: Número medio de usuarios en la cola.\\
%$E(T)$: Tiempo medio de espera en la cola.

\section{\underline{Fundamentos de la Teoría de Colas}}



\begin{wrapfigure}{L}{0.12\textwidth}
 \vspace{-20pt}
  \begin{center}
\begin{tikzpicture}[scale=0.6, every node/.style={scale = 0.8}]
		
		\draw[-to] (-1,-.5)--(0,-.5) node[midway, fill = white]{$\lambda$};
		\draw[thick] (0,0) -- (1.5,0) -- (1.5,-1) -- (0, -1);
		\draw (.75,-.5) node {Cola};
		\draw (1.5,-.5)--(2,-.5);
		\draw (2.4,-.5) circle (.4 cm);
		\draw (2.4,-.5) node {$\mu$};
		\draw[-to] (2.8,-.5)--(3.3,-.5);
		
		\draw [-to] (0,0.4) -- (1.5,0.4) node[midway, fill = white]{$E(w)$};
		\draw [-to] (1.55,0.4) -- (3,0.4) node[midway, fill = white]{$1/\mu$};
		\draw [-to] (0,-1.5) -- (3,-1.5) node[midway, fill = white]{$E(T)$};
		
	\end{tikzpicture}
  \end{center}
   \vspace{-20pt}
\end{wrapfigure}


%		\begin{center}
%	\begin{tikzpicture}[scale=0.6, every node/.style={scale = 0.8}]
%		
%		\draw[-to] (-1,-.5)--(0,-.5) node[midway, fill = white]{$\lambda$};
%		\draw[thick] (0,0) -- (1.5,0) -- (1.5,-1) -- (0, -1);
%		\draw (.75,-.5) node {Cola};
%		\draw (1.5,-.5)--(2,-.5);
%		\draw (2.4,-.5) circle (.4 cm);
%		\draw (2.4,-.5) node {$\mu$};
%		\draw[-to] (2.8,-.5)--(3.3,-.5);
%		
%		\draw [-to] (0,0.4) -- (1.5,0.4) node[midway, fill = white]{$E(w)$};
%		\draw [-to] (1.55,0.4) -- (3,0.4) node[midway, fill = white]{$1/\mu$};
%		\draw [-to] (0,-1.5) -- (3,-1.5) node[midway, fill = white]{$E(T)$};
%		
%	\end{tikzpicture}
%\end{center}

$\lambda$ y $\mu$ para una cola de un servidor:

	\quad $\lambda \rightarrow \mu$ la cola se hace inestable, para que sea \textbf{estable}: $\lambda < \mu$.\\
	\quad Si $\rho \rightarrow 1$ los tiempos de espera aumentan y la cola se llena rápidamente.\\
	
El \textbf{estado de la cola} es el número de usuarios/paquetes en la cola incluido el que pueda estar en el servidor.\\

Un \textbf{proceso de Poisson} es un caso particular del proceso de Markov en el cual la probabilidad de ocurrencia de un evento en $t + \Delta t$ es independiente de sucesos anteriores.

En una \textbf{distribución de Poisson} la probabilidad de $k$ llegadas es:

	\begin{equation*}
		p(k) = \frac{(\lambda \cdot T)^k}{k!} \cdot e^{- \lambda T} \hspace{10 pt} \rightarrow \hspace{10 pt}  \sum_{k=0}^{\infty} p(k) = 1
	\end{equation*}
	
La \textbf{esperanza} es $E(k)= \lambda \cdot T$ y \textbf{varianza} $\sigma^{2} (k) = \lambda \cdot T$.


\textbf{Tiempo entre eventos sucesivos}, función de densidad de probabilidad:

	\begin{equation*}
		f_{\tau} (\tau) = \lambda e^{- \lambda \tau}	 = \frac{\partial F_{\tau}(x)}{\partial x} \text{,   } E(\tau) = \frac{1}{\lambda} \text{  y } \sigma^{2} (\tau) = \frac{1}{\lambda^{2}}
	\end{equation*}


\textbf{Relación de los procesos exponenciales con los poissonianos}, probabilidad de una llegada en un intervalo:

	\begin{equation*}
		Pr\{ \tau > x \} = p_{x}(0) = e^{-\lambda x}
	\end{equation*}
	
Función de distribución de probabilidad acumulativa, mide la probabilidad de que $\tau$ sea menor o igual a $x$:

	\begin{equation*}
		F_{\tau}(x) = 1 - e^{\lambda x}
	\end{equation*}
	
\textbf{Distribución del tiempo de servicio}:

	\begin{equation*}
	f_{r}(r) = \mu \cdot e^{- \mu r}
	\end{equation*}

\textbf{Propiedad}: Suponer $m$ procesos de Poisson independientes de tasas $\lambda_{1}$, $\lambda_{2}$, ..., $\lambda_{m}$ que se combinan, el proceso resultante será también de Poisson con una tasa $\lambda = \sum_{i = 1}^{m} \lambda_{i}$
	
	\quad \textbf{Notación Kendall}
%	
%	\begin{center}
%		\large{A/B/m/K/c/z}
%	\end{center}
	
\begin{wrapfigure}{L}{0.12\textwidth}
  \begin{center}
\large{A/B/m/K/c/z}
  \end{center}
\end{wrapfigure}
	
	\quad \textbf{A}: Distribución del tiempo entre llegadas\\
	\quad \textbf{B}: Distribución del tiempo de servicio\\
	\quad \textbf{m}: Número de servidores en paralelo en el sistema\\
	\quad \textbf{K}: Capacidad de la cola (incluyendo servidores)\\
	\quad \textbf{c}: Tamaño de la fuente\\
	\quad \textbf{z}: Disciplina de la cola\\

Las distribuciones de llegadas y servicio pueden ser:\\

	\quad \textbf{M}: Exponencial\\
	\quad \textbf{$E_{k}$}: Erlang de $k$ etapas\\
	\quad \textbf{$H_{k}$}: Hiperexponencial de $k$ etapas\\
	\quad \textbf{G}: General\\
	\quad \textbf{D}: Determinística (constante)\\
	
Por defecto para colas A/B/N la cola es infinita y la disciplina FIFO.\\

\textbf{Parámetros medidos en las colas}:\\
	\quad \textbf{Medidas de rendimiento en las colas}: $E(n)$, $E(T)$, $\gamma$ y $P_{N}$.\\
	\quad \textbf{Probabilidad de estado $n(p_{n})$}: Probabilidad de que haya $n$ usuarios en la cola, incluyendo los que hay en los servidores.\\

Cuando el sistema alcanza su estado más estable se supone que la probabilidad de usuarios en el mismo no varía con el tiempo. Con el tiempo los valores de $p_{n}$ se irán aproximando a los valores más estables.\\


	\quad \textbf{M/M/1}

Las \textbf{ecuaciones de balance} permiten calcular la probabilidad de estado $p_{n}$. En el diagrama de estados, el $\#$ es el número de clientes en el sistema.

 \vspace{-10pt}
\begin{center}
\begin{tikzpicture}[scale=0.2, every node/.style={scale = 0.7}]

  \node[state] (A)                    {$0$};
  \node[state]         (B) [right of=A] {$1$};
  \node[state]         (C) [right of=B] {$2$};
  \node[state]         (D) [right of=C] {$...$};
  
 	\path (A) edge [bend left]  node[above] {$\lambda$} (B)
  		(B) edge [bend left]  node[below] {$\mu$} (A);
	\path (B) edge [bend left]  node[above] {$\lambda$} (C)
  		(C) edge [bend left]  node[below] {$\mu$} (B);
	\path (C) edge [bend left]  node[above] {$\lambda$} (D)
  		(D) edge [bend left]  node[below] {$\mu$} (C);
\node[draw] at (6,-2.7) {$(\lambda + \mu) \cdot p_{n}(t) = \lambda \cdot p_{n-1}(t) + \mu \cdot p_{n+1}(t)$};
\end{tikzpicture}
\end{center}
 \vspace{-10pt}

\begin{align*}
  \begin{cases}
      (n-1) & \rightarrow \text{Ha llegado un usuario} \\ 
      n & \rightarrow \text{No ha llegado ni salido nadie} \\
      (n+1) & \rightarrow \text{Ha salido un usuario}
    \end{cases}
\end{align*}

\textbf{Principio de balance estacionario}:\\
	\quad $(\lambda + \mu) p_{n}$: Tasa de salida de $n$.\\
	\quad $\lambda p_{n-1} + \mu p_{n+1}$: Tasa de llegada al estado $n$.\\
	
	\begin{IEEEeqnarray*}{rCl}
		\lambda \cdot p_{n} & = & \mu \cdot p_{n+1}  \hspace{10pt} n = 0,1,2... \hspace{10pt} \sum_{n=0}^{\infty} p_{n} = 1\\
			p_{n+1} & = & \rho \cdot p_{n}
	\end{IEEEeqnarray*}
	
	\begin{IEEEeqnarray*}{rCl}
		p_{0} & = & (1 - \rho) \\
		p_{n} & = & (1 - \rho) \rho^{n}
	\end{IEEEeqnarray*}
	
Para una cola M/M/1/N:
	
	\begin{IEEEeqnarray*}{rCl}
		p_{0} & = & \frac{1 - \rho}{1 - \rho^{N+1}} \\
		p_{n} & = & \frac{1 - \rho}{1 - \rho^{N+1}} \cdot \rho^{n}
	\end{IEEEeqnarray*}

La \textbf{probabilidad de bloqueo} es $P_{N} = p_{n}$, es decir, cuando la última cola se llena.

Tiempo medio por cliente en el sistema:

	\begin{equation*}
	\frac{1}{\mu ( 1 - \rho)}
	\end{equation*}

\textbf{Fórmula de Little}: $\gamma \cdot E(T) = E(n)$.\\
\textbf{Número de usuarios en la cola}: $E(q) = \lambda E(w)$.\\
	
Tiempo medio de espera en cola:

	\begin{equation*}
	\frac{\rho}{\mu (1 - \rho)}
	\end{equation*}

Número medio de usuarios en la cola M/M/1:

	\begin{equation*}
		E(n) = \frac{\rho}{1 - \rho}
	\end{equation*}

\subsection{Cola M/M/N}

Permite procesamiento en paralelo al tener múltiples servidores, el factor de utilización es $\rho = \lambda / m \mu$.

La tasa media de llegadas $\lambda$ es constante e independiente del estado del sistema ($\lambda_{n} = \lambda$).

 \vspace{-5pt}
\begin{center}
\begin{tikzpicture}[node distance = 1.4cm,scale=0.5, every node/.style={scale = 0.8}]

\node[state] (A) {$0$};
\node[state] (B) [right of=A] {$1$};
\node[state] (C) [right of=B] {$2$};
\node[state] (E) [right of=C] {$N-2$};
\node[state] (F) [right of=E] {$N-1$};
\node[state] (G) [right of=F] {$N$};

\path (A) edge [bend left] node[above] {$\lambda$} (B)
(B) edge [bend left] node[below] {$\mu$} (A);
\path (B) edge [bend left] node[above] {$\lambda$} (C)
(C) edge [bend left] node[below] {$2\mu$} (B);
%\path (C) edge [bend left] node[above] {$\lambda$} (D)
%(D) edge [bend left] node[below] {$\mu$} (C);
%\path (D) edge [bend left] node {} (E)
%(E) edge [bend left] node {} (D);
\path (E) edge [bend left] node[above] {$\lambda$} (F)
(F) edge [bend left] node[below] {$\mu (N-1)$} (E);
\path (F) edge [bend left] node[above] {$\lambda$} (G)
(G) edge [bend left] node[below] {$\mu N$} (F);

% edge node {1,1,R} (C)
% (B) edge [loop above] node {1,1,L} (B)
% edge node {0,1,L} (C)
% (C) edge node {0,1,L} (D)
% edge [bend left] node {1,0,R} (E)
% (D) edge [loop below] node {1,1,R} (D)
% edge node {0,1,R} (A)
% (E) edge [bend left] node {1,0,R} (A);
;


%\node[draw] at (3,-2) {$\rho = \frac{\lambda}{N \mu}$};

\end{tikzpicture}
\end{center}
 \vspace{-10pt}

	\begin{IEEEeqnarray*}{rCl}
		p_{n} & =  &\frac{n+1}{N \rho} \cdot p_{n+1} \hspace{10pt} n = 0,..., N-1\\
      p_{n+1} & =  &\rho \cdot  p_{n} \hspace{38pt} n \geq N 
	\end{IEEEeqnarray*}



\subsection{Throughput ($\gamma$)}
%
%	\begin{center}
%	\begin{tikzpicture}
%		\node [draw, align=center] at (0,0) {Sistema\\ de Colas};
%%		\node [draw, align=center, fill={rgb:black,1;white,2}, text = white] at (0,-1) {$F_2$};	
%	
%%		\node [draw, align=center, fill={rgb:black,1;white,2}, text = white] at (2,0) {$P_1$};
%%		\node [draw, align=center, fill={rgb:black,1;white,2}, text = white] at (2,-1) {$P_2$};
%	
%		\draw[to-] (-1.75,0.2) -- node[above left] {$\lambda$} (-0.646,0.2);
%		\draw[-to] (-1.75,-0.2) -- node[above left] {$\lambda P_{B}$} (-0.646,-0.2);
%		\draw (0.67,0) -- node[above left] {$\gamma$} (1.75,0);
%		
%%		\draw (0.25,-1) -- node[above left] {} (1.75,-1);
%%		\draw [blue,  -to, thick] (0.5,0) -- (1.5,0) node [right] {};
%%		\draw [red,  -to, thick] (0.5,0) .. controls (1.10,0) and (0.7,-1) .. (1.5,-1);
%%		\draw [red,  -to, thick] (0.5,0) .. controls (1.05,0) and (0.88,-1) .. (0.5,-1);
%	\end{tikzpicture}
%\end{center}



	\begin{IEEEeqnarray*}{rCl}
		\gamma & = & \lambda \cdot (1 - P_{B}) \\
			     & = & \mu \cdot (1 - p_{0})		
	\end{IEEEeqnarray*}
	
Para una \textbf{cola infinita}: $\gamma = \mu \cdot \rho = \lambda$.
	
Para una \textbf{cola finita}:

	\begin{equation*}
		P_{B} = \frac{1 - \rho}{1 - \rho^{N+1}} \cdot \rho^{N} = P_{N}
	\end{equation*}
	
\textbf{Cola M/M/1/N en función de $\rho$}: Si el sistema está lleno, $N$, no se permite la entrada a nuevos clientes al sistema. La tasa de llegadas no es constante y varía con el tiempo en función de si el sistema está lleno o no:
	
	\begin{equation*}
	\lambda_{ef} = \lambda (1 - p_{k})
	\end{equation*}
	
	\begin{equation*}
	p_{n} = \rho^{n} \cdot p_{0}
	\end{equation*}
	
Para estos sistemas no existe el estado $k+1$.

	\begin{equation*}
		p_{n} = \frac{1 - \rho}{1 - \rho^{N+1}} \cdot \rho^{n}
	\end{equation*}
	
	\textbf{Región de congestión}
	
\textbf{Throughput Normalizado ($\gamma / \mu$)}: Mide la relación entre los usuarios que son atendidos en el sistema por unidad de tiempo y los que potencialmente podrían haber sido atendidos.

	\begin{equation*}
		\frac{\gamma}{\mu} = \frac{1 - \rho^{N}}{1 - \rho^{N+1}} \cdot \rho = \frac{Usuarios atendidos / seg}{Usuarios potenciales / seg}
	\end{equation*}
	
	\text{Insertar Gráfica}
	
\textbf{Tiempo Medio de Estancia en la Cola E(T)}
	
	\begin{equation*}
		E(T) = \frac{E(n)}{\gamma} \hspace{15pt} \rightarrow \hspace{15pt} E(T)_{M/M/1} = \frac{1}{\mu - \lambda}
	\end{equation*}
	
\textbf{Colas dependientes del estado: Proceso de nacimiento y muerte}

En las \textbf{colas dependientes del estado} las tasas de llegada $\lambda$ y las de servicio $\mu$ dependen del estado del sistema, presentes en colas de servidores múltiples: $M/M/m$ y $M/M/1/N$. 

	\begin{IEEEeqnarray*}{rCl}
		\lambda & = & \lambda (n) = \lambda_{n} \rightarrow \text{ Procesos de nacimiento}\\
		\mu & = & \mu (n) = \mu_{n} \rightarrow \text{ Procesos de muerte}
	\end{IEEEeqnarray*}	
	

Un \textbf{proceso de nacimiento} tasas de llegada son función del estado. Un \textbf{proceso de muerte} salidas son función del estado de la cola.


\subsection{Ecuaciones de Balance}

	\begin{equation*}
	p_{n} = \frac{\prod_{i = 0}^{n-1} \lambda_{i}}{\prod_{i = 1}^{n} \mu_{i}} \cdot p_{0} \hspace{10pt}  p_{0} = \frac{1}{\sum_{n = 1}^{N} \prod_{i = 1}^{n} \frac{\lambda_{i-1}}{\mu_{i}} +1}
	\end{equation*}
	
%	\begin{center}
%		\includegraphics[scale = .4]{Graficas}
%	\end{center}

	
\subsection{Cola M/G/1}

\textbf{Fórmulas de Pollaczek Khinchine} para distribuciones M/D/1:

Número medio de usuarios E(n):

	\begin{equation*}
	E(n) = \frac{\rho}{1 - \rho} \cdot \left( 1 - \frac{\rho}{2} \right)% (1 - \mu^{2} \sigma^{2} \right)
	\end{equation*}

Tiempo medio de estancia en la cola E(T):

	\begin{equation*}
	E(T) = \frac{E(n)}{\gamma} = \frac{1}{\mu (1 - \rho)} \cdot \left( 1 - \frac{\rho}{2} \right) %(1 - \mu^{2} \sigma^{2}) \right)
	\end{equation*}



\section{\underline{Modelado de Protocolos de N.Enlace}}

%``\textit{Su función principal es el control del enlace de datos entre dos nodos adyacentes.}''\\

\qquad \textbf{Stop \& Wait}

\text{Suposiciones} para el estudio analítico:\\
	\quad Sólo se permite transmisión de datos \texttt{A} $\rightarrow$ \texttt{B}, \texttt{B} sólo envía \texttt{ACK} o \texttt{NACK}.\\
	\quad \texttt{A} transmite continuamente $\rightarrow$ saturación.\\
	\quad El cálculo del throughput se realiza utilizando el límite máximo.
	


Envía cada trama y espera recibir un ACK o NACK:\\
	\quad Si llega un ACK se libera el buffer que almacenaba el paquete transmitido y se envía otra trama.\\
	\quad Si llega un NACK o expira el temporizador, se retransmite la trama.\\

\begin{center}
\begin{tikzpicture}[scale=0.6, every node/.style={scale = 0.75}]
%\draw[help lines, color=gray!30, dashed] (-4.9,-4.9) grid (4.9,4.9);
%\draw[ultra thick] (0,0)--(8,0) node[right]{$tiempo$};
\draw [-to, thick] (0,0) -- (8,0) node[right]{t};

\node [draw, align=center] at (1,.5) {Trama A1 \\ $t_{I}$};
\node [draw, align=center] at (6,.5) {Trama A1 ó A2 \\ $t_{I}$};

\draw[-to] (0,1.3)--(2,1.3) node[midway, fill = white]{$t_{I}$};
\draw[-to] (4.55,1.3)--(7.4,1.3) node[midway, fill = white]{$t_{I}$};

\draw[-to] (2,.5)--(4.55,.5) node[midway, fill = white]{$t_{out}$};
\draw[-to] (0,-.5)--(4.55,-.5) node[midway, fill = white]{$t_{T}$};

\end{tikzpicture}
\end{center}

\begin{center}
\begin{tabular}{|l|l|l|}

\hline
nº & Tiempo & Probabilidad \\ \hline
1  &    $t_{T} = t_{I} + t_{out}$    &        $1 - p$      \\ \hline
2  & $2 \cdot t_{T} $        &   $p(1 - p)$          \\ \hline
%...   &    ...    &     ...        \\ \hline
n  &    $n \cdot t_{T}$    &      $p^{n-1} (1-p) $       \\ \hline
\end{tabular}
\end{center}

	\quad $t_{I}$: Tiempo requerido para transmitir una trama\\
	\quad $t_{out}$: Tiempo del temporizador de espera de ACK\\ \hspace{25pt} $t_{out} \geq 2 \cdot t_{p} + t_{proc} + t_{s}$\\
		\qquad $t_{p}$: Tiempo de retardo de propagación\\
		\qquad $t_{proc}$: Tiempo de procesado en recepción\\
		\qquad $t_{s}$: Tiempo de transmisión de respuesta\\
	\quad $t_{T}$: Tiempo mínimo entre tramas de datos sucesivas: $t_{T} = t_{I} + t_{out}$\\ 

%Apropiado para la transmisión semiduplex \footnote{Envío bidireccional pero no simultáneo}. No es apropiado para transmisión duplex si el tiempo de propagación es mucho mayor que el tiempo empleado para transmitir la trama.
%
%Debido a retransmisiones provocadas por tramas erróneas: $\gamma_{real} < \gamma_{max}$
	
Tiempo medio para una transmisión correcta:
	
	\begin{IEEEeqnarray*}{rCl}
		t_{v} & = & t_{T} (1-p) + 2 t_{T} (1-p) p + 3 t_{T} (1-p) p^{2} + ... = \\
		        & = & t_{T} (1-p) \sum_{i=1}^{\infty} i \cdot p^{i-1} = t_{T} (1-p) \frac{1}{(1-p)^{2}} = \\
		        & = & \frac{t_{T}}{1-p}
	\end{IEEEeqnarray*}	
	
\textbf{Throughput máximo} ($TH_{real} < TH_{max} \leftarrow$ debido a retransmisiones provocadas por tramas erróneas)
	
	\begin{IEEEeqnarray*}{rCl}
		TH_{max_{S\&W}}  = \lambda_{max_{S\&W}} = \frac{1}{t_{v}} & = & \frac{1 - p}{t_{T}}	\\
								   & = & \frac{1 - p}{a \cdot t_{I}} (paq / seg) \hspace{15 pt} a = \frac{t_{T}}{t_{I}} \geq 1
	\end{IEEEeqnarray*}
	
\textbf{Throughput normalizado} real $\rho$:

	\begin{equation*}
	\rho \equiv \lambda \cdot t_{I} \leq \lambda_{max} \cdot t_{I} = \frac{1 - p}{a} < 1
	\end{equation*}


\qquad \textbf{Go-Back-N}

Envía continuamente las tramas, mejorando el throughput. Es útil para comunicaciones full-duplex.\\

Se realizan las siguientes \textbf{suposiciones}: Los números de secuencia se consideran indefinidos, la retransmisión por \texttt{NACK} o $t_{out}$ se estudian igual, transmisión en régimen de saturación de \texttt{A} $\rightarrow$ \texttt{B}, longitud de trama $t_{I}$ fija y temporizador $t_{out}$ fijo y es múltiplo de $t_{I}$.

\textbf{Mínimo tiempo entre transmisiones} es el tiempo de transmisión de una trama $t_{I}$.

	\begin{center}
	\begin{tikzpicture}[scale=0.6, every node/.style={scale = 0.8}]
%\draw[help lines, color=gray!30, dashed] (-4.9,-4.9) grid (4.9,4.9);
%\draw[thick] (0,0)--(8,0) node[right]{$tiempo$};
\draw [-to, thick] (0,0) -- (10,0) node[right]{t};

\node [draw, align=center] at (1,.28) {A1};
\node [draw, align=center] at (3,.28) {A2};
\node [draw, align=center] at (5,.28) {A3};
\node [draw, align=center] at (7,.28) {A1`};
%
%\draw[] (1.75,.5)--(3,.5);
%\draw[] (3.5,.5)--(6,.5) node[above]{$t_{out}$};
%\draw[] (3.5,.5)--(6,.5);

\draw[-to] (0,1)--(1.9,1) node[midway, fill = white]{$t_{I}$};
\draw[-to] (2,1)--(3.9,1) node[midway, fill = white]{$t_{I}$};
\draw[-to] (4,1)--(5.9,1) node[midway, fill = white]{$t_{I}$};
\draw[-to] (6,1)--(7.9,1) node[midway, fill = white]{$t_{I}$};

\draw[-to] (2,1.5)--(5.9,1.5) node[midway, fill = white]{$t_{out}$};
\draw[-to] (0,-.5)--(5.9,-.5) node[midway, fill = white]{$t_{T}$};

\end{tikzpicture}
	\end{center}
	
	
\begin{center}
\begin{tabular}{|l|l|l|}

\hline
nº & Tiempo & Probabilidad \\ \hline
1  &    $t_{I}$    &        $1 - p$      \\ \hline
2  & $t_{I} + t_{T} $        &   $p(1 - p)$          \\ \hline
%...   &    ...    &     ...        \\ \hline
n  &    $t_{I} + (n - 1) \cdot t_{T}$    &      $p^{n-1} (1-p) $       \\ \hline
\end{tabular}
\end{center}
	
Tiempo medio de transmisión de una trama:
	
	\begin{IEEEeqnarray*}{rCl}
		t_{V} & = & (1-p) \cdot t_{I} + (1-p) \cdot p (t_{T} + t_{I}) + (1-p) \cdot p^{2} (2 t_{T} + \cdot t_{I}) + ... = \\
			 & = & (1-p) \cdot t_{I} + (1-p) t_{T} \sum_{i=1}^{\infty} i \cdot p^{i} + (1-p) \cdot t_{I} \sum_{i = 1}^{\infty} p^{i} = \\
			 & = & (1-p) \cdot t_{I} + a \cdot t_{I} \cdot (1 - p) \frac{p}{(1-p)^{2}} + t_{I} \cdot (1-p) \cdot \frac{p}{1-p} = \\
			 & = & t_{I} ( 1 - p + \frac{a p}{1 - p} + p)  = t_{I} \frac{1 - p + ap}{1 - p} = \\
			 & = & t_{I} \frac{1 + (a - 1) p}{1 - p} \hspace{15pt} a \equiv \frac{t_{T}}{t_{I}} = \frac{t_{I} + t_{out}}{t_{I}} = 1 + \frac{t_{out}}{t_{I}}
	\end{IEEEeqnarray*}

Throughput máximo:

	\begin{equation*}
	\lambda_{max} = \frac{1}{t_{v}} = \frac{1 - p}{t_{I} \cdot (1 + (a - 1) p)}
	\end{equation*}
	
Throughput normalizado:

	\begin{equation*}
	\rho = \lambda \cdot t_{I} \leq \frac{1 - p}{1 + (a-1) p} < 1
	\end{equation*}

	\begin{equation*}
	\lambda_{max_{GBN}} = a \cdot \lambda_{max_{S\&W}}
	\end{equation*}

\textbf{ACKs embebidos en tramas B-A}

\text{Influencia de la tasa de error}

\begin{wrapfigure}{L}{0.12\textwidth}
 \vspace{-20pt}
  \begin{center}
		\begin{tikzpicture}[scale=0.5, every node/.style={scale = 0.8}]
			\fill[gray!40!white] (0,0) rectangle (2,1);
			\draw[white] (1,.5) node{Control};
			\fill[gray!40!black] (2.05,0) rectangle (5,1);
			\draw[white] (3.5,.5) node{Información};
			
			\draw[-to] (0,1.4)--(2,1.4) node[midway, fill = white]{$l^{'}$};
			\draw[-to] (2.05,1.4)--(5,1.4) node[midway, fill = white]{$l$};
		\end{tikzpicture}
  \end{center}
   \vspace{-20pt}
\end{wrapfigure}

%	\begin{center}
%		\begin{tikzpicture}[scale=0.5, every node/.style={scale = 0.8}]
%			\fill[gray!40!white] (0,0) rectangle (2,1);
%			\draw[white] (1,.5) node{Control};
%			\fill[gray!40!black] (2.05,0) rectangle (5,1);
%			\draw[white] (3.5,.5) node{Información};
%			
%			\draw[-to] (0,1.4)--(2,1.4) node[midway, fill = white]{$l^{'}$};
%			\draw[-to] (2.05,1.4)--(5,1.4) node[midway, fill = white]{$l$};
%		\end{tikzpicture}
%	\end{center}
	

	
\textbf{Erorres en enlaces de satélite}, como \textbf{hipótesis} se toma que $p_{b}$ es la probabilidad de error de bit independiente de la posición del bit debido al ruido aleatorio. Probabilidad de error de trama $p$: $p = 1 - (1 - p_{b})^{l + l^{'}} = 1 - q_{b}^{l + l^{'}} $. Si $p_{b} << 1 \rightarrow p \approx (l + l^{'}) p_{b} \rightarrow p << 1$.

\textbf{Errores en enlaces terrestres}, los errores se producen en ráfagas. $p$ es proporcional a $l + l^{'}$, si $p_{b} << 1 \rightarrow p \approx (l + l^{'}) k$.\\

\textbf{Tasa de datos normalizada D/C}, la estación emisora está en saturación ($\lambda_{max}$) y se utiliza el protocolo tipo Go-Back-N.

	\begin{IEEEeqnarray*}{rCl}
		D & = & \lambda_{max} \cdot l = l \cdot \frac{1 - p}{ t_{I} \cdot (1 + (a-1)) p} \hspace {20pt} t_{I} = \frac{l + l^{'}}{C} \\
		C & \equiv & \text{Capacidad del canal}\\
		\frac{D}{C} & = & \frac{l}{l + l^{'}} \cdot  \frac{1 - p}{ t_{I} \cdot (1 + (a-1)) p} \leq 1 \\
		\frac{D}{C} & \approx & \frac{l}{l + l^{'}} \cdot (1-p)
	\end{IEEEeqnarray*}	


\qquad \textbf{Cálculo de la longitud óptima}\\

\textbf{Hipótesis}: $a = 1$ (Stop \& Wait) o $(a-1) \cdot p \ll 1$.\\

Caso general de \textbf{satélite}:

	\begin{IEEEeqnarray*}{rCl}
	l_{opt} & = & \frac{l^{'}}{2} \cdot \left( \sqrt{1 - \frac{4}{l^{'} \cdot L_{n} q_{b}}} - 1 \right) \\
	           & \approx & \sqrt{\frac{l^{'}}{p_{b}}} \hspace{10pt} \text{suponiendo } p_{b} \ll 1 \rightarrow p_{b} \cdot l^{'} \ll 1
	\end{IEEEeqnarray*}	

Caso \textbf{terrestre/satélite} simplificado $p = p_{b} (l + l^{'})$:

	\begin{equation*}
		l_{opt} = \approx \sqrt{\frac{l^{'}}{p_{b}}} - l^{'}
	\end{equation*}
	
\subsection{Protocolo HDLC}

%	\begin{center}
%		\begin{tikzpicture}[scale=0.35, every node/.style={scale = 0.6}]
%			\fill[gray!40!white] (0,0) rectangle (1,1);
%			\draw (.5,.5) node{A};
%			
%			\draw[-to] (.5,0) |- (1.5,-4) node[right] {Direcciones};
%			
%			\fill[gray!40!white] (1.05,0) rectangle (2.05,1);
%			\draw (1.5,.5) node{C};
%			
%			\draw[-to] (1.5,0) |- (2,-1.7) node[right] {Información:};
%			
%			\fill[gray!40!white] (5,-2.2) rectangle (6,-1.2);
%			\draw (5.5,-1.7) node{0};
%			\fill[gray!40!white] (6.05,-2.2) rectangle (7.5,-1.2);
%			\draw (6.8,-1.7) node{N(S)};
%			\fill[gray!40!white] (7.55,-2.2) rectangle (9.05,-1.2);
%			\draw (8.3,-1.7) node{P/F};
%			\fill[gray!40!white] (9.1,-2.2) rectangle (10.5,-1.2);
%			\draw (9.8,-1.7) node{N(R)};
%			
%			\draw[-to] (1.5,0) |- (2,-2.8) node[right] {Supervisión:};
%			
%			\fill[gray!40!white] (5,-3.3) rectangle (6,-2.3);
%			\draw (5.5,-2.8) node{1 0};
%			\fill[gray!40!white] (6.05,-3.3) rectangle (7.5,-2.3);
%			\draw (6.8,-2.8) node{SS};
%			\fill[gray!40!white] (7.55,-3.3) rectangle (9.05,-2.3);
%			\draw (8.3,-2.8) node{P/F};
%			\fill[gray!40!white] (9.1,-3.3) rectangle (10.5,-2.3);
%			\draw (9.8,-2.8) node{N(R)};
%			
%			\fill[gray!40!white] (2.1,0) rectangle (5.1,1);
%			\draw (3.6,.5) node{Información};
%			\fill[gray!40!white] (5.15,0) rectangle (7.15,1);
%			\draw (6.2,.5) node{CRC};
%			
%			\draw[-to] (6.2,0) |- (7,-.5) node[right] {Control de Errores};
%			
%		\end{tikzpicture}
%	\end{center}

%SS $\rightarrow$ tipos de tramas de supervisión: \\
%	\quad RR (\textit{Ready to Receive}): Confirma las tramas hasta $N(R) - 1$.
%	\quad RNR (\textit{Not Ready to Receive}): Confirma las tramas hasta $N(R) - 1$ y establece control de flujo con condición de bloqueo.\\
%	\quad REJ (\textit{Reject}): Confirma las tramas hasta $N(R) - 1$ y retransmisión desde la trama $N(R)$.\\
%	\quad SREJ (\textit{Selective Reject}): Pide retransmisión de la trama $N(R)$.\\
	
%\textbf{Análisis en HDLC/ABM}:\\
%	\quad Se \textbf{diferencia respecto a Go-Back-N} de que tiene numeración de secuencia finita y procedimiento de control de errores específico.\\
%	\quad \textbf{Procedimiento de control de errores}, se define una implementación específica sobre la cual se realiza el análisis del protocolo:\\
%		\qquad Recuperación vía REJ siempre que se pueda acelerar el procedimiento de recuperación. Sólo una vez por la misma trama.\\
%		\qquad Recuperación por temporizador (además de REJ) para las retransmisiones sucesivas y errores no detectados por REJ.\\
%		\qquad Envío de tramas P/F para obligar a respuestas forzadas en situaciones especiales de recuperación.\\
%	
%\textbf{Hipótesis} para el análisis:\\
%		\quad Sólo se transmite en el sentido $A \rightarrow B$.\\
%		\quad $A$ transmite en el régimen de saturación.\\
%		\quad $B$ responde con $RR$ ($ACK$) y $REJ$ ($NACK$).\\
%	
%\textbf{Datos} para el análisis:\\
%	\quad \textbf{Módulo}: Número de tramas con diferente numeración que pueden ser enviadas.\\
%	\quad \textbf{Ventana}: Número máximo de tramas que pueden estar en circulación sin haber sido confirmadas.\\
	\quad \textbf{Longitud de tramas}: Información ($t_{I}$) y de supervisión ($t_{S}$):
		\begin{equation*}
			t_{I} = \frac{l + l^{'}}{C} \hspace{15 pt} t_{S} = \frac{l^{'}}{C}
		\end{equation*}
	\quad \textbf{Retardo de propagación} (incluyendo $t_{proc}$): $t_{p}$\\
	\quad \textbf{Tiempo de confirmación}: $t_{ack} = 2 \cdot t_{p} + t_{s}$\\
	\quad \textbf{Temporizador de reenvío}: $t_{out} = 2 \cdot t_{p} + 2 \cdot t_{I} > t_{ack}$\\	
	

\textbf{TERMINAR}\\

\textbf{Cálculo del Throughput} la \textbf{hipótesis} es que cuando ocurre un error se puede recuperar mediante: \texttt{REJ} (primer caso) o \texttt{TEMPORIZADOR} (siguientes casos).

Throughput máximo:

	\begin{equation*}
	\lambda_{max} = \frac{1}{t_{v}} = \frac{1 - p}{t_{I} (1 + (a - 1)p)}
	\end{equation*}

	\quad $p$: Probabilidad de recibir una trama con error\\
	\quad $T_{1}$: Tiempo aleatorio requerido para la primera retransmisión (puede tomar diferentes valores)\\
	\quad $T_{2}$: Tiempo medio requerido para retransmisiones posteriores (es siempre igual en cada transmisión)\\
	\quad $t_{v}$: Tiempo medio de transmisión virtual de una trama de longitud $t_{I}$.


Tiempo medio de transmisión correcta de una trama:

	\begin{IEEEeqnarray*}{rCl}
		t_{V} & = & t_{I} + E[T_{1}] p + \frac{T_{2} p^{2}}{1-p}
	\end{IEEEeqnarray*}	



\section{\underline{Modelado de N.Red y Control de Flujo}}

%Las redes se diseñan con recursos suficientes para soportar un tráfico nominal. Se pueden producir situaciones de congestión y bloqueo.
%
%\begin{center}
%\begin{tikzpicture}[node distance = 2cm,scale=0.5, every node/.style={scale = 0.55}]
%\tikzstyle{every state}=[fill=gray,draw=none,text=white, scale=0.5, every node/.style={transform shape}]
%
%\node [draw, align=center, fill={rgb:black,1;white,2}, text = white] at (-1.5,0) {A};
%\node [draw, align=center, fill={rgb:black,1;white,2}, text = white] at (5,0) {B};
%
%\draw (1,.5)--(2.,.5) node[midway, fill = white]{\scriptsize{CFL}};
%\draw (-1.5,.5)--(0,.5) node[midway, fill = white]{\scriptsize{CFAR}};
%\draw (0,-1.8)--(3.5,-1.8) node[midway, fill = white]{\scriptsize{CFNO/ND}};
%\draw (-1.5,-2.3)--(5,-2.3) node[midway, fill = white]{\scriptsize{CFEE}};
%
%\node[state] (A) {};
%\node[state] (B) [right of=A] {};
%\node[state] (C) [right of=B] {};
%\node[state] (D) [right of=C] {};
%\node[state] (F) [above of=B] {};
%\node[state] (G) [right of=F] {};
%\node[state] (H) [below of=B] {};
%\node[state] (I) [right of=H] {};
%
%\draw (A) -- (B)
%	 (B) -- (C)
%	 (C) -- (D)
%	 (D) -- (G)
%	 (F) -- (G)
%	 (A) -- (F)
%	 (A) -- (H)
%	 (H) -- (I)
%	 (I) -- (D)
%	 (-1.14,0) -- (A)
%	 (4.66,0) -- (D)
%	 ;
%\end{tikzpicture}
%
%	\textit{Niveles en el control de flujo}
%\end{center}
%
%\textbf{CFL}: Control de Flujo Local\\
%\textbf{CFAR}: Control de Flujo de Acceso a la Red\\
%\textbf{CFEE}: Control de Flujo Entre Extremos\\
%\textbf{CFNO/ND}: Control de Flujo entre Nodos Origen y Destino\\

	\begin{equation*}
		\text{Potencia} = \frac{\text{Caudal}^{\alpha}}{\text{Retardo}}
	\end{equation*}

%\textbf{Modelo del Circuito Virtual}
%
%\begin{center}
%	\begin{tikzpicture}[node distance = 2cm]
%\tikzstyle{every state}=[fill=gray,draw=none,text=white, scale=0.5, every node/.style={transform shape}]
%
%\node[state] (A) {1};
%\node[state] (B) [above right of=A] {2};
%\node[state] (C) [below right of=A] {3};
%\node[state] (D) [above right of = C] {4};
%\node[state] (F) [above right of=D] {5};
%
%
%\draw (A) -- (B)
%	(B) -- (D)
%	(A) -- (D)
%	(A) -- (C)
%	(C) -- (D)
%	(D) -- (F)
%%	 (-1.14,0) -- (A)
%%	 (4.66,0) -- (D)
%	 ;
%\end{tikzpicture}
%\end{center}

\textbf{Retardos medios en redes abiertas sin mecanismos de control de flujo}:\\
	\quad En CV $\rightarrow$ $E(T) = \sum_{i = 1}^{N} \frac{1}{\mu_{i} - \lambda_{i}}$ (ROC)\\
	\quad En Red $\rightarrow$ $E(T) = 1/\gamma \sum_{i = 1}^{N} \frac{\lambda_{i}}{\mu_{i} - \lambda_{i}}$ (NOC)\\
	
\textbf{Mecanismos ROC} 

	\quad \textbf{Ventana deslizante}: CV. Homogéneo (M/M/1 idénticos en serie)

\begin{center}

\begin{tikzpicture}[scale=0.4, every node/.style={scale = 0.6}]
		
		\draw[-to] (-1,-.5)--(0,-.5) node[midway, fill = white]{$\lambda$};
		\draw[thick] (0,0) -- (.5,0) -- (.5,-1) -- (0, -1);
		\draw (.5,-.5)--(1,-.5);
		\draw (1.4,-.5) circle (.4 cm);
		\draw (1.4,-.5) node {$\mu$};
		\draw[-to] (1.8,-.5)--(2.3,-.5);
		
		\draw[thick] (2,0) -- (2.5,0) -- (2.5,-1) -- (2, -1);
		\draw (2.5,-.5)--(3,-.5);
		\draw (3.4,-.5) circle (.4 cm);
		\draw (3.4,-.5) node {$\mu$};
		\draw[-to] (3.8,-.5)--(4.3,-.5);
		
		\draw (5,-.5) node {...};
		
		\draw[thick] (6,0) -- (6.5,0) -- (6.5,-1) -- (6, -1);
		\draw (6.5,-.5)--(7,-.5);
		\draw (7.4,-.5) circle (.4 cm);
		\draw (7.4,-.5) node {$\mu$};
		\draw[-to] (7.8,-.5)--(8.3,-.5);
		
		\draw[*-] (8.6,-.35) |- (5,-2.5);
		
		\draw (5,-2) -| (4.5,-3) -- (5, -3);
		\draw (4.5,-2.5)--(4,-2.5);
		\draw (3.6,-2.5) circle (.4 cm);
		\draw (3.6,-2.5) node {$\mu$};
		\draw[-to] (3.2,-2.5)--(2.7,-2.5);
		
		\draw[-*] (2.6,-2.5) -| (-1.3,-.35);
	\end{tikzpicture}
	 \vspace{-10pt}
\end{center}

\begin{equation*}
	u(n,M) = \frac{n}{n + (M-1)} \cdot \mu
\end{equation*}

\quad $\lambda \rightarrow \infty$ $\gamma = u(N) = \frac{N}{N + M - 1} $ $E(N) = N$ $E(T) = E(n) / \gamma = \frac{N + M -1}{\mu}$\\
\quad $\lambda \approx \mu$ $\gamma = u(N) = \frac{N}{N + M - 1} $ $E(N) = N$ $E(T) = E(n) / \gamma = \frac{N + M -1}{\mu} \cdot \frac{M}{M+1}$\\

	\quad \textbf{CV No Homogéneo}: Tabla cálculo recursivo\\

\textbf{Confirmación al final de la ventana}

\begin{center}
 \vspace{-10pt}
\begin{tikzpicture}[scale=0.4, every node/.style={scale = 0.6}]
		
		\draw[-to] (-1,-.5)--(0,-.5) node[midway, fill = white]{$\lambda$};
		\draw[thick] (0,0) -- (.5,0) -- (.5,-1) -- (0, -1);
		\draw (.5,-.5)--(1,-.5);
		\draw (1.4,-.5) circle (.4 cm);
		\draw (1.4,-.5) node {$\mu$};
		\draw[-to] (1.8,-.5)--(2.3,-.5);
		
		\draw[thick] (2,0) -- (2.5,0) -- (2.5,-1) -- (2, -1);
		\draw (2.5,-.5)--(3,-.5);
		\draw (3.4,-.5) circle (.4 cm);
		\draw (3.4,-.5) node {$\mu$};
		\draw[-to] (3.8,-.5)--(4.3,-.5);
		
		\draw (5,-.5) node {...};
		
		\draw[thick] (6,0) -- (6.5,0) -- (6.5,-1) -- (6, -1);
		\draw (6.5,-.5)--(7,-.5);
		\draw (7.4,-.5) circle (.4 cm);
		\draw (7.4,-.5) node {$\mu$};
		\draw[-to] (7.8,-.5)--(8.3,-.5);
		
		
		\draw[*-to] (8.6,-.35) |- (5,-2.5);
		
		\fill[draw = black, fill = white] (6,-3) rectangle (7,-2);
		\draw (6.5,-2.5) node {$W$};
		
		\draw (5,-2) -| (4.5,-3) -- (5, -3);
		\draw (4.5,-2.5)--(4,-2.5);
		\draw (3.6,-2.5) circle (.4 cm);
		\draw (3.6,-2.5) node {$\mu$};
		\draw[-to] (3.2,-2.5)--(2.7,-2.5);
		
		\draw[-*] (2.6,-2.5) -| (-1.3,-.35);
	\end{tikzpicture}
	 \vspace{-10pt}
\end{center}

\begin{equation*}
	n = W - (i+j)
\end{equation*}




%\textbf{Modelo de ventana deslizante} es un modelo de control según actúa el transmisor al encontrarse con la ventana 
%saturada. Se supone el caso más sencillo, el bloqueo de transmisión.\\

%\textbf{Hipótesis}:\\
%	\quad Los \texttt{ACK}s del CV asociados al control de ventana se supone que se transmiten de vuelta por la red con prioridad máxima.\\
%	\quad Se desprecia el retardo de transmisión de \texttt{ACK}s.\\
%	\quad Se modela el control de ventana deslizante sobre un CV como un \textbf{sistema cerrado}.\\
%	
%	
%	\textbf{IMAGEN CV VENTANA}\\
%	
%La fuente y destino están unidas por una cola artificial (Cola $M+1$) con una tasa de servicio $\lambda$ que coincide con la de llegadas al CV. En situación de máxima sobrecarga hay un número fijo de paquetes $N$ circulando por el sistema cerrado. Si hay $N$ paquetes en el CV estarán en las $M$ colas del sistema, la cola $M+1$ estará vacía (condición de bloqueo). Cuando un paquete alcanza al receptor pasará a la cola $M+1$ ($t_s = 0$) y ésta podrá generar servicios. Si hay menos de $N$ paquetes en la ruta, el resto estarán en la cola $M+1$.\\
%
%\textbf{Teorema de Norton de agregación o descomposición de redes de colas}, en redes con solución en forma de producto puede reemplazarse una subred por una cola compuesta con tasa de servicio dependiente del estado. La red mantiene el mismo comportamiento estadístico extremo a extremo. Da
%
%\textbf{Ecuaciones de estado de la cola del CV}:
%
%	\begin{equation*}
%		p_{n} = \frac{\lambda_{e}^{n}}{\prod_{i = 1}^{n} u(i)} \cdot p_{0} \hspace{5pt} \text{, Condición de normalización} \sum_{n = 0}^{N} p_{n} = 1
%	\end{equation*}
%
%\textbf{Hipótesis} validas para su uso:\\
%	\quad Enlaces con igual capacidad e igual tráfico total.\\
%	\quad Para redes con enlaces de capacidades diferentes ($\mu_{1}$ y $\mu_{2}$) siendo $\mu_{1} \gg \mu_{2}$ el cuello de botella estará en los enlaces de menor capacidad $\mu_{2}$, en los de mayor capacidad $\mu_{1}$ no se producen esperas $\rightarrow$ pueden ser eliminados.\\
%	
%	\quad \textbf{Análisis de un CV homogéneo} $\rightarrow$ Enlaces de igual capacidad $\mu$
%	
%	\begin{equation*}
%		u(n,M) = \frac{n}{n + (M-1)} \cdot \mu \text{ \scriptsize{(Dependiente del estado)}}
%	\end{equation*}
%	
%	\begin{IEEEeqnarray*}{rCl}
%		p_{n} & = & \rho^{n}  \binom{n+M-1}{n} p_{0} \hspace{20pt} \rho = \frac{\lambda_{e}}{\mu}\\
%		p_{0} & = & \frac{1}{\sum_{n = 0}^{N} \rho^{n} \binom{n+M-1}{n}}
%	\end{IEEEeqnarray*}	
%	
%\textbf{Cálculo de u(n,M) en un CV homogéneo}, se cortocircuita la red de colas entre $S$ y $D$ y se estudia su throughput $u(n)$ para los $n$ diferentes estados.\\
%
%$u(n) \leq \mu \rightarrow$ La probabilidad de cola no vacía es igual en todas las colas.\\
%
%	\quad $lambda \gg \mu$:
%	
%	\begin{equation*}
%		E(n) = N \rightarrow \gamma = \frac{N}{N+M-1} \cdot \mu \hspace{15pt} E(T) = \frac{E(n)}{\gamma} = \frac{N+M-1}{\mu}
%	\end{equation*}	
%	
%	\quad $\lambda \approx \mu$ (congestión), similar al caso anterior excepto con $M+1$ colas.\\
%	
%\textbf{Análisis simplificado del control de ventana deslizante}, realizado en la región de congestión ($\lambda \geq \mu$).\\
%	\quad Throughput: Promedio ponderado de las $N$ tasas de servicio posibles, 
%		\begin{equation*}
%			\gamma = \sum_{n=1}^{N} u(n)\cdot p_{n}
%		\end{equation*}
%	\quad Retardo extremo a extremo (Little):
%		\begin{equation*}
%			E(T) = \frac{E(n)}{\gamma} = \frac{\sum_{n=1}^{N} n \cdot p_{n}}{\gamma}
%		\end{equation*}
%
%\textbf{\Large FALTAN COSAS POR PONER}\\
%
%\textbf{Redes de Colas} sólo se modelizan las colas de los enlaces de salida, se desprecia el efecto de las colas en los nodos de procesamiento (útil con soluciones en forma de producto).\\
%	\quad \textbf{Redes abiertas}: El tráfico entra y sale de la red, el flujo se conserva: $\lambda_{S} = \lambda_{D}$\\
%	\quad \textbf{Redes cerradas}: El tráfico circula por la red de forma indefinida. No hay llegadas ni salidas. Una red abierta se puede convertir en cerrada poniendo las tasas de llegadas y salidas a cero.\\
%	
%\textbf{\Large Algoritmos de Enrutamiento}\\
%
%	\quad \textbf{Algoritmo A (camino más corto)} $\rightarrow$ Necesita conocimiento topológico global de la red y es centralizado.
%
%	
%	\begin{itemize}
%		\item Incorporar nodos paso a paso
%		\item Construir un árbol de camino más corto, partiendo del nodo fuente y englobando a todos paso a paso.
%		\item En cada paso $k$-ésimo se calcula el camino más corto de los $k$ nodos más cercanos a la fuente.
%	\end{itemize}
%	
%	\qquad $D(v)$: Distancia en coste desde la fuente al nodo $v$.\\
%	\qquad $l(i,j)$: Coste entre el nodo $i$ y $j$.
%	
%	\begin{enumerate}
%		\item \textbf{Inicialización}: $N = \{ s\}$ $\rightarrow$ $s$ es el nodo fuente. Para el resto de nodos $v$ que no están en $N$ se construye una tabla de distancias $D(v) = l(s,v)$.
%		\item \textbf{Repetición} hasta que todos los nodos están en $N$. Se introduce en $N$ el nodo $w$ que todavía no esté en $N$ y cuya distancia $D(w)$ sea mínima. Actualizar las distancias $D(v)$ mediante: $D(v) = \text{min} [ D(v), D(w) + l(w,v) ]$.
%	\end{enumerate}
%	
%	\quad \textbf{Algoritmo B}: Camino más corto, centralizado o descentralizado.\\
%	
%	\qquad $D(v)$: Distancia más corta en coste desde $v$ al destino.
%	\qquad $n$: Identificación del siguiente nodo al $v$ en el camino más corto calculado anteriormente.
%	
%	\begin{enumerate}
%		\item \textbf{Inicialización}: $D(d) = 0$, siendo $d$ el destino, el resto de nodos $v$ se etiquetan con $(.,\infty)$.
%		\item \textbf{Repetición}: Hasta que no se produzcan cambios. Se actualizan las distancias $D(v)$ para todos los nodos $v$ utilizando las distancias $D(w)$ de los nodos $w$ vecinos: $D(v) = \text{min}_{w} [D(w) + l(v,w)]$
%	\end{enumerate}
%
%	\quad \textbf{Algoritmo B Descentralizado} TERRERERMEINTAR
%
%
%
%


%\newpage
%
%\section{\underline{Demostraciones}}
%
%\subsection{Esperanza y varianza en eventos Poissonianos}
%
%	\begin{IEEEeqnarray*}{rCl}
%		E(k) & = & \sum_{k=0}^{\infty} k \cdot p(k) = \sum_{k = 0}^{\infty} k \cdot \frac{{(\lambda \cdot T)}^{k}}{k!} \cdot e^{- \lambda T} = \\
%			& = &   e^{- \lambda T} \cdot \left( \sum_{k = 1}^{\infty} \lambda T \cdot \frac{ k \cdot{(\lambda \cdot T})^{k-1}}{k \cdot (k-1)!} \right) = e^{- \lambda T} \cdot \lambda \cdot T \cdot e^{\lambda T} = \\
%			& = & \lambda \cdot T
%	\end{IEEEeqnarray*}	
%	
%	\begin{IEEEeqnarray*}{rCl}
%		\sigma^{2}_{k} & = & E \left[ (k - E[k])^{2} \right] = E[k^{2}] - E^{2}[k] = \sum_{k = 0}^{\infty} k^{2} \cdot p(k) - (\lambda \cdot T)^{2} = \\
%				       & = & \sum_{k = 0}^{\infty} k^{2} \cdot \frac{{(\lambda \cdot T)}^{k}}{k!} e^{-\lambda T} - \lambda^{2} T^{2} = \\
%				       & = & e^{- \lambda T} \cdot \left( \sum_{k = 1}^{\infty} \lambda T \cdot \frac{ k \cdot{(\lambda \cdot T})^{k-1}}{k \cdot (k-1)!} \right) - \lambda^{2} T^{2} = \\
%				       & = & \lambda T e^{- \lambda T} \cdot \left( \sum_{k = 1}^{\infty} \frac{ (k-1+1) \cdot{(\lambda \cdot T})^{k-1}}{(k-1)!} \right) - \lambda^{2} T^{2} = \\
%				       & = & \lambda T e^{- \lambda T} \cdot \left( \sum_{k = 1}^{\infty}  \frac{ (k-1) \cdot{(\lambda \cdot T})^{k-1}}{(k-1)(k-2)!} + \sum_{k=1}^{\infty} \frac{(\lambda T)^{k-1}}{(k-1)!} \right) - \lambda^{2} T^{2} = \\
%				       & = & \lambda T e^{- \lambda T} \cdot \left( \lambda T \cdot  \sum_{k = 1}^{\infty} \frac{{(\lambda \cdot T})^{k-2}}{(k-2)!} + e^{\lambda T}\right) - \lambda^{2} T^{2} =\\
%				       & = & \lambda T e^{-\lambda T} (\lambda T e^{ \lambda T} + e^{\lambda T}) - \lambda^{2} T^{2} =\\
%				       & = & \lambda \cdot T
%	\end{IEEEeqnarray*}	
%	
%\subsection{Demostración de la distribución de Poisson}
%
%\subsection{Stop \& Wait Tiempo medio para una transmisión correcta}
%
%	\begin{IEEEeqnarray*}{rCl}
%		t_{v} & = & t_{T} (1-p) + 2 t_{T} (1-p) p + 3 t_{T} (1-p) p^{2} + ... = \\
%		        & = & t_{T} (1-p) \sum_{i=1}^{\infty} i \cdot p^{i-1} = t_{T} (1-p) \frac{1}{(1-p)^{2}} = \\
%		        & = & \frac{t_{T}}{1-p}
%	\end{IEEEeqnarray*}	
%
%\subsection{Go Back N Tiempo medio para  una transmisión correcta}
%
%\begin{IEEEeqnarray*}{rCl}
%		t_{V} & = & (1-p) \cdot t_{I} + (1-p) \cdot p (t_{T} + t_{I}) + (1-p) \cdot p^{2} (2 t_{T} + \cdot t_{I}) + ... = \\
%			 & = & (1-p) \cdot t_{I} + (1-p) t_{T} \sum_{i=1}^{\infty} i \cdot p^{i} + (1-p) \cdot t_{I} \sum_{i = 1}^{\infty} p^{i} = \\
%			 & = & (1-p) \cdot t_{I} + a \cdot t_{I} \cdot (1 - p) \frac{p}{(1-p)^{2}} + t_{I} \cdot (1-p) \cdot \frac{p}{1-p} = \\
%			 & = & t_{I} ( 1 - p + \frac{a p}{1 - p} + p)  = t_{I} \frac{1 - p + ap}{1 - p} = \\
%			 & = & t_{I} \frac{1 + (a - 1) p}{1 - p}
%	\end{IEEEeqnarray*}
%

\end{multicols}
\end{document}