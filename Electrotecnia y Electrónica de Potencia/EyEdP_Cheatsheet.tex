\documentclass[10pt,landscape]{article}
\usepackage{multicol}
\usepackage{calc}
\usepackage[landscape]{geometry}
\usepackage{amsmath,amsthm,amsfonts,amssymb}
\usepackage{color,graphicx,overpic}
\graphicspath{ {images/} }
\usepackage{hyperref}
\usepackage{esint}
\usepackage{bm}
\usepackage{relsize}
\usepackage{datetime}
\usepackage[utf8] {inputenc}
\usepackage[spanish, activeacute] {babel}
\usepackage{IEEEtrantools}
\usepackage{framed}

% This sets page margins to .5 inch if using letter paper, and to 1cm
% if using A4 paper. (This probably isn't strictly necessary.)
% If using another size paper, use default 1cm margins.
\ifthenelse{\lengthtest { \paperwidth = 11in}}
    { \geometry{top=.5in,left=.5in,right=.5in,bottom=.5in} }
    {\ifthenelse{ \lengthtest{ \paperwidth = 297mm}}
        {\geometry{top=1cm,left=1cm,right=1cm,bottom=1cm} }
        {\geometry{top=1cm,left=1cm,right=1cm,bottom=1cm} }
    }

% Turn off header and footer
\pagestyle{empty}

% Redefine section commands to use less space
\makeatletter
\renewcommand{\section}{\@startsection{section}{1}{0mm}%
                                {-1ex plus -.5ex minus -.2ex}%
                                {0.5ex plus .2ex}%x
                                {\normalfont\large\bfseries}}
\renewcommand{\subsection}{\@startsection{subsection}{2}{0mm}%
                                {-1explus -.5ex minus -.2ex}%
                                {0.5ex plus .2ex}%
                                {\normalfont\normalsize\bfseries}}
\renewcommand{\subsubsection}{\@startsection{subsubsection}{3}{0mm}%
                                {-1ex plus -.5ex minus -.2ex}%
                                {1ex plus .2ex}%
                                {\normalfont\small\bfseries}}
\makeatother

\newcommand{\Lagr}{\mathcal{L}}

% Define BibTeX command
\def\BibTeX{{\rm B\kern-.05em{\sc i\kern-.025em b}\kern-.08em
    T\kern-.1667em\lower.7ex\hbox{E}\kern-.125emX}}

% Don't print section numbers
\setcounter{secnumdepth}{0}


\setlength{\parindent}{0pt}
\setlength{\parskip}{0pt plus 0.5ex}

%My Environments
\newtheorem{example}[section]{Example}
% ---------------------------------------------------------------

\begin{document}
\raggedright
\footnotesize
\begin{multicols}{3}


% multicol parameters
% These lengths are set only within the two main columns
%\setlength{\columnseprule}{0.25pt}
\setlength{\premulticols}{1pt}
\setlength{\postmulticols}{1pt}
\setlength{\multicolsep}{1pt}
\setlength{\columnsep}{2pt}

\begin{framed}
	\begin{center}
    	\Large{\underline{EyEdP}} \\
    	\scriptsize{3º de Ingeniería de Telecomunicaciones | UPV/EHU}\\
     	%Actualizado por última vez el \today \\
     	"\textsl{Under-promise and over-deliver}." \\
     	%\hspace{5 pt} \\
     	\small{\textbf{Javier de Martín Gil -- 2016}}
	\end{center}
\end{framed}

%
% Cheatsheet code below 
%                                                      

\section{\underline{Circuitos Monofásicos}}

Representación de funciones sinusoidales:

\begin{IEEEeqnarray*}{rCl}
	\underline{I} & = & X \angle{\theta} \\
	\underline{i} & = & X \cdot \sqrt{2} \cdot e^{j \cdot (\omega t + \theta)} \\
	i & = & X \cdot \sqrt{2} \cdot \cos{\omega t + \theta})	
\end{IEEEeqnarray*}

\subsection{Régimen Alterno Sinusoidal Permanente}

Clasificación de los circuitos en función de su impedancia:

\begin{itemize}
	\item Circuito Abierto ($Z = \infty$)
	\item TERMINAR
\end{itemize}

Según el \textbf{argumento de la impedancia}:

\begin{itemize}
	\item Inductivos ($\phi > 0$): La intensidad está retrasada con respecto a la tensión.
	\item Resistivos ($\phi = 0$): Tensión e intensidad están siempre en fase.
	\item Capacitivos ($\phi < 0$): La intensidad está adelantada con respecto a la tensión.
\end{itemize}

\subsubsection{Circuito RLC Serie}

Un \textbf{circuito serie RLC} tiene carácter inductivo o capacitivo en función del predomino de un efecto sobre el otro. SI en un circuito se modficica la frecuencia hasta que los valores de las reactancias inductiva y capacitiva se igualan, el circuito entra en \textbf{resonancia}.


\subsection{\underline{Circuitos Trifásicos}}

El conjunto de tensiones forma un \textbf{sistema equilibrado} de tensiones, ya que está formado por 3 tensiones sinusoidales del mismo valor eficaz, la misma frecuencia y desfasados 120º entre sí. Según cómo se coloquen las fases hay dos secuencias:

\begin{itemize}
	\item \textbf{Secuencia Directa o Positiva}: 1-2-3
	\item \textbf{Secuencia Inversa o Negativa}: 1-3-2
\end{itemize}

Conexiones en circuitos trifásicos:

\begin{itemize}
	\item \textbf{Triángulo}: Sistema trifásico a 3 hilos
	\item \textbf{Estrella}: Con neutro (sistema trifásico a 4 hilos) y sin neutro (sistema trifásico a 3 hilos).
\end{itemize}

\subsection{Conexión en Estrella}

\begin{itemize}
	\item \textbf{Conexión Simple}: Tensión entre conductor de fase y punto neutro.
	\item \textbf{Tensión Compuesta}: Tensión entre los conductores de fase.
\end{itemize}

Para un sistema equilibrado la corriente por una fase:

\begin{equation*}
	\underline{V}_1 = \frac{\underline{U}_{xy}}{\sqrt{3}} \angle{-30º}
\end{equation*}

\begin{itemize}
	\item \textbf{Corrientes de Fase}: Corrientes que circulan por cada fase de la carga.
	\item \textbf{Corrientes de Línea}: Corrientes externas que circulan por la línea de acceso.
	\item \textbf{En la conexión en estrella coinciden las corrientes de fase y de línea.}
\end{itemize}

\subsection{Conexión en Triángulo}

\begin{itemize}
	\item La tensión en las cargas es la tensión compuesta.
	\item La intensidad en cada elemento es igual a la de línea dividida por $\sqrt{3}$.
	\item Las tensiones de cada elemento están desfasadas $120º$ entre sí.
	\item Las corrientes de cada elemento están desfasadas $120º$ entre sí.
	\item \textbf{En la conexión en triángulo coinciden las tensiones de fase y de línea.}
\end{itemize}

\subsection{Relaciones en Circuitos Trifásicos}

\begin{itemize}
	\item \textbf{Tensión de Fase}: Diferencia de potencial que existe en cada una de las ramas monofásicas de un sistema trifásico.
	\item \textbf{Tensión de Línea o Compuesta}: Diferencia de potencial que existe entre dos conductores de línea o entre dos terminales de fase.
	\item \textbf{Intensidad de Fase}: Corriente que circula por cada una de las ramas de un sistema trifásico.
	\item \textbf{Intensidad de Línea}: Corriente que circula por cada uno de los conductores de línea.
\end{itemize}

\begin{equation*}
	\underline{Z}_Y = \frac{\underline{Z}_\triangle}{3} \hspace{20pt} \underline{Z}_\triangle = 3 \cdot \underline{Z}_Y
\end{equation*}


\section{\underline{CIRCUITOS MAGNÉTICOS}}

\textbf{Campo Magnético}: 

	\begin{equation*}
		B = \frac{\phi}{S}
	\end{equation*}

\subsection{Leyes Fundamentales}

\textbf{Hopkinson} $\rightarrow$ $F_{mm} = \phi \cdot R$ \\

\textbf{Lenz} $\rightarrow$ $fem = -N \frac{\partial\phi}{\partial t}$ \\

\textbf{Faraday} $\rightarrow$ $V_{AB} = V B L \sin \theta$ \\

\textbf{Laplace} $\rightarrow$ $F = I B L \sin \phi $

\subsection{Transformadores}

\subsubsection{Transformador monofásico ideal en el vacío}

Relación de Transformación:

	\begin{equation*}
		a = \frac{E_1}{E_2} = \frac{N_1}{N_2}
	\end{equation*}

\subsubsection{Transformador monofásico ideal en carga}

	\begin{equation*}
		\frac{1}{a} = \frac{i_1}{i_2} = \frac{N_2}{N_1}
	\end{equation*}

\subsubsection{Transformador monofásico real en el vacío}

	\begin{equation*}
		V_1 - E_1 = (R_1 + j X_1) I_{10}
	\end{equation*}

	\begin{equation*}
		a = \frac{E_1}{E_2} = \frac{N_1}{N_2}
	\end{equation*}
	
\subsubsection{Transformador monofásico real en carga}

	\begin{equation*}
		i = \frac{S_1}{S_N} = \frac{I_1}{I_{1N}} \approx \frac{I_2}{I_{2N}}
	\end{equation*}	

\subsubsection{Balance de potencias}

	\begin{equation*}
		\eta = \frac{P_2}{P_1} = \frac{P_2}{P_2 + P_{Fe} + P_{Cu}} = \frac{i S_N cos \phi}{i S_N cos \phi + P_{Fe} + R_e (i I_{1N})^2}
	\end{equation*}

\subsection{Transformadores Trifásicos}

%\begin{table}[]
%\centering
%\caption{My caption}
%\label{my-label}
\begin{tabular}{|l|l|l|l|}
\hline
Dd & $\bigtriangledown$  & $\bigtriangledown$ & $RT = \frac{U_1}{U_2} = \frac{V_1}{V_2} = \frac{N_1}{N_2} = RE$ \\ \hline
Dy & $\bigtriangledown$ & Y                  & $RT = \frac{U_1}{U_2} = \frac{V_1}{\sqrt{3} V_2} = \frac{N_1}{N_2} = \frac{RE}{\sqrt{3}}$ \\ \hline
Yd & Y                  & $\bigtriangledown$ & $RT = \frac{U_1}{U_2} = \frac{\sqrt{3} V_1}{V_2} = \frac{N_1}{N_2} = RE \cdot \sqrt{3}$ \\ \hline
Yy & Y                  & Y                  & $RT = \frac{U_1}{U_2} = \frac{V_1}{V_2} = \frac{N_1}{N_2} = RE$ \\ \hline
\end{tabular}
%\end{table}

Por ejemplo, para un transformador trifásico:

	\begin{equation*}
		Yd5 \rightarrow 5 \cdot 30º (desfase)
	\end{equation*}


%\vfill


\end{multicols}
\end{document}