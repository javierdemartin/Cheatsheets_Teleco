% !TEX encoding = UTF-8 Unicode
\documentclass[10pt,landscape]{article}
\usepackage{multicol}
\usepackage{calc}
\usepackage[landscape]{geometry}
\usepackage{amsmath,amsthm,amsfonts,amssymb}
\usepackage{color,graphicx,overpic}
\graphicspath{ {images/} }
\usepackage{hyperref}
\usepackage{esint}
\usepackage{bm}
\usepackage{relsize}
\usepackage{tikz}
\usepackage{circuitikz}
\usepackage{datetime}
\usepackage{pgfplots}
\usepackage[utf8] {inputenc}
\usepackage[spanish, activeacute] {babel}
\usepackage{IEEEtrantools}
\usepackage{framed}

\usepackage{fullpage}
\usepackage{caption}
\usepackage{subcaption}
\usepackage{enumerate}

\usepackage{draftwatermark}
\SetWatermarkText{Javier de Martí­­n}
\SetWatermarkScale{4.8}

% This sets page margins to .5 inch if using letter paper, and to 1cm
% if using A4 paper. (This probably isn't strictly necessary.)
% If using another size paper, use default 1cm margins.
\ifthenelse{\lengthtest { \paperwidth = 11in}}
    { \geometry{top=.5in,left=.5in,right=.5in,bottom=.5in} }
    {\ifthenelse{ \lengthtest{ \paperwidth = 297mm}}
        {\geometry{top=1cm,left=1cm,right=1cm,bottom=1cm} }
        {\geometry{top=1cm,left=1cm,right=1cm,bottom=1cm} }
    }

% Turn off header and footer
\pagestyle{empty}

% Redefine section commands to use less space
\makeatletter
\renewcommand{\section}{\@startsection{section}{1}{0mm}%
                                {-1ex plus -.5ex minus -.2ex}%
                                {0.5ex plus .2ex}%x
                                {\normalfont\large\bfseries}}
\renewcommand{\subsection}{\@startsection{subsection}{2}{0mm}%
                                {-1explus -.5ex minus -.2ex}%
                                {0.5ex plus .2ex}%
                                {\normalfont\normalsize\bfseries}}
\renewcommand{\subsubsection}{\@startsection{subsubsection}{3}{0mm}%
                                {-1ex plus -.5ex minus -.2ex}%
                                {1ex plus .2ex}%
                                {\normalfont\small\bfseries}}
\makeatother



\newcommand{\Lagr}{\mathcal{L}}

% Define BibTeX command
\def\BibTeX{{\rm B\kern-.05em{\sc i\kern-.025em b}\kern-.08em
    T\kern-.1667em\lower.7ex\hbox{E}\kern-.125emX}}

% Don't print section numbers
\setcounter{secnumdepth}{0}


\setlength{\parindent}{0pt}
\setlength{\parskip}{0pt plus 0.5ex}

%My Environments
\newtheorem{example}[section]{Example}
% ---------------------------------------------------------------

\begin{document}
\raggedright
\footnotesize
\begin{multicols}{3}


% multicol parameters
% These lengths are set only within the two main columns
%\setlength{\columnseprule}{0.25pt}
\setlength{\premulticols}{1pt}
\setlength{\postmulticols}{1pt}
\setlength{\multicolsep}{1pt}
\setlength{\columnsep}{2pt}

\begin{framed}
	\begin{center}
    	\Large{\underline{Electrónica de Circuitos}} \\
    	\scriptsize{3º Ingenierí­a de Telecomunicaciones | UPV/EHU}\\
     	%Actualizado por última vez el \today \\
     	"\textsl{Under-promise and over-deliver}." \\
     	%\hspace{5 pt} \\
     	\small{\textbf{Javier de Martí­n -- 2016/17}}
	\end{center}
\end{framed}

%
% Cheatsheet code below 
%                                                      

	\vspace{-15pt}

\section{\underline{Transistor BJT}}

\subsection{Parámetros}

$h_{ix} (\Omega)$: Impedancia de entrada\\

$h_{rx}$: Reverse voltage ratio \\

$h_{fx}$ Forward current transfer ratio\\

$h_{ox} (\Omega^{-1})$: Admitancia de salida\\ 


\begin{tabular}{| c | c | c |}
\hline
Base Común                              & Emisor Común & Colector Común \\ \hline
    $\displaystyle h_{ib} = \frac{v_{eb}}{i_e}$       &  $\displaystyle h_{ie} = \frac{v_{be}}{i_b} = \frac{V_{T} \beta}{I_{C}}$            &   $\displaystyle h_{ic} = \frac{v_{bc}}{i_b}$             \\ \hline
    $\displaystyle h_{rb} = \frac{v_{eb}}{v_{cb}}$   &   $\displaystyle h_{re} = \frac{v_{be}}{v_{ce}}$           &   $\displaystyle h_{rc} = \frac{v_{bc}}{v_{ec}}$             \\ \hline
    $\displaystyle h_{fb} = \frac{i_c}{i_e}$       &  $\displaystyle h_{fe} = \frac{i_{c}}{i_{b}} $            &    $\displaystyle h_{fc} = \frac{i_{e}}{i_{b}}$            \\ \hline
     $\displaystyle h_{ob} = \frac{i_c}{v_{cb}}$      &       $\displaystyle h_{oe} = \frac{i_{c}}{v_{ce}}$       &     $\displaystyle h_{oc} = \frac{i_{e}}{v_{ec}}$           \\ \hline
\end{tabular}

\subsection{Regiones Operativas}

	\vspace{-20pt}

	\begin{center}
		\begin{tikzpicture}[scale = 0.6, transform shape]
		
		    \draw (-0.2,0) -- (2.5,0) node[right] {$I_C$};
    		\draw (0,-0.2) -- (0,2.5) node[above] {$V_{CE}$};
    		
    		\node[text width=3cm] at (2.5,1) {Activa};
    		\node[text width=3cm] at (2.3,0.2) {Saturación};
    		\node[text width=3cm, rotate = 90] at (.1,2.2) {Corte};
    		
		\end{tikzpicture}
	\end{center}
	
	\vspace{-20pt}

\begin{itemize}
	\item \textbf{Región Activa}: La corriente de colector $I_C$ depende directamente de la corriente de base $I_B$, de la ganancia de corriente $\beta$ y de las resistencias conectadas al colector y emisor. En esta región se produce amplificación de la señal.
	\item \textbf{Región Inversa}: 
	\item \textbf{Región de Saturación}: La corriente $I_C = I_E = I_{max}$. La corriente depende del voltaje de alimentación del circuito y de las resistencias conectadas al colector y emisor. Este modo aparece cuando la corriente de base es lo suficientemente grande como para inducir una corriente de colector $\beta$ veces más grande.
	\item \textbf{Región de Corte}: La corriente $I_C = I_E = 0$. El voltaje $V_{CE}$ es el de alimentación del circuito, al no haber corriente circulando no hay caí­da tensión. Este modo aparece, normalmente, cuando $I_B = 0$.
\end{itemize}

\subsection{Configuraciones de Montaje}

\subsubsection{Base Común}

\begin{center}
\begin{circuitikz}[scale=.5,american voltages, american currents, transform shape]
	\draw (0,0) node[npn](npn)	{}
		(npn.C) node[anchor=east] {C} % Collector
    	(npn.B) node[anchor=north west] {B} % Gate
        (npn.E) node[anchor= north east] {E} % Emisor
        (npn.B) -- (-1, 0)
        (-1, 0) [short, -*] to (-1, 1)
        (-1, 0) [short, -*] to (-1, -1)
        (npn.E) [short, -*] to (0, -1) 
        (npn.C) [short, -*] to (0, 1);
	%\draw (1,0) -- (0,0) -- (0,1);
	%\draw (-1, 0) -- (pnp.C);
	;
	
	
	\draw (2,0) to [short,i_=$i_b$, o-] (3,0)
				to [short, -] (3,-1)
				to [generic, l=$h_{ib}$] (3,-2)
				to [short] (3,-3) -- (5,-3) -- (5,-2)
				(5,-1) to [I, i= $h_{fe} \cdot i_e$] (5,-2)
				to [short] (5,0)
				to [short, i<_=$i_c$, -o] (6,0)
				;		
			\draw (4,-3) to [short, -o] (4,-4) node[anchor=east] {$B$};
			\draw (2,0) node[anchor=east] {$E$};
			\draw (6,0) node[anchor=west] {$C$};
\end{circuitikz}
\end{center}

%	\begin{center}
%		\begin{circuitikz}[scale=.5,american voltages, american currents, transform shape]
%			\draw (0,0) to [short,i_=$i_b$, o-] (1,0)
%				to [short, -] (1,-1)
%				to [generic, l=$h_{ib}$] (1,-2)
%				to [short] (1,-3) -- (3,-3) -- (3,-2)
%				(3,-1) to [I, i= $h_{fe} \cdot i_e$] (3,-2)
%				to [short] (3,0)
%				to [short, i<_=$i_c$, -o] (4,0)
%				;		
%			\draw (2,-3) to [short, -o] (2,-4) node[anchor=east] {$B$};
%			\draw (0,0) node[anchor=east] {$E$};
%			\draw (4,0) node[anchor=west] {$C$};
%		\end{circuitikz}
%	\end{center}

\begin{itemize}
	\item Baja impedancia de entrada.
	\item Alta impedancia de salida
	\item Ganancia unidad, o menor, de corriente.
	\item Ganancia alta de tensión.
\end{itemize}


\subsubsection{Emisor Común}

%\begin{center}
%\begin{circuitikz}[scale=.5,american voltages, american currents, transform shape]
%	\draw (0,0) node[npn](npn)	{}
%		(npn.C) node[anchor=east] {C} % Collector
%    	(npn.B) node[anchor=north east] {B} % Gate
%        (npn.E) node[anchor= east] {E} % Emisor
%        (npn.B) [short, -*] to (-1, 0)
%        (npn.E) -- (0, -1) [short, -*] to (1, -1)
%        (0, -1) [short, -*] to (-1, -1)
%        (npn.C) -- (0, 1) [short, -*] to (1, 1);
%	%\draw (1,0) -- (0,0) -- (0,1);
%	%\draw (-1, 0) -- (pnp.C);
%\end{circuitikz}
%\end{center}

%	\begin{center}
%		\begin{circuitikz}[scale=.5,american voltages, american currents, transform shape]
%			\draw (0,0) node[npn](npn)	{}
%		(npn.C) node[anchor=east] {C} % Collector
%    	(npn.B) node[anchor=north east] {B} % Gate
%        (npn.E) node[anchor= east] {E} % Emisor
%        (npn.B) [short, -*] to (-1, 0)
%        (npn.E) -- (0, -1) [short, -*] to (1, -1)
%        (0, -1) [short, -*] to (-1, -1)
%        (npn.C) -- (0, 1) [short, -*] to (1, 1);
%	%\draw (1,0) -- (0,0) -- (0,1);
%	%\draw (-1, 0) -- (pnp.C);
%		
%			\draw (0,0) to [short,i_=$i_b$, o-] (1,0)
%				to [short, -] (1,-1)
%				to [generic, l=$h_{ie}$] (1,-2)
%				to [short] (1,-3) -- (3,-3) -- (3,-2)
%				(3,-1) to [I, i= $h_{fe} \cdot i_b$] (3,-2)
%				to [short] (3,0)
%				to [short, i<_=$i_c$, -o] (4,0)
%				;		
%			\draw (2,-3) to [short, -o] (2,-4) node[anchor=east] {$E$};
%			\draw (0,0) node[anchor=east] {$B$};
%			\draw (4,0) node[anchor=west] {$C$};
%			
%			\draw (5,0) to [short,i_=$i_b$, o-] (6,0)
%				to [short, -] (6,-1)
%				to [generic, l=$h_{ie}$] (6,-2)
%				to [short] (6,-3) -- (8,-3) -- (8,-2)
%				(8,-1) to [I, i= $h_{fe} \cdot i_b$] (8,-2)
%				to [short] (8,0)
%				to [short] (10,0)
%				to [short, , i<_=$i_c$, -o] (11,0)
%				;
%			\draw (8,-3) to [short] (10,-3) -- (10,-2)
%				to [generic, l_=$h_{oe}^{-1}$] (10,-1)
%				to [short] (10,0);		
%			\draw (7,-3) to [short, -o] (7,-4) node[anchor=east] {$E$};
%			\draw (5,0) node[anchor=east] {$B$};
%			\draw (11,0) node[anchor=west] {$C$};
%		\end{circuitikz}
%	\end{center}
	
	\begin{center}
		\begin{circuitikz}[scale=.5,american voltages, american currents, transform shape]
			\draw (0,0) to [short,i_=$i_b$, o-] (1,0)
				to [short, -] (1,-1)
				to [generic, l=$h_{ie}$] (1,-2)
				to [short] (1,-3) -- (3,-3) -- (3,-2)
				(3,-1) to [I, i= $h_{fe} \cdot i_b$] (3,-2)
				to [short] (3,0)
				to [short] (5,0)
				to [short, , i<_=$i_c$, -o] (6,0)
				;
			\draw (3,-3) to [short] (5,-3) -- (5,-2)
				to [generic, l_=$h_{oe}^{-1}$] (5,-1)
				to [short] (5,0);		
			\draw (2,-3) to [short, -o] (2,-4) node[anchor=east] {$E$};
			\draw (0,0) node[anchor=east] {$B$};
			\draw (6,0) node[anchor=west] {$C$};
		\end{circuitikz}
	\end{center}

\begin{itemize}
	\item Impedancia de entrada media.
	\item Impedancia de salida media.
	\item Alta ganancia de corriente.
	\item Alta ganancia de tensión.
\end{itemize}

\subsubsection{Colector Común}

%\begin{center}
%\begin{circuitikz}[scale=.5,american voltages, american currents, transform shape]
%	\draw (0,0) node[npn](npn)	{}
%		(npn.C) node[anchor=east] {C} % Collector
%    	(npn.B) node[anchor=north east] {B} % Gate
%        (npn.E) node[anchor= east] {E} % Emisor
%        (npn.B) [short, -*] to (-1, 0)
%        (npn.E) -- (0, -1) [short, -*] to (1, -1)
%        (0, -1) [short, -*] to (-1, -1)
%        (npn.C) -- (0, 1) [short, -*] to (1, 1);
%	%\draw (1,0) -- (0,0) -- (0,1);
%	%\draw (-1, 0) -- (pnp.C);
%\end{circuitikz}
%\end{center}

	\begin{center}
		\begin{circuitikz}[scale=.5,american voltages, american currents, transform shape]
			\draw (0,0) to [short,i_=$i_b$, o-] (1,0)
				to [short, -] (1,-1)
				to [generic, l=$h_{ic}$] (1,-2)
				to [short] (1,-3) -- (3,-3) -- (3,-2)
				(3,-1) to [I, i^= $h_{fc} \cdot i_b$] (3,-2)
				to [short] (3,0)
				to [short, i<^=$i_e$, -o] (4,0)
				;		
			\draw (2,-3) to [short, -o] (2,-4) node[anchor=east] {$C$};
			\draw (0,0) node[anchor=east] {$B$};
			\draw (4,0) node[anchor=west] {$E$};
		\end{circuitikz}
	\end{center}

\begin{itemize}
	\item Alta impedancia de entrada.
	\item Muy baja impedancia de salida.
	\item Alta ganancia de corriente.
	\item Unidad, o menor, ganancia de tensión.
\end{itemize}

\section{\underline{Transistor FET}}

\begin{equation*}
	g_{m} = \frac{2}{|G_{GS_{OFF}}|} \sqrt{I_{D} \cdot I_{DSS}}
\end{equation*}

	\begin{center}
		\begin{tikzpicture}[scale = 0.4, transform shape]
		    \draw (-4,0) -- (4,0) node[right] {$V_{GS}$};
    		\draw (0,-3) -- (0,3) node[above] {$I_D$};
			\draw[scale=2,domain=0:1,thick,variable=\x,blue] plot (\x + 1,\x*\x) node[left] {$nMOS acum$};
			\draw (1.7,-0.2) node[right] {$V_T$};
			
			\draw[scale=2,domain=0:1,thick,variable=\x,red] plot (\x - 1,\x*\x) node[left] {$nMOS depl$};
			\draw (-1.9,-0.2) node[right] {$V_T$};
			\draw (.05,2) node[right] {$I_{D_{SS}}$};
			
			\draw[scale=2,domain=0:1,thick,variable=\x,orange] plot (-\x + 1,-\x*\x) node[right] {$pMOS depl$};
			
			\draw[scale=2,domain=0:1,thick,variable=\x,purple] plot (-\x - 1,-\x*\x) node[right] {$pMOS acum$};
    		
		\end{tikzpicture}
		
		\hspace{5pt}
		
		\begin{tikzpicture}[scale = 0.6, transform shape]
		    \draw (-0.2,0) -- (3,0) node[right] {$V_{DS}$};
    		\draw (0,-0.2) -- (0,3) node[above] {$I_{D_{SS}}$};
    		
    		\node[text width=3cm] at (2.7,1.2) {Saturación};
    		\node[text width=3cm] at (3,0.1) {Corte};
    		\node[text width=3cm, rotate = 70] at (.7,2.2) {Óhmica};
    		
		\end{tikzpicture}
	\end{center}
	
	
	\begin{center}
		\begin{tikzpicture}[scale = 0.45, transform shape]
		    \draw (-0.2,0) -- (3,0) node[right] {$V_{DS}$};
    		\draw (0,-0.2) -- (0,3) node[above] {$I_{D_{SS}}$};
    		
    		\node[text width=3cm] at (2.7,1.2) {Saturación};
    		\node[text width=3cm] at (3,0.1) {Corte};
    		\node[text width=3cm, rotate = 70] at (.7,2.2) {Óhmica};
    		
		\end{tikzpicture}
	\end{center}

\section{\underline{Equivalencias}}

	\begin{center}
		\begin{circuitikz}[scale=.5,american voltages, american currents, transform shape]
			\draw (0,0) node[npn](npn1) {}
				(npn1.C) -| (npn1.B);	
			\draw[-latex] (1,0) -- (2,0) ;
			\draw (3,1) to[diode] (3,-1);
			
			\draw (6,0) node[pnp](npn1) {}
				(npn1.C) -| (npn1.B);	
			\draw[-latex] (7,0) -- (8,0) ;
			\draw (9,1) to[diode] (9,-1);


		\end{circuitikz}
	\end{center}


\section{\underline{Resolución de Ejercicios con Transistores}}

\subsubsection{Puntos de Polarización}

	\begin{enumerate}
		\item Poner $v_{i}$, y a veces $v_{o}$, a cero voltios.
		\item Buscar rama de diferencial $\rightarrow$ Corriente de referencia.
	\end{enumerate}
	
\subsubsection{Pequeña Señal}

	\begin{enumerate}
		\item Localizar entradas inversora y no inversora.
	\end{enumerate}

	\begin{itemize}
		\item Modo Diferencial $\rightarrow$ $v_{i}/2$ y puntos de unión a tierra.
		\item Modo Común $\rightarrow$ $v_{i}$ y puntos de unión mediante $R$ a tierra.
	\end{itemize}

Condensador entre dos patas de un BJT.

	\begin{center}
		\begin{circuitikz}[scale=.5,american voltages, american currents, transform shape]
			\draw (0,0) to [short,i_=$i_b$, o-] (1,0)
				to [short, -] (1,-1)
				to [generic, l=$h_{ie}$] (1,-2)
				to [short] (1,-3) -- (3,-3) -- (3,-2)
				(3,-1) to [I, i= $h_{fe} \cdot i_b$] (3,-2)
				to [short] (3,0)
				to [short] (4,0)
%				to [short, i<=$i_c$, -o] (5,0)
				;
			\draw (.9,0) -- (.9,.5) to[C, l = $C$] (3.1,.5) -- (3.1,0);
%			\draw (3,-3) to [short] (5,-3) -- (5,-2)
%				to [generic, l_=$h_{oe}^{-1}$] (5,-1)
%				to [short] (5,0);		
			\draw (2,-3) to [short, -o] (2,-4) node[anchor=east] {$E$};
			\draw (-1,0) node[anchor=north] {$B$};
			\draw (5,0) node[anchor=west] {$C$};
			
			
			\draw (7,0) to [short,i^=$i_b$, o-] (8,0)
				-| (9,-1)
				to [generic, l=$h_{ie}$] (9,-2)
				to [short] (9,-3) -- (11,-3) -- (11,-2)
				(11,-1) to [I, i= $h_{fe} \cdot i_b$] (11,-2)
				to [short] (11,0)
				to [short] (13,0)
				to [short, , i<^=$i_c$, -o] (14,0)
				;
			
			\draw (8,0) circle [radius = 1.5pt] node[anchor = south]{$V_{A}$};
			\draw (13,0) circle [radius = 1.5pt] node[anchor = south]{$V_{B}$};

			\draw (8,0) to[C, l_= $C_{M}$] (8,-3)
				(7.75,-3) -- (8.25,-3);
			\draw (13,0) to[C, l= $C'$] (13,-3)
				(12.75,-3) -- (13.25,-3);
	
			\draw (10,-3) to [short, -o] (10,-4) node[anchor=east] {$E$};
			\draw (7,0) node[anchor=north] {$B$};
			\draw (14,0) node[anchor=west] {$C$};
		\end{circuitikz}
	\end{center}
	
	\begin{equation*}
		\frac{1}{2 \pi \cdot f \cdot C} = \text{(Componentes en paralelo con el C)}
	\end{equation*}
	
\begin{equation*}
	C_{M} = C \left( 1 - \frac{V_{B}}{V_{A}} \right) \hspace{10pt} C = C \left( 1 - \frac{V_{A}}{V_{B}} \right)
\end{equation*}

\subsubsection{Impedancias de Entrada y Salida}

\subsubsection{Factor de Rechazo al Modo Común}
	
	\begin{equation*}
		\text{CMRR} = 20 \cdot \log_{10} \left| \frac{g_{D}}{g_{C}} \right|
	\end{equation*}
	
Para mejorar CMRR $\rightarrow$ disminuir la ganancia en el modo común $\rightarrow$ incrementar $R_{E}$. Sustituir las resistencias por cargas activas (fuentes de corriente que en alterna se comportan como resistencias).

\section{\underline{Resolución de Ejercicios con Operacionales}}

\begin{itemize}
	\item Identificar redes de realimentación $\rightarrow$ cuando hya un camino que une la salida de una etapa con la entrada.
\end{itemize}


\section{\underline{1. Etapas de Dos Transistores}}

\subsubsection{Configuración en Paralelo}

	\begin{center}
		\begin{circuitikz}[scale=.6,american voltages, american currents, transform shape]
			\draw (0,0) node[npn](npn1) {$Q_1$}
				(npn1.E) to [R, l=$R$] (0,-2) -- (0,-2.5) 
				(npn1.B) to [short, -o]  (-1,0) node[left] {$B_P$}; 
 			\draw (1,0) node[npn](npn2) {$Q_2$}
 				(npn2.E) to [R, l=$R$] (1,-2) -- (1,-2.5); 
 			\draw (3,0) node[npn](npn3) {$Q_N$}
 				(npn3.E) to [R, l=$R$] (3,-2) -- (3,-2.5); 
 				
 			\draw (npn1.B) [short] to (1.7,0);
 			\draw (1.7,0) node[right] {...};
 			
 			\d3porub
			raw (npn1.C) -- (1.7,0.78) 
 					(1.7,0.78) node[right] {...}
 					(2.2, .78) -- (npn3.C);
 					
 			\draw (0,-2.5) -- (1.7,-2.5) 
 					(1.7,-2.5) node[right] {...}
 					(2.2, -2.5) -- (3,-2.5);
 			
 			\draw (1.5,0.78) to [short, -o, i<_=$I_{CP}$] (1.5,1.28) node[right] {$C_P$};
 			\draw (1.5,-2.5) to [short, -o] (1.5,-3) node[right] {$E_P$};
 			
 			\draw (-4,0) node {$h_{ie_{CP}} = \frac{h_{ie_k}}{n} = \frac{V_T}{I_{B_k} \cdot n} = \frac{V_T}{I_{B_P}}$};
 			\draw (-5,-.5) node {$h_{fe_{CP}} = h_{f_{ek}} = \beta$};
 			\draw (-4,-1) node {$h_{oe_{CP}} = h_{oe_k} \cdot n = \frac{I_{C_k} \cdot n}{V_A} = \frac{I_{C_P}}{V_A}$};
 			
		\end{circuitikz}
	\end{center}

Se comporta como \textbf{un único transistor}, necesita una resistencia $R$ (\textit{ballast resistor}) para estabilizar el reparto de corriente entre los transistores. Permite trabajar en \textbf{altas corrientes}.\\

Las resistencias de emisor permiten estabilizar el circuito tensión.

	\begin{equation*}
		I_{CN} = \frac{I_{CP}}{N} \hspace{20pt} V_{BE_N} + I_{C_N}
	\end{equation*}

\subsubsection{Configuración Darlington}

	\begin{center}
	\begin{circuitikz}[scale=1,american voltages, american currents, transform shape]
  		\begin{scope}[scale = .7, transform shape]
			\draw (0,0) node[npn](npn1) {}
  			(npn1.base) node[anchor=east] {B};
 			\draw (1,-0.78) node[npn](npn2) {};
 			\draw (npn1.E) [short, -, i_=$I_{B2}$] to (npn2.B);
 			\draw (npn2.E) node[anchor=east] {E};
 			%\draw (npn1.C)--(npn2.C);
 			\draw (npn1.C) [short, -, i<^=$I_{C1}$] to (1, 0.78) [short, -, i_=$I_{C2}$] to (npn2.C);
 			\draw (1,0.78) [short, -*, i<_=$I_C$] to (1, 1.5) node[anchor=east] {C};
			
			 \draw (-2,1) node {$\frac{I_C}{I_B} \approx \beta_1 \beta_2$};
	 		\draw (-2.3,0) node {$h_{ie_{D}} = 2 \cdot h_{ie_{1}}$};
			\draw (-2,-.5) node {$h_{fe_{D}} \approx h_{fe_{1}} \cdot h_{fe_{2}}$};
			\draw (-2,-1) node {$h_{oe_{D}} \approx h_{oe_{1}} \cdot h_{fe_{2}}$};
		\end{scope}
 		
		\begin{scope}[scale = .5, transform shape]
			\draw (3,2) to [short,i_=$i_b$, o-] (4,2)
				to [short, -] (4,1)
				to [generic, l=$h_{ie_D}$] (4,0)
				to [short] (4,-1) -- (6,-1) -- (6,0)
				(6,1) to [I, i= $h_{fe_D} \cdot i_{b_D}$] (6,0)
				to [short] (6,2)
				to [short] (8,2)
				to [short, , i<_=$i_c$, -o] (9,2)
				;
			\draw (6,-1) to [short, -] (8,-1) -- (8,0)
				to [generic, l_=$h_{oe_D}^{-1}$] (8,1)
				to [short, -*] (8,2);		
			\draw (6,-1) to [short, *-o] (6,-2) node[anchor=east] {$E_D$};
			\draw (3,2) node[anchor=east] {$B_D$};
			\draw (9,2) node[anchor=west] {$C_D$};
		\end{scope}
	\end{circuitikz}
	\end{center}

\begin{equation*}
	h_{ie_{D}} = h_{ie_{1}} ( 1 + h_{fe_{1}}) \cdot h_{ie_{2}}
\end{equation*}

\textbf{Gran ganancia de corriente} pero \textbf{baja impedancia de salida}. Las fugas del primer transistor son amplificadas por el segundo, sólo es aconsejable en agrupaciones de 2 transistores.
	
%	\begin{center}
%		\begin{circuitikz}[scale=.5,american voltages, american currents, transform shape]
%			\draw (0,0) to [short,i_=$i_b$, o-] (1,0)
%				to [short, -] (1,-1)
%				to [R, l=$h_{ie_D}$] (1,-2)
%				to [short] (1,-3) -- (3,-3) -- (3,-2)
%				(3,-1) to [I, i= $h_{fe_D} \cdot i_{b_D}$] (3,-2)
%				to [short] (3,0)
%				to [short] (5,0)
%				to [short, , i<_=$i_c$, -o] (6,0)
%				;
%			\draw (3,-3) to [short, -] (5,-3) -- (5,-2)
%				to [R, l_=$h_{oe_D}^{-1}$] (5,-1)
%				to [short, -*] (5,0);		
%			\draw (3,-3) to [short, *-o] (3,-4) node[anchor=east] {$E_D$};
%			\draw (0,0) node[anchor=east] {$B_D$};
%			\draw (6,0) node[anchor=west] {$C_D$};
%			
%			
%		\end{circuitikz}
%	\end{center}

	\begin{equation*}
		\beta_T \approx \beta_1 \cdot \beta_2 \approx \frac{I_C}{I_B}
	\end{equation*}

\subsubsection{Amplificador Diferencial}

	\begin{center}
		\begin{circuitikz}[scale=.6,american voltages, american currents, transform shape]
			\draw (0,0) to [short, o-*] (0,-0.5)
				(0.5,0) node {$+V_{CC}$};
				
			\draw (-1,-0.5) to [short] (1,-0.5)
				(-1,-0.5) to [short, i_= $I_{C1}$] (-1,-1)
				(1,-0.5) to [short, i= $I_{C2}$] (1,-1);
			\draw (-1,-1) to [R, l_=$R_{C1}$] (-1,-2)
				to [short] (-1,-2.2)
				to [short, *-o] (-0.5,-2.2)
				(-0.4,-2.5) node {$+V_{O1}$}
				(-0.45,-2.7) node[ground]{} (-0.45,-2.9)
				(-1,-2.2) to [short] (-1,-2.4);
			\draw (-1,-3.1) node[npn](npn1) {}
				(npn1.base) to [short, -o] (-2.2,-3.1)
				(-2.2,-3.4) node {$+V_{i1}$}
				(-2.2,-3.7) node[ground]{} (-2.2,-3.8)
  				(npn1.base) node[anchor=south east] {$Q_1$};
			
			\draw (1,-1) to [R, l=$R_{C2}$] (1,-2)
				to [short] (1,-2.2)
				to [short, *-o] (0.5,-2.2)
				(0.4,-2.5) node {$+V_{O2}$}
				(0.45,-2.7) node[ground]{} (0.45,-2.9)
				(1,-2.2) to [short] (1,-2.4);
			\draw (1,-3.1) node[npn,xscale = -1](npn2) {}
				(npn2.base) to [short, -o] (2.2,-3.1)
				(2.2,-3.4) node {$+V_{i2}$}
				(2.2,-3.7) node[ground]{} (2.2,-3.8)
  				(npn2.base) node[anchor=south west] {$Q_2$}
  				(npn1.emitter) to [short] (npn2.emitter);
  			\draw (0,-3.85) to [short, *-] (0,-4.35)
  				to [R, l=$R_{EE}$] (0,-5.35)
  				to [short, -o] (0,-5.85)
  				(0,-5.85) node[anchor=west] {$-V_{CC}$};
		\end{circuitikz}

	\end{center}

\textbf{Amplificador de continua}, simétrico con 2 entradas y 2 salidas. Amplifica exclusivamente la diferencia de sus entradas: $V_O = k \cdot (V_{i1} - V_{i2})$. Tiene los siguientes modos de funcionamiento:

	\begin{itemize}
		\item \textbf{Salida Diferencial}: La señal de salida se toma entre las salidas de cada semi-red ($V_O = V_{O_1} - V_{O_2}$).
		\item \textbf{Salida Asimétrica}: La señal de salida se toma de la salida de una semi-red ($V_O = V_{O_x}$). Para este modo es necesaria una $R_{EE}$ elevada.
	\end{itemize}

Tiene efectos parásitos como las \textbf{corrientes de polarización} que provocan caí­das de tensión en las resistencias internas de las fuentes de señal a amplificar.\\


\text{Mirar pagina 23--- salida simetrica y tal}

\subsubsection{Teorema de Barttlet}

Simplifica el análisis de \textbf{redes simétricas}. Las corrientes totales que atraviesan las ramas que unen ambas semi-redes son nulas. El comportamiento del circuito no se altera si se abren dichas ramas.

	\begin{center}
		\begin{circuitikz}[scale=.5,american voltages, american currents, transform shape]
			 \draw (0,.3) to [short, o-, color = blue] (.5,.3);
			 \draw (0,2) to [short, o-, color = blue] (.5,2);
			  \node[draw,minimum width=2cm,minimum height=2.4cm,anchor=south west, color = blue] at (.5,0){$N/2$};
			  \draw (2.5,.3) to [short, i_=$I_{2}$] (3.22,.3)
			  	(3.27,.3) to [short, i<_=$I_{2}$] (4,.3)
			  	 (2.5,2) to [short, i_=$I_{1}$] (3.23,2)
				 (3.27,2) to [short, i<_=$I_{1}$] (4,2);
			\draw (0,1.9) node[below] {$+$};
			\draw (0,1.15) node {$V_{ic}$};
			\draw (0,.4) node[above] {$-$};
			
			\draw (6.5,1.9) node[below] {$+$};
			\draw (6.5,1.15) node {$V_{ic}$};
			\draw (6.5,.4) node[above] {$-$};
			  \node[draw,minimum width=2cm,minimum height=2.4cm,anchor=south west, color = red] at (4,0){$N/2$};
          			\draw (6,.3) to [short, -o, color = red] (6.5,.3)
			  	 (6,2) to [short, -o, color = red] (6.5,2);
				 
				 
			 \draw (8,.3) to [short, o-, color = blue] (8.5,.3);
			 \draw (8,2) to [short, o-, color = blue] (8.5,2);
			  \node[draw,minimum width=2cm,minimum height=2.4cm,anchor=south west, color = blue] at (8.5,0){$N/2$};
			  \draw (10.5,.3) to [short] (11,.3)
			  	(11.5,.3) to [short] (12,.3)
			  	 (10.5,2) to [short] (11,2)
				 (11.5,2) to [short] (12,2);
			\draw (8,1.9) node[below] {$+$};
			\draw (8,1.15) node {$V_{ic}$};
			\draw (8,.4) node[above] {$-$};
			
			\draw (14.5,1.9) node[below] {$+$};
			\draw (14.5,1.15) node {$V_{ic}$};
			\draw (14.5,.4) node[above] {$-$};
			  \node[draw,minimum width=2cm,minimum height=2.4cm,anchor=south west, color = red] at (12,0){$N/2$};
          			\draw (14,.3) to [short, -o, color = red] (14.5,.3)
			  	 (14,2) to [short, -o, color = red] (14.5,2);

		\end{circuitikz}
	\end{center}

\textbf{Red simétrica con entrada diferencial}, los nodos que unen ambas semi-redes están a cero voltios por ser nula la corriente que atraviesan las impedancias que los unen a masa. El comportamiento del circuito no se altera si se abren dichas ramas y se unen nodos comunes a masa.

	\begin{center}
		\begin{circuitikz}[scale=.5,american voltages, american currents, transform shape, european]
			 \draw (0,.3) to [short, o-, color = blue] (.5,.3);
			 \draw (0,2) to [short, o-, color = blue] (.5,2);
			  \node[draw,minimum width=2cm,minimum height=2.4cm,anchor=south west, color = blue] at (.5,0){$N/2$};
			  \draw (2.5,.3) to [short, i_=$I_{2}$] (3.22,.3)
			  	(3.27,.3) to [short, i_=$I_{2}$] (4,.3)
			  	 (2.5,2) to [short, i_=$I_{1}$] (3.23,2)
				 (3.27,2) to [short, i_=$I_{1}$] (4,2);
			\draw (0,1.9) node[below] {$+$};
			\draw (0,1.15) node {$V_{id} / 2$};
			\draw (0,.4) node[above] {$-$};
			
			\draw (6.5,4) to[R, scale = .5, l = $Z_{1}$] (6.5,2);
			\draw (3.25,1.25) node[ground, scale = .5] {$-$};
			
			\draw (6.5,.55) to[R, scale = .5, l = $Z_{2}$] (6.5,-1.25);
			\draw (3.25,-.55) node[ground, scale = .5] {$-$};
			
			\draw (6.5,1.9) node[below] {$+$};
			\draw (6.5,1.15) node {$V_{id} / 2$};
			\draw (6.5,.4) node[above] {$-$};
			  \node[draw,minimum width=2cm,minimum height=2.4cm,anchor=south west, color = red] at (4,0){$N/2$};
          			\draw (6,.3) to [short, -o, color = red] (6.5,.3)
			  	 (6,2) to [short, -o, color = red] (6.5,2);
				 
				 
			 \draw (8,.3) to [short, o-, color = blue] (8.5,.3);
			 \draw (8,2) to [short, o-, color = blue] (8.5,2);
			  \node[draw,minimum width=2cm,minimum height=2.4cm,anchor=south west, color = blue] at (8.5,0){$N/2$};
			  \draw (10.5,.3) to [short] (11,.3)
			  	(11.5,.3) to [short] (12,.3)
			  	 (10.5,2) -| (11,0)
				 (11.5,0) |- (12,2);
				 ;
			\draw (11.5,0) node[ground, scale = 0.5]{};
			\draw (11,0) node[ground, scale = .5]{};
			\draw (8,1.9) node[below] {$+$};
			\draw (8,1.15) node {$V_{id} / 2$};
			\draw (8,.4) node[above] {$-$};
			
			\draw (14.5,1.9) node[below] {$+$};
			\draw (14.5,1.15) node {$V_{id} / 2$};
			\draw (14.5,.4) node[above] {$-$};
			  \node[draw,minimum width=2cm,minimum height=2.4cm,anchor=south west, color = red] at (12,0){$N/2$};
          			\draw (14,.3) to [short, -o, color = red] (14.5,.3)
			  	 (14,2) to [short, -o, color = red] (14.5,2);

		\end{circuitikz}
	\end{center}


\subsubsection{Circuito Cascodo}

	\begin{center}
		\begin{circuitikz}[scale=.5,american voltages, american currents, transform shape]
			\draw (0,0) to [short, o-*] (0,-0.5)
				(0.5,0) node {$+V_{CC}$};
			\draw (-1,-0.5) to [short] (1,-0.5)
				(-1,-0.5) to [short] (-1,-1)
				(1,-0.5) to [short] (1,-1);
			\draw (-1,-1) to [R, l_=$R_1$] (-1,-2)
				to [short] (-1,-2.2)
				(-1,-2.2) to [short] (-1,-3.1);
			\draw (1,-1) to [R, l=$R_L$] (1,-2)
				to [short] (1,-2.2)
				to [short, *-o] (1.5,-2.2)
				(1,-2.2) to [short] (1,-2.4);
			\draw (1,-3.1) node[npn](npn2) {}
  				(npn2.base) node[anchor=south east] {$Q_2$}
  				(npn2.base) to [short, -*] (-1,-3.1)
  				to [short] (-2,-3.1)
  				to [short] (-2,-3.5)
  				to [C] (-2,-3.8)
  				to [short] (-2,-4)
  				(-2,-4) node[ground]{} (-2,-4.3)
  				;
  			\draw (-1,-3.1) to [short] (-1,-3.6)
  				to [R, l=$R_2$] (-1,-5.3)
  				to [short] (-1,-6.05)
  				;
  			\draw (npn2.emitter) to [R, l=$R_C$] (1,-5.3);
  			\draw (1,-6.05) node[npn](npn1) {}
  				(npn1.base) node[anchor=south east] {$Q_1$}
  				(npn1.base) to [short, -*] (-1,-6.05)
  				(-1,-6.05) to [short, -o] (-2, -6.05)
  				(-2, -6.25) node {$V_i$}
  				(-1, -6.05) to [short] (-1, -6.8)
  				(-1, -6.8) to [R, l=$R_3$] (-1,-8)
  				to [short] (-1,-8.2);
			\draw (-1,-8.2) to [short] (1,-8.2)
				(0,-8.2) to  [short, *-] (0,-8.3)
				to node[ground]{} (0,-8.5);
			\draw (1,-6.7) to [R, l=$R_L$] (1,-8.2)
				(1,-6.7) to [short] (2.5,-6.7)
				to [C] (2.5,-8.2)
				to [short] (1,-8.2);  			
		\end{circuitikz}
	\end{center}


Permite trabajar con \textbf{mayores tensiones de salida}. Buen comportamiento en \textbf{alta frecuencia}. Se utiliza como etapa de entrada, no intermedia.

\subsubsection{Etapas CMOS}

\section{\underline{2.- Amplificadores Multietapa}}

\subsubsection{Clasificación}

\textbf{De alterna o de acoplo RC}: Se agrupan en cascada de amplificadores monoetapa conectándose mediante condensadores de acoplo. \textbf{De continua o de acoplo directo} No existen condensadores de paso y amplifican tensiones continuas.

\subsubsection{Análisis en Continua}

\textbf{Amplificadores RC}: Cada etapa puede analizarse independientemente. No existe dependencia entre los puntos de trabajo de las distintas etapas.

	\begin{center}
		\begin{circuitikz}[scale=.5,american voltages, american currents, transform shape]
			 \draw (-1.6,.3) to [short] (1,.3)
			  	 (0,1.8) to [C, color = red] (1,1.8)
				  (-1.6,1.8) to [R, l=$R_{S}$] (0,1.8)
				 (-1.6,.3) to [sV, l=$V_{S}$] (-1.6,1.8);
			  \node[draw,minimum width=2cm,minimum height=2.4cm,anchor=south west] at (1,0){Etapa 1 ($A_{V1}$)};
			  \draw (3.15,.3) to [short] (4,.3)
			  	 (3.15,1.8) to [C, color = red] (4,1.8);
			  \node[draw,minimum width=2cm,minimum height=2.4cm,anchor=south west] at (4,0){Etapa 2 ($A_{V2}$)};
			   \draw (6.15,.3) to [short] (7,.3)
			  	 (6.15,1.8) to [C, color = red] (7,1.8);
			 \draw [densely dashed] (7,.3)--(8,.3);
			  \draw [densely dashed] (7,1.8)--(8,1.8);
			  \draw (8,.3) to [short] (9,.3)
			  	 (8,1.8) to [C, color = red] (9,1.8);
			\node[draw,minimum width=2cm,minimum height=2.4cm,anchor=south west] at (9,0){Etapa n ($A_{Vn}$)};
			 \draw (11.2,.3) to [short] (12,.3)
			  	 (11.2,1.8) to [C, color = red] (12,1.8)
				 (12,.3) to [R, l_=$R_{L}$] (12,1.8);
		\end{circuitikz}
	\end{center}
	
\textbf{Amplificadores de continua}: Todas las etapas están inter-relacionadas, es aconsejable seguir un orden para su análisis.
			

\subsubsection{Ganancia en Pequeña Señal}

	\begin{center}
		\begin{circuitikz}[scale=.5,american voltages, american currents, transform shape]
			 \draw (-1.6,.3) to [short] (1,.3)
			  	 (0,1.8) to [C, color = red] (1,1.8)
				  (-1.6,1.8) to [R, l=$R_{S}$] (0,1.8)
				 (-1.6,.3) to [sV, l=$V_{S}$] (-1.6,1.8);
			  \node[draw,minimum width=2cm,minimum height=2.4cm,anchor=south west, color = blue] at (1,0){Etapa 1 ($A_{V1}$)};
			  \draw (3.15,.3) to [short] (4,.3)
			  	 (3.15,1.8) to [C, color = blue] (4,1.8);
			  \node[draw,minimum width=2cm,minimum height=2.4cm,anchor=south west, color = green] at (4,0){Etapa 2 ($A_{V2}$)};
			   \draw (6.15,.3) to [short] (7,.3)
			  	 (6.15,1.8) to [C, color = green] (7,1.8);
			 \draw [densely dashed] (7,.3)--(8,.3);
			  \draw [densely dashed] (7,1.8)--(8,1.8);
			  \draw (8,.3) to [short] (9,.3)
			  	 (8,1.8) to [C, color = red] (9,1.8);
			\node[draw,minimum width=2cm,minimum height=2.4cm,anchor=south west, color = orange] at (9,0){Etapa n ($A_{Vn}$)};
			 \draw (11.2,.3) to [short, color = orange] (12,.3)
			  	 (11.2,1.8) to [C, color = orange] (12,1.8)
				 (12,.3) to [R, l_=$R_{L}$, color = orange] (12,1.8);
		\end{circuitikz}
	\end{center}

La \textbf{ganancia total} es el producto de las ganancias de cada etapa:

	\begin{equation*}
		A_{v} = \frac{v_{o}}{v_{s}} = \frac{v_{o}}{v_{o(n-1)}} \dots \frac{v_{o2}}{v_{o1}} 
	\end{equation*}
	
\textbf{Ganancia de una etapa:}

	\begin{center}
		\begin{circuitikz}[scale=.5,american voltages, american currents, transform shape]

			\draw[-latex] (8,.4) -- (8,1.7) node[left] {$V_{o(k-1)}$};
			\draw[-latex] (11.4,.4) -- (11.4,1.7) node[left] {$V_{ok}$};
			  \draw (8,.3) to [short] (9,.3)
			  	 (8,1.8) to [short] (9,1.8);
			\node[draw,minimum width=2cm,minimum height=2.4cm,anchor=south west, color = blue] at (9,0){Etapa K ($A_{Vk}$)};
			 \draw (11.2,.3) to [short, color = blue] (12.5,.3)
			  	 (11.2,1.8) to [C, color = blue] (12.5,1.8)
				 (12.5,.3) to [R, l_=$Z_{i(k+1)}$, color = blue] (12.5,1.8);
		\end{circuitikz}
	\end{center}	
	
Es preciso tener en cuenta la impedancia de la siguiente etapa.

	\begin{equation*}
		A_{vk} = \frac{v_{ok}}{v_{o(k-1)}} |_{Z_{i(k+1)}}
	\end{equation*}
	

\subsubsection{Margen dinámico a la Salida}

Máxima señal obtenible en la carga sin distorsión, se mide en $V_{P}$ o en $V_{PP}$. Viene determinado por las situaciones de corte/saturación de los transistores.

\textbf{Margen dinámico debido a una etapa}: Si la etapa k-ésima tiene un margen dinámico a su salida permitirá a la salida del amplificador un margen dinámico.\\

	\begin{equation*}
		M_{dko} = M_{dk} \cdot | A_{v(k+1)} \cdot A_{v(k+2)} \cdot ... \cdot A_{vn} |
	\end{equation*}

\textbf{Margen dinámico del amplificador}: Menor de los márgenes dinámicos debidos a cada etapa.

	\begin{equation*}
		M_{do} = min \{  M_{d1o}, M_{d12}, ..., M_{dn} \}
	\end{equation*}

\subsubsection{Margen dinámico a la Entrada}



\begin{center}
		\begin{circuitikz}[scale=.5,american voltages, american currents, transform shape]
			 
			 \draw[blue, thick] (-.25,.95) sin (0,1.45) cos (.25,.95) sin (.5,.45) cos (.75,.95);
			 
			 \draw (0,.3) to [short] (1,.3)
			  	 (0,1.8) to [C, color = red] (1,1.8);
			  \node[draw,minimum width=2cm,minimum height=2.4cm,anchor=south west] at (1,0){Etapa 1 ($A_{V1}$)};
			  \draw (3.15,.3) to [short] (4,.3)
			  	 (3.15,1.8) to [C, color = red] (4,1.8);
			  \node[draw,minimum width=2cm,minimum height=2.4cm,anchor=south west] at (4,0){Etapa 2 ($A_{V2}$)};
			   \draw (6.15,.3) to [short] (7,.3)
			  	 (6.15,1.8) to [C, color = red] (7,1.8);
			 \draw [densely dashed] (7,.3)--(8,.3);
			  \draw [densely dashed] (7,1.8)--(8,1.8);
			  \draw (8,.3) to [short] (9,.3)
			  	 (8,1.8) to [C, color = red] (9,1.8);
			\node[draw,minimum width=2cm,minimum height=2.4cm,anchor=south west] at (9,0){Etapa n ($A_{Vn}$)};
			 \draw (11.2,.3) to [short] (12,.3)
			  	 (11.2,1.8) to [C, color = red] (12,1.8);
				 
			\draw[blue, thick] (12,.95) sin (12.25,1.95) cos (12.5,.95) sin (12.75,.05) cos (13,.95);
		\end{circuitikz}
	\end{center}
	
Máximo nivel de señal en la entrada para no tener distorsión a la salida.

\begin{equation*}
	M_{di} = \frac{M_{do}}{| A_{v} |}
\end{equation*}


\section{\underline{3. Respuesta en Frecuencia}}

\subsection{Amplificadores Reales}

Las señales son amplificadas de forma distinta según sea su frecuencia.
 
\begin{equation*}
	A_{v} = \frac{V_{o}}{V_{i}} \hspace{10pt} \rightarrow \text{frecuencia} \rightarrow \hspace{10pt} A_{v}(f) = \frac{V_{o}(f)}{V_{i}(f)}
\end{equation*}

\begin{center}
	\begin{tikzpicture}[scale = .5,transform shape]
	\begin{axis}[axis lines = middle,xmin=-0.1,xmax = 3.2,ymin=-0.3,ymax=2, xticklabels={,,}, yticklabels={,,}] %,grid=both]
				
		% Valores del eje X
		\node at (axis cs:1, 0) [anchor=north] {$f_a$};
		\node at (axis cs:2.5, 0) [anchor=north] {$f_b$};
		
		\node at (axis cs:0, 3) [anchor=north west] {$|A(f)|$};
		\node at (axis cs:3.2, 0) [anchor=south east] {$f$};
		
		
		
	\end{axis}
	
	\draw[dashed, gray] (2.28,.5) -- (2.28,5);
	\draw[dashed, gray] (5.4,.5) -- (5.4,5);
	
%	\draw (5,1) node[left]{Frecuencias Medias};
	\draw[red,thick] (2.28,3) -- (5.4,3);
	
	\draw (1,1) node[left]{Bajas Frecuencias};
	\draw[orange, thick] (.5,.6) .. controls (1.5,.7) and (1.5,2.8) .. (2.28,3);
	
%	\draw (5,5) node[left]{Altas Frecuencias};
	\draw[orange, thick] (5.4,3) .. controls (6,2.8) and (6,.8) .. (6.5,.6);
\end{tikzpicture}
\end{center}

Debido a los efectos de las capacidades, de paso y parásitas, $|A(f)| \downarrow$.

\subsection{Diagramas de Bode}

\begin{equation*}
|A(f)|(dB) \rightarrow 20 \cdot \log_{10} \left| \frac{A(f)}{A_{o}} \right|
\end{equation*}

\subsection{Expresión de la Ganancia}

La \textbf{ganancia de un amplificador} para la región útil y zonas no muy alejadas de ella se expresa en función de $s$:

	\begin{equation*}
		A(s) = A_{o} \frac{s}{s + w_{b}} \frac{w_{a}}{s + w_{a}}
	\end{equation*}

Si el amplificador está bien diseñado, $w_{a} \gg w_{b}$. $w_{a}$ y $w_{b}$ son siempre números reales (circuitos RC no realimentados).

\subsection{Comportamiento Asintótico}

\subsection{Frecuencias de Corte}

Son aquellas frecuencias en las que la ganancia se reduce $\sqrt{2}$ veces $\rightarrow 20 \log (1 / \sqrt{2}) = -3 \text{dB}$

\subsection{Respuesta en Bajas Frecuencias}

\textbf{Terminar}







\section{\underline{4. Fuentes de Corriente y Cargas Activas}}

Una \textbf{carga activa o carga dinámica} es un componente de circuito que se comporta como una resistencia no lineal estable contra corriente. Utilizado frecuentemente en la entrada de amplificadores operacionales para incrementar considerablemente la ganancia.

\subsubsection{Introducción y Figura de Mérito}

La \textbf{figura de mérito} de un amplificador es el producto de la ganancia por el ancho de banda.

	\begin{equation*}
		GB = |A_{M}| BW
	\end{equation*}

\subsection{Configuraciones de Fuentes de Corriente}

\textbf{Espejo de Corriente}: Permite obtener una corriente constante (fuente de corriente).

	\begin{center}
		\begin{circuitikz} [scale=.6, transform shape]
			\draw[blue, thick] (0,0) node[npn, color = blue, xscale = 1, yscale = -1, rotate = 180](npn1) {$Q_1$}
				(npn1.C) to [short, blue, *-] (.84,.78) to [short, blue, i=$I_{BB}$] (npn1.B)
				(npn1.C) to [R, color = blue, l=$R_{CC}$] (0,2) [short, -o, i<_=$I_{ref}$] to (0,3) node[anchor=east] {$V_{CC}$};
			\draw (2,0) node[npn, xscale = 1, yscale = 1, rotate = 0](npn2) {$Q_2$}
				(npn2.B) -- (npn1.B)
				(npn1.E) [short, thick, blue, -|] to (1,-.78);
			\draw (npn2.C) [short, -o, ,i<_=$I_O$] to (2,1.5);
			\draw (1,-0.78) [short] to (npn2.E);
			\draw[blue, thick] (1,-.78) [short, *-] to (1, -1.28) node[ground, color = blue]{} (1, -1.5); 
			\draw (1.9,-.6) node[anchor=east] {\tiny $V_{BE1} = V_{BE2}$};
			\draw (3,2.7) node[anchor=east] {$I_O = \frac{I_{ref}}{1 + \frac{2}{\beta_F}} $};
			
		\end{circuitikz}
	\end{center}

	\begin{center}
		\begin{circuitikz}[scale=.4,american voltages, american currents, transform shape]
			\draw (0,0) to [short,i_=$i_b$, o-] (1,0)
				to [short, -] (1,-1)
				to [R, l=$h_{ie}$] (1,-2)
				to [short] (1,-3) -- (3,-3) -- (3,-2)
				(3,-1) to [I, i= $h_{fe_D} \cdot i_{b_D}$] (3,-2)
				to [short] (3,0)
				to [short] (5,0)
				to [short, , i<_=$i_c$, -o] (6,0)
				;
			\draw (3,-3) to [short, -] (5,-3) -- (5,-2)
				to [R, l_=$h_{oe}^{-1}$] (5,-1)
				to [short, -*] (5,0);		
			\draw (2,-3) to [short, *-] (2,-3.1) node[ground]{} (-2, -2.2);
			%\draw (0,0) node[anchor=east] {$B_D$};
			%\draw (6,0) node[anchor=west] {$C_D$};
			\draw[blue, thick] (-1,0) -- (-1,-1) to [R, color = blue, l=$R_{Q0}$] (-1,-2) -- (-1,-3) node[ground, color = blue]{} (-1, -2.2);
			\draw[blue, thick] (-2,0) -- (-2,-1) to [R, color = blue, l_=$R_{CC}$] (-2,-2) -- (-2, -3) node[ground, color = blue]{} (-2, -2.2)
				(-2,0) [short, color = blue, -*] to (0,0);
		\end{circuitikz}
	\end{center}
	
	\begin{center}
		\begin{circuitikz} [scale=.6, transform shape]
			\draw[blue, thick] (4,0) node[npn, color = blue](npn1) {}
				(npn1.C) -| (npn1.B) %i=${I_{BB}}$] to (npn1.B)
				(npn1.C) to [R, color = blue, l=$R_{CC}$] (4,2.4) [short, -o, i<_=$I_{ref}$] to (4,3) node[anchor=east] {$V_{CC}$};
			\draw[blue, thick] (2,0) node[npn, color = blue, xscale = 1, yscale = -1, rotate = 180](npn2) {};
			\draw (1,0) node[right] {...};
			\draw[blue, thick] (0,0) node[npn, color = blue, xscale = 1, yscale = -1, rotate = 180](npn0) {};
			\draw[blue] (1.5,0) -- (npn1.B);
		\end{circuitikz}
	\end{center}
	
Si hay varias etapas se divide la corriente:

	\begin{equation*}
		I_O = \frac{I_{ref}}{1 + \frac{N}{\beta_F}}
	\end{equation*}
	
\subsubsection{Fuente Widlar}

Variación del circuito anterior con una resistencia en el emisor del transistor de salida para obtener corrientes pequeñas constantes en la salida.

	\begin{center}
		\begin{circuitikz} [scale=.6, transform shape]
			\draw[blue, thick] (0,0) node[npn, color = blue, xscale = 1, yscale = -1, rotate = 180](npn1) {$Q_1$}
				(npn1.C) to [short, blue, *-] (.84,.78) to [short, blue, i=$I_{BB}$] (npn1.B)
				(npn1.C) to [R, color = blue, l=$R_{CC}$] (0,2) [short, -o, i<_=$I_{ref}$] to (0,3) node[anchor=east] {$V_{CC}$}
				;
			\draw (2,0) node[npn, xscale = 1, yscale = 1, rotate = 0](npn2) {$Q_2$}
				%(npn2.B) -- (npn1.B)
				%(npn1.E) [short, thick] to (1,-.78)
				%(npn1.E) to [short, blue] (0, -2.78)
				(npn2.E) to [R, l=$R_1$] (2,-2.78)
				(npn2.B) to [short] (.78,0)
				;
			\draw[blue] (npn1.E) to [short, blue] (0, -2.78) -- (1, -2.78);
			\draw (1,-2.78) -- (2, -2.78);
			\draw (npn2.C) [short, -o, ,i<_=$I_O$] to (2,1.5);
			%\draw (1,-0.78) [short] to (npn2.E);
			\draw[blue, thick] (1,-2.78) [short, *-] to (1, -3.28) node[ground, color = blue]{} (1, -3.5); 
			
			%Dibujo derecha
			
		
			\draw (7,0) node[npn, xscale = 1, yscale = 1, rotate = 0](npn3) {$Q_2$}
				(npn3.E) to [R, l=$R_1$] (7,-2.78)
				(npn3.B) to [short] (5.78,0)
				;
			\draw [blue, thick] (5,0) to[D*, color = blue] (5,-2)
				(5,0) -| (npn3.B)
				(5,0) to [R, color = blue, l=$R_{CC}$] (5,2) [short, -o, i<_=$I_{ref}$] to (5,3) node[anchor=east] {$V_{CC}$}
				;
			\draw[blue] (5,-2) to [short, blue] (5, -2.78) -- (6, -2.78);
			\draw (6,-2.78) -- (8, -2.78);
			\draw (npn3.C) [short, -o, ,i<_=$I_O$] to (7,1.5);
			\draw[blue, thick] (6,-2.78) [short, *-] to (6, -3.28) node[ground, color = blue]{} (6, -3.5); 
			
		\end{circuitikz}
	\end{center}

	\begin{center}
		\begin{circuitikz} [scale=.6, transform shape]
			\draw[blue, thick] (0,0) to[D*, color = blue] (0,-2)
				(0,0) |- (npn2.B)
				(0,0) to [R, color = blue, l=$R_{CC}$] (0,2) [short, -o, i<_=$I_{ref}$] to (0,3) node[anchor=east] {$V_{CC}$}
				;
			\draw (2,0) node[npn, xscale = 1, yscale = 1, rotate = 0](npn2) {$Q_2$}
				(npn2.E) to [R, l=$R_1$] (2,-2.78)
				(npn2.B) to [short] (.78,0)
				;
			\draw[blue] (npn1.E) to [short, blue] (0, -2.78) -- (1, -2.78);
			\draw (1,-2.78) -- (2, -2.78);
			\draw (npn2.C) [short, -o, ,i<_=$I_O$] to (2,1.5);
			\draw[blue, thick] (1,-2.78) [short, *-] to (1, -3.28) node[ground, color = blue]{} (1, -3.5); 
		\end{circuitikz}
	\end{center}

	\begin{equation*}
		I_0^{(k+1)} = \frac{V_T}{R_E} ln \left( \frac{I_{ref}}{I_O^{(k)}} \right)
	\end{equation*}

\subsubsection{Fuente de Cascodo}

Proporciona impedancia de salida alta, mucho mayor que en las otras fuentes.

	\begin{center}
		\begin{circuitikz} [scale=.6, transform shape]
			\draw[blue, thick] (0,0) node[npn, color = blue, xscale = 1, yscale = -1, rotate = 180](npn1) {$Q_1$}
				(npn1.C) to [short, blue, *-] (.84,.78) to [short, blue, i=$I_{BB}$] (npn1.B)
				(npn1.C) to [R, color = blue, l=$R_{CC}$] (0,2) [short, -o, i<_=$I_{ref}$] to (0,3) node[anchor=east] {$V_{CC}$};
			\draw (2,0) node[npn, xscale = 1, yscale = 1, rotate = 0](npn2) {$Q_2$}
				(npn2.B) -- (npn1.B)
				%(npn1.E) [short, thick, blue, -|] to (1,-.78);
				;
			\draw (npn2.C) [short, -o,i<_=$I_O$] to (2,1.5);
			\draw[blue, thick] (1,-2.78) [short, *-] to (1, -3.28) node[ground, color = blue]{} (1, -2.5); 
			
			\draw[blue, thick] (0,-2) node[npn, color = blue, xscale = 1, yscale = -1, rotate = 180](npn3) {$Q_3$}
				(npn3.C) to [short, blue, *-] (.84,-1.22) to [short, blue, i=$I_{BB}$] (npn3.B)
				(npn1.E) -- (npn3.C)
				;
			\draw (2,-2) node[npn, xscale = 1, yscale = 1, rotate = 0](npn4) {$Q_4$}
				(npn4.B) -- (npn3.B)
				(npn3.E) [short, thick, blue, -|] to (1,-2.78);
			\draw (npn4.C) [short, -o] to (2,-.5);
			\draw (1,-2.78) [short] to (npn4.E);
		\end{circuitikz}
	\end{center}


\subsubsection{Fuente de Wilson}

	\begin{center}
		\begin{circuitikz} [scale=.6, transform shape]
			\draw[blue, thick] (0,0) node[npn, color = blue, xscale = 1, yscale = -1, rotate = 180](npn1) {$Q_1$}
				(npn1.C) to [short, blue, *-] (.84,.78) to [short, blue, i=$I_{BB}$] (npn1.B)
				(npn1.C) to [R, color = blue, l=$R_{CC}$] (0,2) [short, -o, i<_=$I_{ref}$] to (0,3) node[anchor=east] {$V_{CC}$};
			\draw (2,0) node[npn, xscale = 1, yscale = 1, rotate = 0](npn2) {$Q_2$}
				(npn2.B) -- (npn1.B)
				%(npn1.E) [short, thick, blue, -|] to (1,-.78);
				;
			\draw (npn2.C) [short, -o,i<_=$I_O$] to (2,1.5);
			\draw[blue, thick] (1,-2.78) [short, *-] to (1, -3.28) node[ground, color = blue]{} (1, -2.5); 
			
			\draw[blue, thick] (0,-2) node[npn, color = blue, xscale = 1, yscale = -1, rotate = 180](npn3) {$Q_3$}
				(npn3.C) to [short, blue, *-] (.84,-1.22) to [short, blue, i=$I_{BB}$] (npn3.B)
				(npn1.E) -- (npn3.C)
				;
			\draw (2,-2) node[npn, xscale = 1, yscale = 1, rotate = 0](npn4) {$Q_4$}
				(npn4.B) -- (npn3.B)
				(npn3.E) [short, thick, blue, -|] to (1,-2.78);
			\draw (npn4.C) [short, -o] to (2,-.5);
			\draw (1,-2.78) [short] to (npn4.E);
		\end{circuitikz}
	\end{center}

Permite obtener alta ganancia de corriente e impedancia de salida elevada.

\subsection{Cargas Activas}

\subsection{FC y CA con Amplificadores Diferenciales}

\subsection{Polarización Independiente de $V_{CC}$}

\subsection{Desplazador de Nivel}

\section{\underline{5. Etapas de Potencia}}

\subsubsection{Clasificación}

\textbf{Clase A}: El transistor conduce durante el \textbf{ciclo completo}. \\

\textbf{Clase B}: El transistor conduce durante \textbf{medio ciclo}.\\

\textbf{Clase AB}: El transistor conduce durante algo \textbf{más de medio ciclo}.\\

\textbf{Clase C}: El transistor conduce durante algo \textbf{menos de medio ciclo}.

\subsubsection{Definiciones}

\textbf{Potencia Consumida}: Potencia suministrada por la fuente de alimentación de continua.

	\begin{equation*}
		P_{CC} = \frac{1}{T} \int_0^T V_{CC} \cdot i_C(t) dt = V_{CC} \cdot \langle i_C (t) \rangle
	\end{equation*}
	
\textbf{Potencia Entregada a la Carga}: Potencia de la señal amplificada en la carga de alterna.

	\begin{equation*}
			P_O = V_{oeff} \cdot I_{oeff} = V_{Leff} \cdot I_{Leff} = P_L
	\end{equation*}
	
\textbf{Potencia disipada por el transistor}: Consumida en el transistor, lo calienta.

	\begin{equation*}
		P_D = P_{CC} - P_O - P_{resto}
	\end{equation*}
	
\textbf{Rendimiento de la etapa}: Potencia entregada a la carga, respecto de la consumida de la fuente de alimentación.

	\begin{equation*}
		\eta(\%) = \frac{P_O}{P_{CC}} \cdot 100 \%
	\end{equation*}

\subsubsection{Amplificadores Clase A}

\subsubsection{Amplificadores Clase B y Clase AB}

\subsubsection{Consideraciones Térmicas}

\section{\underline{6. Amplificador Operacional}}

\begin{itemize}
	\item Impedancia de entrada infinita
	\item Impedancia de salida nula
	\item Ganancia diferencial infinita
	\item CMRR infinito
	\item Margen dinámico $\pm V_{CC}$
\end{itemize}

\begin{center}
\begin{circuitikz}[scale=.6, transform shape, european]
  \draw
  (0, 0) node[op amp] (opamp) {}
  (opamp.-) to[R, l=$R_{1}$] (-3, 0.5)
  (opamp.-) to[short,*-] ++(0,1.5) coordinate (leftC)
  to[R, l = $R_{2}$] (leftC -| opamp.out)
  to[short,-*] (opamp.out)
  (opamp.+)  to[short,*-, -|] ++(0,-1) % coordinate (leftC)
	-| (1.5,-1.5)
	(-3,-1.5) -| (0,-1.5);
	
	\draw[-latex] (1.2,-1.3) -- (1.2,-.1) node[right] {$V_{o}$};
	\draw[-latex] (-3,-1.3) -- (-3,.3) node[left] {$V_{i}$};
\end{circuitikz}
\end{center}

\textbf{Amplificador Inversor}:

\begin{center}
\begin{circuitikz}[scale=.6, transform shape, european]
  \draw
  (0, 0) node[op amp] (opamp) {}
  (opamp.-) to[R, l=$R_{1}$] (-3, 0.5)
  (opamp.-) to[short,*-] ++(0,1.5) coordinate (leftC)
  to[R, l = $R_{2}$] (leftC -| opamp.out)
  to[short,-*] (opamp.out) -- (2,0)
%  (opamp.+)  to[short,*-, -|] ++(0,-1) % coordinate (leftC)
%	-| (1.5,-1.5)
%	(-3,-1.5) -| (0,-1.5);
;	
%	\draw (1.5,-.1) node[right] {$V_{o}$};
	\draw (opamp.+) node[above] {$V_{i}$};
	\draw (-3,.5) [short, *-] to (-3,.5);% node[ground]{} (-3,.4); 
	
	\draw (opamp.+) |- (-1.2,-1.25);
	\draw (-3,-1.25) -- (2,-1.25);
	
	\draw[-latex] (2,-1.2) -- (2,-.1) node[right] {$V_{o}$};
	\draw[-latex] (-3,-1.2) -- (-3,.45) node[left] {$V_{i}$};
	
	\draw (3.5,1) node {$V_{o} = - R_{2} / R_{1} V_{i}$};
	\draw (3.5,.5) node {$Z_{i} = \frac{V_{i}}{i_{1}} = R_{1}$};
	\draw (3.25,0) node {$Z_{o} = 0$};
	
\end{circuitikz}
\end{center}

\textbf{Amplificador No Inversor}:\\
	

\begin{center}
\begin{circuitikz}[scale=.6, transform shape, european]
  \draw
  (0, 0) node[op amp] (opamp) {}
  (opamp.-) to[R, l=$R_{1}$] (-3, 0.5)
  (opamp.-) to[short,*-] ++(0,1.5) coordinate (leftC)
  to[R, l = $R_{2}$] (leftC -| opamp.out)
  to[short,-*] (opamp.out) -- (2,0);
  	
	\draw (opamp.+) node[above] {$V_{i}$};
	\draw (-3,.5) [short, -] to (-3,.5) -- (-3,-0);% node[ground]{} (-3,.4); 
	\draw (-3.25,0) -- (-2.75,0);
	
%	\draw (opamp.+) |- (-1.2,-1.25);
%	\draw (-3,-1.25) -- (2,-1.25);
	
%	\draw[-latex] (2,-1.2) -- (2,-.1) node[right] {$V_{o}$};
%	\draw[-latex] (-3,-1.2) -- (-3,.45) node[left] {$V_{i}$};
	
	\draw (3.5,1) node {$V_{o} =  V_{i} (1 + R_{2} / R_{1})$};
	\draw (3.5,.5) node {$Z_{i} = \frac{V_{i}}{I_{i}} = \infty$};
	\draw (3.25,0) node {$Z_{o} = 0$};
	
\end{circuitikz}
\end{center}

\textbf{Seguidor}:\\

%	$V_{o} = V_{i}$\\
%	$Z_{i} = \frac{V_{i}}{I_{i}} = \infty$\\
%	$Z_{o} = 0$
	
	\begin{center}
\begin{circuitikz}[scale=.6, transform shape, european]
  \draw
  (0, 0) node[op amp] (opamp) {}
%  (opamp.-) to[short] (-3, 0.5)
  (opamp.-) to[short,*-] ++(0,1.2) coordinate (leftC)
  to[short] (leftC -| opamp.out)
  to[short,-*] (opamp.out)
;	
	\draw (1.2,-.1) node[right] {$V_{o}$};
	\draw (opamp.+) node[above] {$V_{i}$};
%	\draw (-3,.5) [short, *-] to (-3,.5) node[ground]{} (-3,.4); 

	\draw (3.5,1) node {$V_{o} = V_{i}$};
	\draw (3.5,.5) node {$Z_{i} = \frac{V_{i}}{I_{i}} = \infty$};
	\draw (3.5,0) node {$Z_{o} = 0$};

\end{circuitikz}
\end{center}

\subsubsection{Efectos no ideales}

\begin{center}
		\begin{circuitikz}[scale=.5,american voltages, american currents, transform shape]
			 			 
			 \draw (0,.3) to [short, o-] (1,.3)
			  	 (0,1.8) to [short, o-] (1,1.8);
				 
			  \node[draw,minimum width=2cm,minimum height=2.4cm,anchor=south west] at (1,0){Etapa Diferencial};
			  \draw (3.5,1.05) to [short] (4.5,1.05);
				 
			  \node[draw,minimum width=2cm,minimum height=2.4cm,anchor=south west] at (4.5,0){Etapa de Ganancia};
			   \draw (7.25,1.05) to [short] (8.2,1.05);
;
			\node[draw,minimum width=2cm,minimum height=2.4cm,anchor=south west] at (8.2,0){Etapa de Salida};
			 \draw (10.5,1.05) to [short, -o] (11,1.05);
				 
		\end{circuitikz}
	\end{center}
	
\subsubsection{Corrientes de Polarización}



\section{\underline{7. Realimentación}}



%\vfill

\newpage


\end{multicols}
\end{document}