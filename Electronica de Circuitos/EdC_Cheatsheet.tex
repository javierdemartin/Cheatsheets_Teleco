\documentclass[10pt,landscape]{article}
\usepackage{multicol}
\usepackage{calc}
\usepackage[landscape]{geometry}
\usepackage{amsmath,amsthm,amsfonts,amssymb}
\usepackage{color,graphicx,overpic}
\graphicspath{ {images/} }
\usepackage{hyperref}
\usepackage{esint}
\usepackage{bm}
\usepackage{relsize}
\usepackage{tikz}
\usepackage{circuitikz}
\usepackage{datetime}
\usepackage{pgfplots}
\usepackage[utf8] {inputenc}
\usepackage[spanish, activeacute] {babel}
\usepackage{IEEEtrantools}
\usepackage{framed}

%\usepackage{draftwatermark}
%\SetWatermarkText{Javier de Martí­n}
%\SetWatermarkScale{4.8}

% This sets page margins to .5 inch if using letter paper, and to 1cm
% if using A4 paper. (This probably isn't strictly necessary.)
% If using another size paper, use default 1cm margins.
\ifthenelse{\lengthtest { \paperwidth = 11in}}
    { \geometry{top=.5in,left=.5in,right=.5in,bottom=.5in} }
    {\ifthenelse{ \lengthtest{ \paperwidth = 297mm}}
        {\geometry{top=1cm,left=1cm,right=1cm,bottom=1cm} }
        {\geometry{top=1cm,left=1cm,right=1cm,bottom=1cm} }
    }

% Turn off header and footer
\pagestyle{empty}

% Redefine section commands to use less space
\makeatletter
\renewcommand{\section}{\@startsection{section}{1}{0mm}%
                                {-1ex plus -.5ex minus -.2ex}%
                                {0.5ex plus .2ex}%x
                                {\normalfont\large\bfseries}}
\renewcommand{\subsection}{\@startsection{subsection}{2}{0mm}%
                                {-1explus -.5ex minus -.2ex}%
                                {0.5ex plus .2ex}%
                                {\normalfont\normalsize\bfseries}}
\renewcommand{\subsubsection}{\@startsection{subsubsection}{3}{0mm}%
                                {-1ex plus -.5ex minus -.2ex}%
                                {1ex plus .2ex}%
                                {\normalfont\small\bfseries}}
\makeatother

\newcommand{\Lagr}{\mathcal{L}}

% Define BibTeX command
\def\BibTeX{{\rm B\kern-.05em{\sc i\kern-.025em b}\kern-.08em
    T\kern-.1667em\lower.7ex\hbox{E}\kern-.125emX}}

% Don't print section numbers
\setcounter{secnumdepth}{0}


\setlength{\parindent}{0pt}
\setlength{\parskip}{0pt plus 0.5ex}

%My Environments
\newtheorem{example}[section]{Example}
% ---------------------------------------------------------------

\begin{document}
\raggedright
\footnotesize
\begin{multicols}{3}


% multicol parameters
% These lengths are set only within the two main columns
%\setlength{\columnseprule}{0.25pt}
\setlength{\premulticols}{1pt}
\setlength{\postmulticols}{1pt}
\setlength{\multicolsep}{1pt}
\setlength{\columnsep}{2pt}

\begin{framed}
	\begin{center}
    	\Large{\underline{Electrónica de Circuitos}} \\
    	\scriptsize{3º Ingeniería de Telecomunicaciones | UPV/EHU}\\
     	%Actualizado por última vez el \today \\
     	"\textsl{Under-promise and over-deliver}." \\
     	%\hspace{5 pt} \\
     	\small{\textbf{Javier de Martín -- 2016/17}}
	\end{center}
\end{framed}

%
% Cheatsheet code below 
%                                                      

\section{\underline{Repaso BJTs}}

\subsection{Parámetros}

$h_{ix} (\Omega)$: Impedancia de entrada\\

$h_{rx}$: Reverse voltage ratio \\

$h_{fx}$ Forward current transfer ratio\\

$h_{ox} (\Omega^{-1})$: Admitancia de salida\\


\begin{tabular}{|c|c|c|}
\hline
Base Común                              & Emisor Común & Colector Común \\ \hline
    $h_{ib} = \frac{v_{eb}}{i_e}$       &  $h_{ie} = \frac{v_{be}}{i_b}$            &   $h_{ic} = \frac{v_{bc}}{i_b}$             \\ \hline
    $h_{rb} = \frac{v_{eb}}{v_{cb}}$   &              &                \\ \hline
    $h_{fb} = \frac{i_c}{i_e}$       &              &                \\ \hline
     $h_{ob} = \frac{i_c}{v_{cb}}$      &              &                \\ \hline
\end{tabular}

\subsection{Regiones Operativas}

	\begin{center}
		\begin{tikzpicture}
		    \draw (-0.2,0) -- (2.5,0) node[right] {$I_C$};
    		\draw (0,-0.2) -- (0,2.5) node[above] {$V_{CE}$};
    		
    		\node[text width=3cm] at (2.5,1) {Activa};
    		\node[text width=3cm] at (2.3,0.1) {Saturación};
    		\node[text width=3cm, rotate = 90] at (.1,2.2) {Corte};
    		
		\end{tikzpicture}
	\end{center}

\begin{itemize}
	\item \textbf{Región Activa}: La corriente de colector $I_C$ depende directamente de la corriente de base $I_B$, de la ganancia de corriente $\beta$ y de las resistencias conectadas al colector y emisor. En esta región se produce amplificación de la señal.
	\item \textbf{Región Inversa}: 
	\item \textbf{Región de Saturación}: La corriente $I_C = I_E = I_{max}$. La corriente depende del voltaje de alimentación del circuito y de las resistencias conectadas al colector y emisor. Este modo aparece cuando la corriente de base es lo suficientemente grande como para inducir una corriente de colector $\beta$ veces más grande.
	\item \textbf{Región de Corte}: La corriente $I_C = I_E = 0$. El voltaje $V_{CE}$ es el de alimentación del circuito, al no haber corriente circulando no hay caída tensión. Este modo aparece, normalmente, cuando $I_B = 0$.
\end{itemize}

\subsection{Configuraciones de Montaje}

\subsubsection{Base Común}

\begin{center}
\begin{circuitikz} %[scale=1]
	\draw (0,0) node[npn](npn)	{}
		(npn.C) node[anchor=east] {C} % Collector
    	(npn.B) node[anchor=north west] {B} % Gate
        (npn.E) node[anchor= north east] {E} % Emisor
        (npn.B) -- (-1, 0)
        (-1, 0) [short, -*] to (-1, 1)
        (-1, 0) [short, -*] to (-1, -1)
        (npn.E) [short, -*] to (0, -1) 
        (npn.C) [short, -*] to (0, 1);
	%\draw (1,0) -- (0,0) -- (0,1);
	%\draw (-1, 0) -- (pnp.C);
	;
\end{circuitikz}
\end{center}

	\begin{center}
		\begin{circuitikz}[scale=.5,american voltages, american currents, transform shape]
			\draw (0,0) to [short,i_=$i_b$, o-] (1,0)
				to [short, -] (1,-1)
				to [generic, l=$h_{ib}$] (1,-2)
				to [short] (1,-3) -- (3,-3) -- (3,-2)
				(3,-1) to [I, i= $h_{fe} \cdot i_e$] (3,-2)
				to [short] (3,0)
				to [short, i<_=$i_c$, -o] (4,0)
				;		
			\draw (2,-3) to [short, -o] (2,-4) node[anchor=east] {$B$};
			\draw (0,0) node[anchor=east] {$E$};
			\draw (4,0) node[anchor=west] {$C$};
		\end{circuitikz}
	\end{center}

\begin{itemize}
	\item Baja impedancia de entrada.
	\item Alta impedancia de salida
	\item Ganancia unidad, o menor, de corriente.
	\item Ganancia alta de tensión.
\end{itemize}


\subsubsection{Emisor Común}

\begin{center}
\begin{circuitikz} %[scale=1]
	\draw (0,0) node[npn](npn)	{}
		(npn.C) node[anchor=east] {C} % Collector
    	(npn.B) node[anchor=north east] {B} % Gate
        (npn.E) node[anchor= east] {E} % Emisor
        (npn.B) [short, -*] to (-1, 0)
        (npn.E) -- (0, -1) [short, -*] to (1, -1)
        (0, -1) [short, -*] to (-1, -1)
        (npn.C) -- (0, 1) [short, -*] to (1, 1);
	%\draw (1,0) -- (0,0) -- (0,1);
	%\draw (-1, 0) -- (pnp.C);
\end{circuitikz}
\end{center}

	\begin{center}
		\begin{circuitikz}[scale=.5,american voltages, american currents, transform shape]
			\draw (0,0) to [short,i_=$i_b$, o-] (1,0)
				to [short, -] (1,-1)
				to [generic, l=$h_{ie}$] (1,-2)
				to [short] (1,-3) -- (3,-3) -- (3,-2)
				(3,-1) to [I, i= $h_{fe} \cdot i_b$] (3,-2)
				to [short] (3,0)
				to [short, i<_=$i_c$, -o] (4,0)
				;		
			\draw (2,-3) to [short, -o] (2,-4) node[anchor=east] {$E$};
			\draw (0,0) node[anchor=east] {$B$};
			\draw (4,0) node[anchor=west] {$C$};
		\end{circuitikz}
	\end{center}
	
	\begin{center}
		\begin{circuitikz}[scale=.5,american voltages, american currents, transform shape]
			\draw (0,0) to [short,i_=$i_b$, o-] (1,0)
				to [short, -] (1,-1)
				to [generic, l=$h_{ie}$] (1,-2)
				to [short] (1,-3) -- (3,-3) -- (3,-2)
				(3,-1) to [I, i= $h_{fe} \cdot i_b$] (3,-2)
				to [short] (3,0)
				to [short] (5,0)
				to [short, , i<_=$i_c$, -o] (6,0)
				;
			\draw (3,-3) to [short] (5,-3) -- (5,-2)
				to [generic, l_=$h_{oe}^{-1}$] (5,-1)
				to [short] (5,0);		
			\draw (2,-3) to [short, -o] (2,-4) node[anchor=east] {$E$};
			\draw (0,0) node[anchor=east] {$B$};
			\draw (6,0) node[anchor=west] {$C$};
		\end{circuitikz}
	\end{center}

\begin{itemize}
	\item Impedancia de entrada media.
	\item Impedancia de salida media.
	\item Alta ganancia de corriente.
	\item Alta ganancia de tensión.
\end{itemize}

\subsubsection{Colector Común}

\begin{center}
\begin{circuitikz} %[scale=1]
	\draw (0,0) node[npn](npn)	{}
		(npn.C) node[anchor=east] {C} % Collector
    	(npn.B) node[anchor=north east] {B} % Gate
        (npn.E) node[anchor= east] {E} % Emisor
        (npn.B) [short, -*] to (-1, 0)
        (npn.E) -- (0, -1) [short, -*] to (1, -1)
        (0, -1) [short, -*] to (-1, -1)
        (npn.C) -- (0, 1) [short, -*] to (1, 1);
	%\draw (1,0) -- (0,0) -- (0,1);
	%\draw (-1, 0) -- (pnp.C);
\end{circuitikz}
\end{center}

	\begin{center}
		\begin{circuitikz}[scale=.5,american voltages, american currents, transform shape]
			\draw (0,0) to [short,i_=$i_b$, o-] (1,0)
				to [short, -] (1,-1)
				to [generic, l=$h_{ic}$] (1,-2)
				to [short] (1,-3) -- (3,-3) -- (3,-2)
				(3,-1) to [I, i= $h_{fc} \cdot i_b$] (3,-2)
				to [short] (3,0)
				to [short, i<_=$i_e$, -o] (4,0)
				;		
			\draw (2,-3) to [short, -o] (2,-4) node[anchor=east] {$C$};
			\draw (0,0) node[anchor=east] {$B$};
			\draw (4,0) node[anchor=west] {$E$};
		\end{circuitikz}
	\end{center}

\begin{itemize}
	\item Alta impedancia de entrada.
	\item Muy baja impedancia de salida.
	\item Alta ganancia de corriente.
	\item Unidad, o menor, ganancia de tensión.
\end{itemize}

\section{\underline{Transistor FET}}

	\begin{center}
		\begin{tikzpicture}
		    \draw (-4,0) -- (4,0) node[right] {$V_{GS}$};
    		\draw (0,-3) -- (0,3) node[above] {$I_D$};
			\draw[scale=2,domain=0:1,thick,variable=\x,blue] plot (\x + 1,\x*\x) node[left] {$nMOS acum$};
			\draw (1.7,-0.2) node[right] {$V_T$};
			
			\draw[scale=2,domain=0:1,thick,variable=\x,red] plot (\x - 1,\x*\x) node[left] {$nMOS depl$};
			\draw (-1.9,-0.2) node[right] {$V_T$};
			\draw (.05,2) node[right] {$I_{D_{SS}}$};
			
			\draw[scale=2,domain=0:1,thick,variable=\x,orange] plot (-\x + 1,-\x*\x) node[right] {$pMOS depl$};
			
			\draw[scale=2,domain=0:1,thick,variable=\x,purple] plot (-\x - 1,-\x*\x) node[right] {$pMOS acum$};
    		
		\end{tikzpicture}
	\end{center}
	
	
	\begin{center}
		\begin{tikzpicture}
		    \draw (-0.2,0) -- (3,0) node[right] {$V_{DS}$};
    		\draw (0,-0.2) -- (0,3) node[above] {$I_{D_{SS}}$};
    		
    		\node[text width=3cm] at (2.7,1.2) {Saturación};
    		\node[text width=3cm] at (3,0.1) {Corte};
    		\node[text width=3cm, rotate = 70] at (.7,2.2) {Óhmica};
    		
		\end{tikzpicture}
	\end{center}

\section{\underline{Etapas de Dos Transistores}}

\subsubsection{Configuración en Paralelo}

	\begin{center}
		\begin{circuitikz} %[scale=1]
			\draw (0,0) node[npn](npn1) {$Q_1$}
				(npn1.E) to [R, l=$R$] (0,-2) -- (0,-2.5) 
				(npn1.B) to [short, -o]  (-1,0) node[left] {$B_P$}; 
 			\draw (1,0) node[npn](npn2) {$Q_2$}
 				(npn2.E) to [R, l=$R$] (1,-2) -- (1,-2.5); 
 			\draw (3,0) node[npn](npn3) {$Q_N$}
 				(npn3.E) to [R, l=$R$] (3,-2) -- (3,-2.5); 
 				
 			\draw (npn1.B) [short] to (1.7,0);
 			\draw (1.7,0) node[right] {...};
 			
 			\draw (npn1.C) -- (1.7,0.78) 
 					(1.7,0.78) node[right] {...}
 					(2.2, .78) -- (npn3.C);
 					
 			\draw (0,-2.5) -- (1.7,-2.5) 
 					(1.7,-2.5) node[right] {...}
 					(2.2, -2.5) -- (3,-2.5);
 			
 			\draw (1.5,0.78) to [short, -o, i<_=$I_{CP}$] (1.5,1.28) node[right] {$C_P$};
 			\draw (1.5,-2.5) to [short, -o] (1.5,-3) node[right] {$E_P$};
 			
		\end{circuitikz}
	\end{center}

Se comporta como un único transistor, necesita una resistencia $R$ (\textit{ballast resistor}) para estabilizar el reparto de corriente entre los transistores.

	\begin{equation*}
		I_{CN} = \frac{I_{CP}}{N} \hspace{20pt} V_{BE_N} + I_{C_N}
	\end{equation*}

\subsubsection{Configuración Darlington}

Gran ganancia de corriente pero baja impedancia de salida. Las fugas del primer transistor son amplificadas por el segundo, sólo es aconsejable en agrupaciones de 2 transistores.
	
	\begin{center}
	\begin{circuitikz}
  		\draw (0,0) node[npn](npn1) {}
  		(npn1.base) node[anchor=east] {B};
 		\draw (1,-0.78) node[npn](npn2) {};
 		\draw (npn1.E) [short, -, i_=$I_{B2}$] to (npn2.B);
 		\draw (npn2.E) node[anchor=east] {E};
 		%\draw (npn1.C)--(npn2.C);
 		\draw (npn1.C) [short, -, i<^=$I_{C1}$] to (1, 0.78) [short, -, i_=$I_{C2}$] to (npn2.C);
 		\draw (1,0.78) [short, -*, i<_=$I_C$] to (1, 1.5) node[anchor=east] {C};
 		
	\end{circuitikz}
	\end{center}
	
	\begin{center}
		\begin{circuitikz}[scale=1,american voltages, american currents, transform shape]
			\draw (0,0) to [short,i_=$i_b$, o-] (1,0)
				to [short, -] (1,-1)
				to [R, l=$h_{ie_D}$] (1,-2)
				to [short] (1,-3) -- (3,-3) -- (3,-2)
				(3,-1) to [I, i= $h_{fe_D} \cdot i_{b_D}$] (3,-2)
				to [short] (3,0)
				to [short] (5,0)
				to [short, , i<_=$i_c$, -o] (6,0)
				;
			\draw (3,-3) to [short, -] (5,-3) -- (5,-2)
				to [R, l_=$h_{oe_D}^{-1}$] (5,-1)
				to [short, -*] (5,0);		
			\draw (3,-3) to [short, *-o] (3,-4) node[anchor=east] {$E_D$};
			\draw (0,0) node[anchor=east] {$B_D$};
			\draw (6,0) node[anchor=west] {$C_D$};
		\end{circuitikz}
	\end{center}

	\begin{equation*}
		\beta_T \approx \beta_1 \cdot \beta_2 \approx \frac{I_C}{I_B}
	\end{equation*}

\subsubsection{Amplificador Diferencial}

\subsubsection{Circuito Cascodo}

\subsubsection{Etapas CMOS}

\section{\underline{Amplificadores Multietapa}}

\subsubsection{Clasificación}

\subsubsection{Análisis en Continua}

\subsubsection{Ganancia en Pequeña Señal}

\subsubsection{Margen Dinámico a la Entrada}

\section{\underline{3. Respuesta en Frecuencia (aUn no)}}

\section{\underline{4. Fuentes de Corriente y Cargas Activas}}

\subsubsection{Introducción y Figura de Mérito}

\subsubsection{Configuraciones de Fuentes de Corriente}

\subsubsection{Cargas Activas}

\subsubsection{FC y CA con Amplificadores Diferenciales}

\subsubsection{Polarización Independiente de $V_{CC}$}

\subsubsection{Desplazador de Nivel}

\section{\underline{5. Etapas de Potencia}}

\subsubsection{Clasificación}

\textbf{Clase A}: El transistor conduce durante el \textbf{ciclo completo}. \\

\textbf{Clase B}: El transistor conduce durante \textbf{medio ciclo}.\\

\textbf{Clase AB}: El transistor conduce durante algo \textbf{más de medio ciclo}.\\

\textbf{Clase C}: El transistor conduce durante algo \textbf{menos de medio ciclo}.

\subsubsection{Definiciones}

\textbf{Potencia Consumida}: Potencia suministrada por la fuente de alimentación de continua.

	\begin{equation*}
		P_{CC} = \frac{1}{T} \int_0^T V_{CC} \cdot i_C(t) dt = V_{CC} \cdot \langle i_C (t) \rangle
	\end{equation*}
	
\textbf{Potencia Entregada a la Carga}: Potencia de la señal amplificada en la carga de alterna.

	\begin{equation*}
			P_O = V_{oeff} \cdot I_{oeff} = V_{Leff} \cdot I_{Leff} = P_L
	\end{equation*}
	
\textbf{Potencia disipada por el transistor}: Consumida en el transistor, lo calienta.

	\begin{equation*}
		P_D = P_{CC} - P_O - P_{resto}
	\end{equation*}
	
\textbf{Rendimiento de la etapa}: Potencia entregada a la carga, respecto de la consumida de la fuente de alimentación.

	\begin{equation*}
		\eta(\%) = \frac{P_O}{P_{CC}} \cdot 100 \%
	\end{equation*}

\subsubsection{Amplificadores Clase A}

\subsubsection{Amplificadores Clase B y Clase AB}

\subsubsection{Consideraciones Térmicas}


%\vfill

\newpage

aaa

\end{multicols}
\end{document}