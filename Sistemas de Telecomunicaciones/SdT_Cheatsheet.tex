\documentclass[10pt,portrait, twocolumn]{article}
\usepackage{multicol}
\usepackage{calc}
\usepackage[portrait]{geometry}
\usepackage{amsmath,amsthm,amsfonts,amssymb}
\usepackage{times}
\usepackage{color,graphicx,overpic}
\graphicspath{ {images/} }
\usepackage{hyperref}
\usepackage{pgfplots}
\usepackage{esint}
\usepackage{bm}
\usepackage{tikz}
\usepackage{relsize}
\usepackage{datetime}
\usepackage[utf8] {inputenc}
\usepackage[spanish, activeacute] {babel}
\usepackage{IEEEtrantools}
\usepackage{framed}

\usepackage{pdflscape}


\usepackage{draftwatermark}
\SetWatermarkText{Javier de Martín}
\SetWatermarkScale{0.8}

% This sets page margins to .5 inch if using letter paper, and to 1cm
% if using A4 paper. (This probably isn't strictly necessary.)
% If using another size paper, use default 1cm margins.
\geometry{top=.5cm,left=.5cm,right=.5cm,bottom=.5cm}
    
\pgfplotsset{
    dirac/.style={
        mark=triangle*,
        mark options={scale=2},
        ycomb,
        scatter,
        visualization depends on={y/abs(y)-1 \as \sign},
        scatter/@pre marker code/.code={\scope[rotate=90*\sign,yshift=-2pt]}
    }
}

% Turn off header and footer
\pagestyle{empty}

% Redefine section commands to use less space
\makeatletter
\renewcommand{\section}{\@startsection{section}{1}{0mm}%
                                {-1ex plus -.5ex minus -.2ex}%
                                {0.5ex plus .2ex}%x
                                {\normalfont\large\bfseries}}
\renewcommand{\subsection}{\@startsection{subsection}{2}{0mm}%
                                {-1explus -.5ex minus -.2ex}%
                                {0.5ex plus .2ex}%
                                {\normalfont\normalsize\bfseries}}
\renewcommand{\subsubsection}{\@startsection{subsubsection}{3}{0mm}%
                                {-1ex plus -.5ex minus -.2ex}%
                                {1ex plus .2ex}%
                                {\normalfont\small\bfseries}}
\makeatother

\newcommand{\Lagr}{\mathcal{L}}

% Define BibTeX command
\def\BibTeX{{\rm B\kern-.05em{\sc i\kern-.025em b}\kern-.08em
    T\kern-.1667em\lower.7ex\hbox{E}\kern-.125emX}}

% Don't print section numbers
\setcounter{secnumdepth}{0}


\setlength{\parindent}{0pt}
\setlength{\parskip}{0pt plus 0.5ex}

%My Environments
\newtheorem{example}[section]{Example}
% ---------------------------------------------------------------

\begin{document}

\begin{landscape}

\raggedright
\footnotesize
\begin{multicols}{3}


% multicol parameters
% These lengths are set only within the two main columns
%\setlength{\columnseprule}{0.25pt}
\setlength{\premulticols}{1pt}
\setlength{\postmulticols}{1pt}
\setlength{\multicolsep}{1pt}
\setlength{\columnsep}{2pt}

\begin{framed}
	\begin{center}
    	\Large{\underline{Sistemas de Telecomunicación}} \\
    	\scriptsize{3º Ingeniería de Telecomunicaciones | UPV/EHU}\\
     	%Actualizado por última vez el \today \\
     	"\textsl{Under-promise and over-deliver}." \\
     	%\hspace{5 pt} \\
     	\small{\textbf{Javier de Martín -- 2016}}
	\end{center}
\end{framed}

%
% Cheatsheet code below 
%                                                      

\section{\underline{Unidades Logarítmicas}}

%\begin{center}
%	\begin{tikzpicture}
%		\node [draw, align=center] (1) at (0,0) {Fuente};
%		\node [draw, align=center] (2) at (0,-1) {Procesado en TX};
%	
%		\node [draw, align=center] (3) at (2,-1) {Red};
%		
%		\node [draw, align=center] (4) at (4,-1) {Procesado en RX};
%		\node [draw, align=center] (5) at (4,0) {Presentación};
%	
%		\path (1) edge node [right] {} (2);
%		\path (2) edge node [above] {} (3);
%		\path (3) edge node [above] {} (4);
%		\path (4) edge node [left] {} (5);
%	\end{tikzpicture}	
%\end{center}

\textbf{dB} es una unidad que describe una \textbf{relación} entre magnitudes.

\begin{IEEEeqnarray*}{rCl}
	L(dB) & = & 10 \cdot \log_{10} \left( \frac{P_2}{P_1} \right) \\ & = & 20 \cdot \log_{10} \left( \frac{V_2}{V_1} \right) + 10 \cdot \log_{10} \left( \frac{R_1	}{R_2} \right)
\end{IEEEeqnarray*}

\subsection{Unidades Derivadas del dB}

\begin{itemize}
	\item $dBm$: Potencia de la señal en un punto cualquiera de un circuito referida a una potencia de $1mW$.
		
		\begin{equation*}
			L(dBm) = 10 \cdot \log_{10} \left( \frac{P(mW)}{1mW} \right)
		\end{equation*}
	\item $dBW$: Potencia de la señal referida a una potencia de $1W$.
		
		\begin{equation*}
			L(dBW) = 10 \cdot \log_{10} \left( \frac{P(W)}{1W} \right)
		\end{equation*}
		
	\item $dBmV$: Nivel de un voltaje comparado con $1mV$ sobre una carga de $75\Omega$.
		
		\begin{equation*}
			L(dbmV) = 20 \cdot \log_{10} \left( \frac{V(mV)}{1mV} \right)
		\end{equation*}
		
	\item $dBV$: Nivel de un voltaje comparado con $0.0775 V$ (tensión eficaz) sobre una carga de $600\Omega$.
		
		\begin{equation*}
			L(dBm) = dBV + 10 \cdot \log_{10} \left( \frac{600}{R} \right)
		\end{equation*}
\end{itemize}

Estas medidas están relacionadas por:

	\begin{equation*}
		L(dBm) = L(dBV) + 10 \cdot \log_{10} \left( \frac{600}{R} \right)
	\end{equation*}

\subsection{Niveles}

\begin{itemize}
	\item $dBr$: Expresa el nivel relativo en un punto con respecto a otro punto; es una medida en $dB$ sin sufijo, $r$ se incluye para denotar que se trata de un valor relativo a un cierto punto de referencia.
	
		\begin{equation*}
			L(dBr) = 10 \log_{10} \frac{P}{P_{ref}}
		\end{equation*}
	
	\item $dBm0$: Indica la potencia en $dBm$ presente en el punto de nivel relativo cero.
		
		\begin{equation*}
			L(dBm0) = L_A(dBm) - L_A(dBr)
		\end{equation*}
\end{itemize}

Si se emite un tono de prueba ($0dBm$) $\rightarrow L_A (dBm) = L_A (dBr)$.

\section{\underline{Perturbaciones y Medios de TX}}

\subsection{Distorsión Lineal}

%\begin{center}
%	\begin{tikzpicture}
%		\draw (0,0) -- node[above left] {$x(t)$} (0.6,0);
%		\node [draw, align=center] at (0.95,0) {$h(t)$};	
%		\draw (1.3,0) -- node[above right] {$y(t) = k \cdot x(t - t_0)$} (1.9,0);
%	\end{tikzpicture}	
%\end{center}

%Hay distorsión no lineal si en la banda de trabajo:

\begin{itemize}
	\item de amplitud: $k \neq cte$, $k = k(f)$
	\item de fase: $t_o \neq cte$, $t_0 = t_o(f)$
\end{itemize}

\subsection{Distorsión No Lineal o Armónica}

%No se puede utilizar la respuesta en frecuencia $H(f)$, se utilizará la \textbf{característica de transferencia}.

\begin{IEEEeqnarray*}{rCl}
	y(t) = f(x(t)) \underbrace{=}_\textrm{\tiny{Serie de Taylor}}  \overbrace{a_0}^\textrm{C.C} & + & \overbrace{a_1 x(t)}^\textrm{\tiny{Término Lineal}} + \overbrace{a_2 x^2(t)}^\textrm{\tiny{Término Cuadrático}} + \\ & + & \overbrace{a_3 x^3(t)}^\textrm{\tiny{Término Cúbico}} + ... + a_n x^n(t)
\end{IEEEeqnarray*}

%\begin{center}
%	\begin{tikzpicture}[scale = 0.5]
%	\begin{axis}[axis lines = middle,xmin=-0.1,xmax = 3.2,ymin=-0.3,ymax=3, xticklabels={,,}, yticklabels={,,}] %,grid=both]
%		\addplot +[dirac] coordinates {(1, 2)};
%		
%		% Valores del eje X
%		\node at (axis cs:1, 0) [anchor=north] {$f_0$};
%		\node at (axis cs:0, 3) [anchor=north west] {$X(f)$};
%		\node at (axis cs:3.2, 0) [anchor=south east] {$f$};
%		
%	\end{axis}
%\end{tikzpicture}
%\hspace{2pt}
%$\rightarrow$
%\begin{tikzpicture}[scale = 0.5]
%	\begin{axis}[axis lines = middle,xmin=-0.1,xmax = 3.2, ymin=-0.3,ymax=3, xticklabels={,,}, yticklabels={,,}] %,grid=both]
%		\addplot +[dirac] coordinates {(1, 2)};
%		%\node at (axis cs:1,2) [anchor=north east] {test};
%		\addplot +[orange, dirac] coordinates {(2, 1.6)};
%		\node at (axis cs:2, 1.6) [anchor=north east] {$V_{d_3}$};
%		\addplot +[orange, dirac] coordinates {(3, 1.3)};
%		\node at (axis cs:3, 1.3) [anchor=north east] {$V_{d_3}$};
%		
%		% Valores del eje X
%		\node at (axis cs:1, 0) [anchor=north] {$f_0$};
%		\node at (axis cs:2, 0) [anchor=north] {$ 2 \cdot f_0$};
%		\node at (axis cs:3, 0) [anchor=north] {$3 \cdot f_0$};
%		\node at (axis cs:0, 3) [anchor=north west] {$Y(f)$};
%		\node at (axis cs:3.2, 0) [anchor=south east] {$f$};
%		
%		% Nota: "Amplitud de distorsión
%		\draw [<-] (axis cs:2,1.7)-- +(20pt,15pt) node[above] {Amplitud de distorsión};
%		\draw [<-] (axis cs:3,1.4)-- +(-20pt,30pt) node[above] {}; %{Amplitud de distorsión};
%	\end{axis}
%\end{tikzpicture}
%\end{center}

El grado del polinomio a la salida del sistema no lineal indica cuántas frecuencias nuevas van a ser generadas por dicho sistema.

\begin{itemize}
	\item \textbf{Coeficiente de Distorsión del Armónico n-ésimo $d_n$}:
		\begin{IEEEeqnarray*}{rCl}
			d_n & = & \frac{V_{d_n}}{V_1} \hspace{5pt} n = 2, 3, ..., k \\
			D_n & = & 20 \cdot \log_{10} \frac{V_{d_n}}{V_1}	
		\end{IEEEeqnarray*}
	\item \textbf{Atenuación del Armónico n-ésimo $A_n$}:
		\begin{equation*}
			A_n = 20 \cdot \log_10 \frac{V_1}{V_{d_n}} = -D_n
		\end{equation*}
	\item \textbf{Coeficiente de Distorsión Total $d$}:
		\begin{equation*}
			d = \sqrt{\sum_{n>1} d_n^2}
		\end{equation*}
	\item \textbf{Total Harmonic Distortion(THD)}:
		\begin{equation*}
			THD(\%) = \frac{1}{V_1} \sqrt{\sum_{n>1} V_{d_n}^2} \cdot 100 \%
		\end{equation*}
\end{itemize}

Si $V_1$ aumenta $\Delta (dB) \rightarrow V_{d_n}$ aumenta $n \cdot \Delta (dB)$. 


\subsection{Intermodulación}

\begin{center}
	\begin{tikzpicture}
		\draw (0,0) -- node[above left] {$x(t)$} (0.6,0);
		\node [draw, align=center] at (0.95,0) {$h(t)$};	
		\draw (1.3,0) -- node[above right] {$y(t) = a_0 + a_1 \cdot x(t)+ ... + a_n \cdot x^n(t)$} (1.9,0);
	\end{tikzpicture}	
\end{center}

A la salida del sistema aparecen nuevas frecuencias:

\begin{itemize}
	\item Armónicos: $2 \cdot f_1$, $2 \cdot f_2$, $3 \cdot f_1$, $3 \cdot f_2$, ..., $n \cdot f_1$, $n \cdot f_2$
	\item Combinación lineal de las frecuencias de x(t):
		\begin{itemize}
			\item Segundo orden: $f_1 + f_2$, $f_1 - f_2$, ...
			\item Tercer orden: $2 f_1 + f_2$, $f_1 - 2 f_2$, ... 
		\end{itemize} 
\end{itemize}

\begin{center}
	\begin{tikzpicture}[scale = 0.5]
	\begin{axis}[axis lines = middle,xmin=-0.1,xmax = 3.2,ymin=-0.3,ymax=3, xticklabels={,,}, yticklabels={,,}] %,grid=both]
		\addplot +[dirac, blue] coordinates {(1, 2)};
			\node at (axis cs:1, 2) [anchor=north east] {$V_1$};
		\addplot +[dirac, blue] coordinates {(1.5, 2)};
			\node at (axis cs:1.5, 2) [anchor=north east] {$V_2$};
		
		% Valores del eje X
		\node at (axis cs:1, 0) [anchor=north] {$f_1$};
		\node at (axis cs:1.5, 0) [anchor=north] {$f_2$};
		
		\node at (axis cs:0, 3) [anchor=north west] {$x(t)$};
		\node at (axis cs:3.2, 0) [anchor=south east] {$f$};
		
	\end{axis}
\end{tikzpicture}
\hspace{2pt}
$\rightarrow$
\begin{tikzpicture}[scale = 0.5]
	\begin{axis}[axis lines = middle,xmin=-0.1,xmax = 5, ymin=-0.4,ymax=3, xticklabels={,,}, yticklabels={,,}] %,grid=both]
		\addplot +[dirac, blue] coordinates {(1.5, 2)};
			\node at (axis cs:1.5, 2) [anchor=north east] {$V_1$};
			\node at (axis cs:1.5, 0) [anchor=north] {$f_1$};
		\addplot +[dirac, blue] coordinates {(2.75, 2)};
			\node at (axis cs:2.75, 2) [anchor=north east] {$V_2$};
			\node at (axis cs:2.75, 0) [anchor=north] {$f_2$};
		
			% Armónicos de f1 y f2
		\addplot +[dirac, green] coordinates {(3.75, 1)};
			\node at (axis cs:3.75, 1) [anchor=north east] {$V_{d_2}$};
			\node at (axis cs:3.75, 0) [anchor=north] {\tiny $2f_1$};
		\addplot +[dirac, green] coordinates {(4.75, 1)};
			\node at (axis cs:4.75, 1) [anchor=north east] {$V_{d_2}$};
			\node at (axis cs:4.75, 0) [anchor=north] {\tiny $2f_2$};
			
%		\addplot +[dirac, red] coordinates {(3, 1)};
%			\node at (axis cs:3, 1) [anchor=north west] {$V_{d_3}$};
%			\node at (axis cs:3, -0.2) [anchor=north] {\tiny $2f_2$};
%		\addplot +[dirac, red] coordinates {(4.5, 1)};
%			\node at (axis cs:4.5, 1) [anchor=north west] {$V_{d_4}$};
%			\node at (axis cs:4.5, 0) [anchor=north] {\tiny $3f_2$};
%		
			% Productos de intermodulación de Orden 2	
		\addplot +[dirac, black] coordinates {(0.5, 1.5)}; % f2 - f1
			\node at (axis cs:0.5, 1.5) [anchor=north east] {$V_{i_2}$};
			\node at (axis cs:0.35, -0.1) [anchor=north, rotate = 0] {\tiny $f_2 - f_1$};
		\addplot +[dirac, black] coordinates {(4.25, 1.5)}; % f1 + f2
			\node at (axis cs:4.25, 1.5) [anchor=north west] {$V_{i_2}$};
			\node at (axis cs:4.25, -0.25) [anchor=south, rotate = 0] {\tiny $f_1 + f_2$};
			
			% Productos de intermodulacion de Orden 3
		\addplot +[dirac, red] coordinates {(2.25, 1.5)}; % 2f2 - f1
			\node at (axis cs:2.25, 1.5) [anchor=north east] {$V_{i_3}$};
			\node at (axis cs:2.15, -0.15) [anchor=north, rotate = 0] {\tiny $2f_2 - f_1$};
		\addplot +[dirac, red] coordinates {(3.25, 1.5)}; % 2f1 + f2
			\node at (axis cs:3.25, 1.5) [anchor=north east] {$V_{i_3}$};	
			\node at (axis cs:3.15, -0.15) [anchor=north, rotate = 30] {\tiny $2f_1 + f_2$};
%		
%		% Valores del eje X
%		\node at (axis cs:1, 0) [anchor=north] {\tiny $f_1$};
%		\node at (axis cs:1.5, 0) [anchor=north] {\tiny $f_2$};
%		
%		\node at (axis cs:2, 0) [anchor=north] {\tiny $ 2 \cdot f_0$};
%		\node at (axis cs:3, 0) [anchor=north] {\tiny $3 \cdot f_0$};
%		\node at (axis cs:0, 3) [anchor=north west] {$y(t)$};
%		\node at (axis cs:5, 0) [anchor=south east] {$f$};
	
	\end{axis}
\end{tikzpicture}
\end{center}

\begin{center}
	Intermodulación orden $n$ $>$ Armónico orden $n$
\end{center}

\begin{itemize}
	\item \textbf{Coeficiente de Intermodulación enésimo ($i_n$)}:
		\begin{IEEEeqnarray*}{rCl}
			i_n & = & \frac{V_{d_1}}{V_1} = n \cdot d_n	\\
			I_n & = & 20 \cdot \log_{10} \frac{V_{i_n}}{V_1} = 20 \cdot \log_{10} n \cdot \frac{V_{d_1}}{V_1} = D_n + 20 \cdot \log_{10} n
		\end{IEEEeqnarray*}
\end{itemize}

Si $x(t)$ cambia y ahora tiene $\Delta (dB)$ menos:

	\begin{IEEEeqnarray*}{rCl}
		D'_n & = & D_n + (n - 1) \cdot \Delta \\
		I_N & = & D_n + 20 \cdot \log_{10} n \\
		I'_n & = & D'_n + 20 \cdot \log_{10} n = D_n + (n - 1) \Delta + 20 \cdot \log_{10} n = \\
		     & = & I_n + (n-1)\cdot \Delta	
	\end{IEEEeqnarray*}


\subsection{Diafonía}

\begin{center}
	\begin{tikzpicture}
		\node [draw, align=center, fill={rgb:black,1;white,2}, text = white] at (0,0) {$F_1$};
		\node [draw, align=center, fill={rgb:black,1;white,2}, text = white] at (0,-1) {$F_2$};	
	
		\node [draw, align=center, fill={rgb:black,1;white,2}, text = white] at (2,0) {$P_1$};
		\node [draw, align=center, fill={rgb:black,1;white,2}, text = white] at (2,-1) {$P_2$};
	
		\draw (0.25,0) -- node[above left] {} (1.75,0);
		\draw (0.25,-1) -- node[above left] {} (1.75,-1);
		\draw [blue,  -to, thick] (0.5,0) -- (1.5,0) node [right] {};
		\draw [red,  -to, thick] (0.5,0) .. controls (1.10,0) and (0.7,-1) .. (1.5,-1);
		\draw [red,  -to, thick] (0.5,0) .. controls (1.05,0) and (0.88,-1) .. (0.5,-1);
	\end{tikzpicture}
\end{center}

El circuito \textbf{perturbador} es el circuito en el que se genera la perturbación y el circuito \textbf{perturbado} es en el que se recibe la diafonía.\\

\textbf{Clasificación} de la diafonía:

\begin{itemize}
	\item Según como sea percibida la señal perturbadora en el circuito perturbado:
		\begin{itemize}
			\item Inteligible
			\item Ininteligible
		\end{itemize}
	\item Según el número de circuitos que atraviesa la señal perturbadora:
		\begin{itemize}
			\item Directa: No se atraviesan circuitos intermedios
				
%				\begin{center}
%					\begin{tikzpicture}
%						\node [draw, align=center, fill={rgb:black,1;white,2}, text = white] at (0,0) {$F_1$};
%						\node [draw, align=center, fill={rgb:black,1;white,2}, text = white] at (0,-1) {$F_2$};	
%	
%						\node [draw, align=center, fill={rgb:black,1;white,2}, text = white] at (2,0) {$P_1$};
%						\node [draw, align=center, fill={rgb:black,1;white,2}, text = white] at (2,-1) {$P_2$};
%	
%						\draw (0.25,0) -- node[above left] {} (1.75,0);
%						\draw (0.25,-1) -- node[above left] {} (1.75,-1);
%						\draw [red,  -to, thick] (0.5,0) .. controls (1.10,0) and (0.7,-1) .. (1.5,-1);
%					\end{tikzpicture}
%				\end{center}
			
			\item Indirecta: Se atraviesan uno o más circuitos intermedios
			
				\begin{itemize}
					\item Transversal
					
%						\begin{center}
%							\begin{tikzpicture}[scale = 0.7]
%								\node [draw, align=center, fill={rgb:black,1;white,2}, text = white] at (0,0) {$F_1$};
%								\node [draw, align=center, fill={rgb:black,1;white,2}, text = white] at (0,-1) {$F_2$};
%								\node [draw, align=center, fill={rgb:black,1;white,2}, text = white] at (0,-2) {$F_3$};		
%	
%								\node [draw, align=center, fill={rgb:black,1;white,2}, text = white] at (2,0) {$P_1$};
%								\node [draw, align=center, fill={rgb:black,1;white,2}, text = white] at (2,-1) {$P_2$};
%								\node [draw, align=center, fill={rgb:black,1;white,2}, text = white] at (2,-2) {$P_3$};
%	
%								\draw (0.25,0) -- node[above left] {} (1.75,0);
%								\draw (0.25,-1) -- node[above left] {} (1.75,-1);
%								\draw (0.25,-2) -- node[above left] {} (1.75,-2);
%								\draw [red,  -to, thick] (0.5,0) .. controls (1.10,0) and (0.7,-2) .. (1.5,-2);
%							\end{tikzpicture}
%						\end{center}
					
					\item Longitudinal
					
%						\begin{center}
%							\begin{tikzpicture}[scale = 0.7]
%								\node [draw, align=center, fill={rgb:black,1;white,2}, text = white] at (0,0) {$F_1$};
%								\node [draw, align=center, fill={rgb:black,1;white,2}, text = white] at (0,-1) {$F_2$};
%								\node [draw, align=center, fill={rgb:black,1;white,2}, text = white] at (0,-2) {$F_3$};		
%	
%								\node [draw, align=center, fill={rgb:black,1;white,2}, text = white] at (2,0) {$P_1$};
%								\node [draw, align=center, fill={rgb:black,1;white,2}, text = white] at (2,-1) {$P_2$};
%								\node [draw, align=center, fill={rgb:black,1;white,2}, text = white] at (2,-2) {$P_3$};
%	
%								\draw (0.36,0) -- node[above left] {} (1.64,0);
%								\draw (0.36,-1) -- node[above left] {} (1.64,-1);
%								\draw (0.36,-2) -- node[above left] {} (1.64,-2);
%								
%								\draw [red, thick] (0.36,0) -- (0.44,0);
%								\draw [red, thick] (0.44,0) .. controls (0.53,-0.05) and (0.58, -0.95) .. (0.7,-1);
%								\draw [red, thick] (0.7,-1) -- (0.95,-1);
%								\draw [red, thick] (0.95,-1) .. controls (1.1,-1.05) and (1.25, -1.9) .. (1.5,-2);
%								\draw [red, -to, thick] (1.5,-2) -- (1.64,-2);
%							\end{tikzpicture}
%						\end{center}

						\end{itemize}
					\item Según el extremo que recibe la perturbación
					
						\begin{itemize}
							\item Paradiafonía: Perturbación recibida en el mismo extremo que se genera la señal, conocida como \textit{NEXT} (Near End Cross Talk).
							
%								\begin{center}
%%									\begin{tikzpicture}[scale = 0.7]
%										\node [draw, align=center, fill={rgb:black,1;white,2}, text = white] at (0,0) {$F_1$};
%										\node [draw, align=center, fill={rgb:black,1;white,2}, text = white] at (0,-1) {$F_2$};	
%		
%										\node [draw, align=center, fill={rgb:black,1;white,2}, text = white] at (2,0) {$P_1$};
%										\node [draw, align=center, fill={rgb:black,1;white,2}, text = white] at (2,-1) {$P_2$};
%	
%										\draw (0.36,0) -- node[above left] {} (1.64,0);
%										\draw (0.36,-1) -- node[above left] {} (1.64,-1);
%										\draw [red,  -to, thick] (0.35,0) .. controls (1.10,0) and (1.1,-1) .. (0.35,-1);
%									\end{tikzpicture}
%								\end{center}
							
							\item Telediafonía: Recibida en el extremo opuesto. Conocida como \textit{FEXT} (Far End Cross Talk).
							
%								\begin{center}
%									\begin{tikzpicture}[scale = 0.7]
%										\node [draw, align=center, fill={rgb:black,1;white,2}, text = white] at (0,0) {$F_1$};
%										\node [draw, align=center, fill={rgb:black,1;white,2}, text = white] at (0,-1) {$F_2$};	
%	
%										\node [draw, align=center, fill={rgb:black,1;white,2}, text = white] at (2,0) {$P_1$};
%										\node [draw, align=center, fill={rgb:black,1;white,2}, text = white] at (2,-1) {$P_2$};
%	
%										\draw (0.36,0) -- node[above left] {} (1.64,0);
%										\draw (0.36,-1) -- node[above left] {} (1.64,-1);
%										\draw [red,  -to, thick] (0.5,0) .. controls (1.10,0) and (0.7,-1) .. (1.5,-1);
%									\end{tikzpicture}
%							\end{center}
				\end{itemize}
		\end{itemize}
\end{itemize}

Parámetros de medida de la diafonía:

\begin{itemize}
	\item $P_1$: Potencia de la señal en un punto del circuito perturbador.
	\item $P_2$: Potencia de la señal perturbada medida en un punto equivalente del circuito perturbado.
\end{itemize}

\begin{itemize}
	\item Relación de Diafonía ($R_d$):
		\begin{equation*}
			R_d = 10 \log_{10} \left( \frac{P_2}{P_1} \right)
		\end{equation*}
	\item Atenuación de Diafonía ($A_d$):
		\begin{equation*}
			A_d = 10 \log_{10} \left( \frac{P_1}{P_2} \right) = - R_d
		\end{equation*}
	\item Cross Talk Unit (CU):
		\begin{equation*}
			CU = 20 \log_{10} \left( \frac{V_2}{V_1} \cdot 10^6 \right) = 120 - A_d
		\end{equation*}
\end{itemize}


\begin{center}
	\begin{tikzpicture}[scale = 0.3]
	
	\draw[red, very thick] (0.5,5) -- (6, 3.5) node[right] {Telediafonía};;
	\draw[green, very thick] (0.5,3) -- (6, 2) node[right] {Paradiafonía};
	
	\draw[gray, dashed, thick] (0.75,5) -- (0.75, 0.5);
	\draw[gray, dashed, thick] (5.75,3.5) -- (5.75, 0.5);
	
	\begin{axis}[axis lines = middle,xmin=-0.1,xmax = 3.2,ymin=-0.3,ymax=3, xticklabels={,,}, yticklabels={,,}] ,grid=both]
		
		%TERMINAR AQUI
		
			% Ejes
		\node at (axis cs:0, 3) [anchor=north west] {$A_d$};
		\node at (axis cs:3.2, 0) [anchor=south east] {$f$};
		
		\draw[red, very thick] (0,0) -- (2.5,1.5);% -- (0,1);
		\draw (0,0) -- (2,2);
		
	\end{axis}
	\end{tikzpicture}
\end{center}

\subsection{Ruido}

\subsubsection{Ruido Térmico}

\begin{IEEEeqnarray*}{rCl}
	n & = & k \cdot t \cdot B \\
	N & = & 10 \cdot \log_{10} (ktB)	
\end{IEEEeqnarray*}

\begin{itemize}
	\item $k$: Constante de Boltzmann ($1.38 \cdot 10^{-23} W/K/Hz$)
	\item $t$ (Kelvin): Temperatura
	\item $b$ (Hz): Ancho de banda
\end{itemize}

\subsubsection{Ruido en un Cuadripolo}

\begin{center}
	\begin{tikzpicture}[scale = 0.7]		
		\draw [draw, align=center, fill={rgb:black,1;white,2}, text = white] (0,0) rectangle (2,2) node[pos=.5] {Cuadripolo}; 
	
		\draw (-0.5,0.5) -- node[above left] {$n_e$} (0,0.5);
		\draw (-0.5,1.5) -- node[above left] {$S_e$} (0,1.5);
				
		\draw (2,0.5) -- node[above right] {$S_s = S_e \cdot g$} (2.5,0.5);
		\draw (2,1.5) -- node[above right] {$n_s = n_e \cdot g + n_{interno}$} (2.5,1.5);
	\end{tikzpicture}
\end{center}

Parámetros de caracterización del ruido:

\begin{itemize}
	\item \textbf{Temperatura Equivalente de Ruido ($T_{eq}$)}: Temperatura a la que tendría que estar la entrada del circuito para que a la salida se vea el mismo ruido que se produce suponiendo que el cuadripolo es ideal.
	
		\begin{equation*}
			n_{int} = k \cdot t_{eq} \cdot b \cdot g
		\end{equation*}
	
	\item \textbf{Factor de Ruido en un Cuadripolo ($f$)}: Cociente entre la potencia de ruido a la salida comparada con la potencia de ruido que habría a la salida si la entrada estuviera a temperatura estándar y el cuadripolo no añadiera ruido térmico.
		
		\begin{IEEEeqnarray*}{rCl}
			f & = & \frac{n_s}{k \cdot t_o \cdot b \cdot g} = 1 + \frac{t_{eq}}{t_o} \hspace{10px} f= \frac{\left( \frac{S}{N} \right)_e}{\left( \frac{S}{N} \right)_s} \\	
			F & = & 10 \cdot \log_{10} (f) = \left( \frac{S}{N} \right)_e - \left( \frac{S}{N} \right)_s
		\end{IEEEeqnarray*}
\end{itemize}

Relación entre $t_{eq}$ y $f$:

	\begin{equation*}
		t_{eq} = t_0 \cdot (f - 1) \hspace{10px} f = 1 + \frac{t_{eq}}{t_o}
	\end{equation*}

\subsubsection{Asociación de Cuadripolos}

\begin{center}
	\begin{tikzpicture}[scale = 0.65]
		
		\draw [draw, align=center, fill={rgb:black,1;white,2}, text = white] (0,0) rectangle (2,2) node[pos=.5] {$g_1$ $f_1$ $t_{eq_1}$}; 
	
		\draw (-0.5,0.5) -- node[above left] {} (0,0.5);
		\draw (-0.5,0.5) -- (-0.5,0.25);
		\draw (-0.575,0.25) -- (-0.425,0.25) -- (-0.425,-0.25) -- (-0.575,-0.25) -- (-0.575,0.25); % Resistencia
		\draw (-0.5,-0.25) -- (-0.5,-0.5);
		\draw (-0.5,-0.25) -- (-0.5,-0.5);
		\draw (-0.625,-0.5) -- (-0.375,-0.5);
		\draw (-0.5,-0.25) node[above left] {$t_{eq_1}$} (-0.5,0);

		\draw (-1.5,1.5) -- node[above,xshift=-0.4cm] {$n_e = k \cdot t_e \cdot b$} (0,1.5);
		
				
		%\draw (2,1.5) -- node[above right] {$k \cdot t_e \cdot b \cdot g_1$} (2.5,1.5);
		\draw (2,1.5) -- node[above right] {} (4,1.5);
		
		
		\draw [draw, align=center, fill={rgb:black,1;white,2}, text = white] (3.5,0) rectangle (5.5,2) node[pos=.5] {$g_2$ $f_2$ $t_{eq_2}$}; 
	
		\draw (3,0.5) -- node[above left] {} (3.5,0.5);
		\draw (3,0.5) -- (3,0.25);
		\draw (3.075,0.25) -- (2.925,0.25) -- (2.925,-0.25) -- (3.075,-0.25) -- (3.075,0.25); % Resistencia
		\draw (3,-0.25) -- (3,-0.5);
		\draw (3,-0.25) -- (3,-0.5);
		\draw (3.125,-0.5) -- (2.875,-0.5);
		\draw (3,-0.25) node[above left] {$t_{eq_2}$} (3,0);
		
		\draw (5.5,1.5) -- node[above right] {} (7.5,1.5);
		
		\draw [draw, align=center, fill={rgb:black,1;white,2}, text = white] (7,0) rectangle (9,2) node[pos=.5] {$g_3$ $f_3$ $t_{eq_3}$}; 
	
		\draw (6.5,0.5) -- node[above left] {} (7,0.5);
		\draw (6.5,0.5) -- (6.5,0.25);
		\draw (6.575,0.25) -- (6.425,0.25) -- (6.425,-0.25) -- (6.575,-0.25) -- (6.575,0.25); % Resistencia
		\draw (6.5,-0.25) -- (6.5,-0.5);
		\draw (6.5,-0.25) -- (6.5,-0.5);
		\draw (6.625,-0.5) -- (6.375,-0.5);
		\draw (6.5,-0.25) node[above left] {$t_{eq_3}$} (6.5,0);
		
		\draw (9,1.5) -- node[above right] {} (9.5,1.5);

	\end{tikzpicture}
\end{center}

\begin{equation*}
	n_s = k \cdot b \cdot g_1 \cdot g_2 \cdot g_3 \left( t_o + t_{eq_1} + \frac{t_{eq_2}}{g_1} + \frac{t_{eq_3}}{g_1 \cdot g_2} \right)	
\end{equation*}

Fórmula de Friis:

	\begin{equation*}
		f_T = f_1 + \frac{f_2 - 1}{g_1} + \frac{f_3 - 1}{g_1 \cdot g_2} + ... + \frac{f_n - 1}{g_1 \cdot g_2 \cdot ... \cdot g_{n-1}}
	\end{equation*}

\section{\underline{Tráfico}}

\subsection{Tráfico Telefónico}

El \textbf{tráfico} es una medida del conjunto de peticiones de uso y de ocupación de los recursos de un determinado sistema de telecomunicaciones.

\begin{itemize}
	\item \textbf{Ritmo de afluencia de las llamadas} ($\lambda$, $\frac{\mbox{Número de Llamadas}}{\mbox{Tiempo}}$)
	\item \textbf{Tiempo medio de duración} de las llamadas ($T_m$)
	\item \textbf{Volumen de Tráfico}: Tiempo de ocupación de los recursos, para $N$ circuitos:
		\begin{equation*}
			V(N) = \sum_i V_i
		\end{equation*}
		Se mide en:
		\begin{itemize}
			\item LLR: Llamadas reducidas - 120 segundos $\rightarrow 1(E) = 30 \frac{LLR}{H}$
			\item CCS: \textit{Century Call Seconds} - 100 segundos $\rightarrow 1(E) = 30 \frac{LLR}{H}$
		\end{itemize}
	\item \textbf{Intensidad de Tráfico (\textit{A})}: Volumen a lo largo de un periodo de observación, se mide en \textit{Erlangs}.
		\begin{equation*}
			A = \frac{t_{\mbox{ocupación}}}{t_{\mbox{observación}}} = \lambda \cdot t_{medio}
		\end{equation*}
	\item \textbf{Tiempo de Observación para las medidas del tráfico (\textit{A})} El tráfico depende tanto de la duración como de la distribución de llegada de las llamadass
\end{itemize}

\subsection{Bloqueo - Llamadas Perdidas - GoS - Disponibilidad}

\begin{itemize}
	\item \textbf{Tráfico Ofrecido ($A_O$)}: Tráfico que soportaría la red si fuera capaz de servir todas las solicitudes de servicio.
	\item \textbf{Tráfico Bloqueado ($A_B$)}: Tráfico rechazado por ocupación de todos los circuitos $B \cdot A_O$.
	\item \textbf{Tráfico Cursado ($A_C$)}: Tráfico servido por la red $A_O (1-B)$.
\end{itemize}

\begin{itemize}
	\item En un sistema sin pérdidas: $A_O = A_C$.
	\item En un sistema con pérdidas: $A_O = A_C + A_B$.
	\item Con $N$ circuitos o servidores, $\rho = \frac{A}{N}$ será el tráfico, ofrecido/cursado, por circuito o servidor.
\end{itemize}

Un \textbf{conmutador tiene disponibilidad total} cuando cada entrada tiene acceso a cada una de las salidas.

\subsection{Distribuciones Estadísticas para Fuentes de Tráfico}

\begin{itemize}
	\item Duración de llamada constante: redes de conmutación de paquetes
	\item Duración de llamadas exponencial negativa: conversación telefónica
\end{itemize}

%\subsection{Modelos de Gestión de Llamadas Bloqueadas}
%
%\begin{itemize}
%	\item \textbf{Lost Calls Held (\textit{LCH})}: Práctica norteamericana, la llamada se pierde y el usuario volverá a intentarlo de forma inmediata. El segundo intento está estadísticamente relacionado con el primero.
%	\item \textbf{Lost Calls Cleared (\textit{LCC})}: Práctica europea, la llamada se pierde y el usuario dejará pasar cierto tiempo antes de volver a intentarlo. El segundo intento está considerado como una petición aleatoria más.
%	\item \textbf{Lost Calls Delayed (\textit{LCD})}: La llamada no se pierde, existe una cola de espera hasta que se libere algún acceso.
%	\item \textbf{Lost Calls Retried (\textit{LCR})}: Variación de \textit{LCC}, es un caso especial
%\end{itemize}

\subsection{Modelo de Llamadas Perdidas Despejadas}

\subsubsection{Modelo LLC $\rightarrow$ Erlang-B}

Distribución Erlang B para el cáclulo de la probabilidad de bloqueo

	\begin{equation*}
		B(N,A) = \frac{\frac{A^N}{N!}}{\sum_{i=0}^N \frac{A^i}{i!}}
	\end{equation*}

\qquad \textit{$B(N,A)$: Probabilidad de Bloqueo}\\
\qquad \textit{$N$: Número de órganos} \\
\qquad \textit{$A$: Tráfico ofrecido}

\subsubsection{Sistemas con Retardo - LCD}

Las solicitudes de servicio que encuentran todos los servidores ocupados son puestas en una cola. Los servidores verán un ritmo constante de llegadas. Parámetros:

\begin{itemize}
	\item Tiempo de Servicio o Tiempo de Ocupación ($T_O$).
	\item Tiempo de Espera ($T_w$).
	\item Tiempo total en el sistema ($T_s = T_m + T_w$).
\end{itemize}

%Terminología de Colas
%
%\begin{enumerate}
%	\item \textbf{Input Specification}:
%		\begin{itemize}
%			\item G: General (no assumptions)
%			\item M: Purely random
%		\end{itemize}
%	\item \textbf{Service Time Distribution}:
%		\begin{itemize}
%			\item G: General (no assumptions)
%			\item M: Negative Exponential
%			\item D: Constant
%		\end{itemize}
%	\item \textbf{Number of Sources}:
%		\begin{itemize}
%			\item M: Finite
%			\item  : Indinite
%		\end{itemize}
%	\item \textbf{Queue Length}:
%		\begin{itemize}
%			\item L: Finite Length
%			\item  : Infinite Length
%		\end{itemize}
%\end{enumerate}

\subsubsection{Sistemas M/M/N (Erlang-C)}

Llegadas aleatorias, tiempo de servicio exponencial y $N$ servidores.

	\begin{equation*}
		p(t_w > t) = C(N,A) \cdot e^{- \frac{(N-A)t}{T_m}}
	\end{equation*}
	
	\begin{equation*}
		T_w = \frac{C(N,A) \cdot T_m}{N - A}
	\end{equation*}
	
Número medio de usuarios en cola:

	\begin{equation*}
		u_w = \lambda \cdot T_w
	\end{equation*}

%\begin{equation*}
%	C(N,A) = \frac{N \cdot E_B}{[ N - A (1 - E_B) ]}
%\end{equation*}
%
%\begin{itemize}
%	\item $N$: Números de servidores del sistema
%	\item $A$: Tráfico ofrecido al sistema
%\end{itemize}

\subsubsection{Sistemas M/M/1}

Llegadas aleatorias, tiempo de servicio exponencial y $1$ servidor.

	\begin{equation*}
		C(N,A) = A = \rho
	\end{equation*}

	\begin{equation*}
		p(t_w > t) = A \cdot e^{- \frac{(1-A)t}{T_m}}
	\end{equation*}
	
	\begin{equation*}
		T_w = \frac{\rho \cdot T_m}{1 - \rho}
	\end{equation*}

\subsubsection{Sistemas M/D/1}

Llegadas aleatorias, tiempos de servicio fijos y 1 servidor.

	\begin{equation*}
		T_w = \frac{\rho \cdot T_m}{2 \cdot (1 - \rho)}
	\end{equation*}

	\begin{equation*}
		p(t_w > 0) = A = \rho
	\end{equation*}
	
	
\end{multicols}
\end{landscape}


\begin{framed}
	\begin{center}
    	\Large{\underline{Sistemas de Telecomunicación}} \\
    	\scriptsize{3º Ingeniería de Telecomunicaciones | UPV/EHU}\\
     	%Actualizado por última vez el \today \\
     	"\textsl{Under-promise and over-deliver}." \\
     	%\hspace{5 pt} \\
     	\small{\textbf{Javier de Martín -- 2016}}
	\end{center}
\end{framed}

%%%%%%%%%%%%%%%%%%%%%%%%%%%%%%%%%%%%%%%%%%%%%%%%%%%%%%%%%%%%%%%%%
% Tema 4

\hrulefill

\section{4. Redes de Acceso}

\hrulefill

``\textit{El objetivo es asegurar la comunicacioón oral entre los usuarios del servciio atendiendo a lso estándares de la ITU que fijan las normas para obtener un servicio mínimo de calidad}''

Para proveer el servicio es necesario:

\begin{itemize}
	\item Red o conjunto de medios que posibiliten el \textbf{acceso a los usuarios del servicio}: terminales de abonado, medios de transmisión (bucle local) y centrales de acceso y conmutación (centrales locales - remotas).
	\item Red o conjunto de medios que interconectan los medios de acceso de los usuarios para proveer conectividad total entre los usuarios: Redes de transporte y centros de conmutación.
\end{itemize}

La \text{red} se define como un método de interconexión de centrales para poder proveer conectividad total. En telefonía hay tres métodos utilizados para interconectar centrales:

	\begin{itemize}
		\item Malla (todos con todos): Eficiente en zonas con usuarios concentrados cerca de los nodos y si el tráfico de usuarios es alto, el número de enlaces es alto $\frac{N(N-1)}{2}$ y la escalabilidad es compleja.
		\item Estrella (a través de un centro de y estrella doble (varias estrellas a través de un centro de tránsito de segundo orden): Menor número de conexiones ($N$), necesita nodos intermedios de conmutación, óptimo en lugares con poco tráfico, permite dar acceso a zonas aisladas y su escalabilidad es sencilla.
		\item Estrella doble: Varias estrellas a través de un centro de tránsito de segundo orden.
	\end{itemize}

\subsection{RDI: Red Digital Integrada}

	\begin{center}
		\includegraphics[scale=0.2]{images/RDII}
	\end{center}

La \textbf{red de acceso} son centrales digitales remotas. Se componen de \textbf{concentradores}: son equipos de conmutación para dar accesos a abonados alejados de la central, las funciones de conmutación las realiza la central autónoma, el camino telefónico de una llamada local pasa por la central autónoma.

	\begin{center}
		\includegraphics[scale=0.2]{images/RedAcceso}
	\end{center}

\textbf{Unidades remotas de abonados}: son equipos con más capacidad de accesos que el concentrador, están conectados a la central autónoma mediante sistemas MIC por cable o bien enlaces de fibra, su funcionamiento es más autónomo que en el caso de los concentradores y tienen capacidad de conmutación para los enlaces locales.

	\begin{center}
		\includegraphics[scale=0.2]{images/RDI}
	\end{center}
		
\hrulefill		

\subsection{Terminal Telefónico}

\subsubsection{Transmisores y Receptores}

El primer sistema basado en un transmisor, un receptor y una batería dispuestos en serie funcionaba por le principio de resistencia variable. El problema era la potencia del transmisor ya que no existían amplificadores. En la actualizad el transmisor es un micrófono de carbón (no lineal y muy sensible).

\subsubsection{El bucle de Abonado}

El \textbf{bucle de abonado} es una conexión física con la central telefónica, inicialmente era un único hilo (aprovechando la tierra) y actualmente son dos hilos (cable de pares). Inicialmente nació como líneas dedicadas entre dos usuarios, posteriormente fueron líneas de conmutación manual (centralitas) y finalmente son centrales de conmutación (strowger).

\subsubsection{La Bobina de Inducción}

Inicialmente eran un transmisor y un receptor en serie. Tenían \textbf{problemas} como la impedancia del altavoz, difícil adaptación de impedancias y la corriente continua que circulaba a través del receptor reducía su eficiencia. Como \textbf{solución} se utiliza la bobina de inducción, se aíslan los TX y RX facilitando la adaptación y evitando que circule corriente continua por el altavoz.

\subsubsection{El Efecto Local (Sidetone)}

Sigue habiendo un \textbf{problema}, las señales del micrófono se escuchan en el auricular. Hay un tercer camino de alta ganancia entre el altavoz y auricular que hace que el ususario se escuche a sí mismo demasiado alto y tienda a hablar más bajo.
 	
	\begin{center}
		\includegraphics[scale=0.2]{images/Sidetone}
	\end{center}

La adaptación no es perfecta ($Z_{L}$ variable), pero un cierto nivel de sidetone es beneficioso.

\subsubsection{Alimentación y llamada}

La alimentación se realiza desde la central local con $48V (CC)$. El consumo de corriente del terminal permite identificar y dar servicio a los usuarios (estados: \textit{on-hook}, colgado, y \textit{off-hook}, descolgado). El timbre está conectado en paralelo y es de alta impedancia, la señal que se envía a la central es de $75 V_{rms}$ y $50Hz$. 

\subsubsection{Marcación}

Se realiza por dos formas: \textbf{por pulsos o decamétrica} o \textbf{DTMF} (\textit{Dual Tone Multifrecuency}).

\subsubsection{Otros tipos de terminales: Módems}

La ITU define dos tipos de módems:

\begin{itemize}
	\item \textbf{Modems Digitales}: Envía señales G.711 y recibe señales V.34 codificadas con el estándar V.34. Se conecta a una red con conmutación digital con un interfaz digital.
	\item \textbf{Modems Analógicos}: Generan señales V.34 y reciben señales G.711 que han sido decodificadas en una central local de abonado telefónico y preparadas para su envío a través de un bucle de abonado.
\end{itemize}

\begin{center}
	{\scriptsize G.711 - Pulse Code Modulation (PCM) of voice frequencies\\
	V.34 - A modem operating (up to 33.600 bit/s) for use in 2-wire analog PSTN}
\end{center}


	\begin{center}
		\includegraphics[scale=0.2]{images/SpecsModem}
	\end{center}

Los módems tienen distintas especificaciones que varían según las redes.

\textbf{Funciones} de un modem:

	\begin{enumerate}
		\item \textbf{Circuitos para compatibilizar la señalización RDI}: Descolgado, marcado y detección de tono marcado.
		\item \textbf{Circuitos de envío de la señal de datos} en la banda de frecuencias vocal
		\item \textbf{Transmisión a cuatro hilos equivalentes}: Diferentes portadoras en cada sentido e inclusión de filtros para minimizar interferencias.
	\end{enumerate}
	
	\begin{center}
		\includegraphics[scale=0.2]{images/FuncModem}
	\end{center}
	
\subsubsection{Operación de un Modem}

El módem receptor espera en answer mode, en el otro extremo espera en call mode y se simula el off-hook escuchando el tono de invitación a marcar y se envían pulsos o tonos para marcar. El modem en modo answer detecta las señales de llamada, simula el off-hook y envía una portadora. El modem en modo llamada envía una portadora y se intercambian los datos ambos.

\subsubsection{Módem V.34}

Permite mayor número de bits por símbolo y permite modulación QAM transmitiendo información en la amplitud y en la fase, incluye códigos de corrección y permite compresión. Estos módems se diseñaron para optimizar la situación en la que ambos lados de la comunicación eran analógicos.

	\begin{center}
		\includegraphics[scale=0.2]{images/ModemV34}
	\end{center}

Este tipo de módems está \textbf{limitado} por el ruido de cuantificación. El estándar V34 no tiene en cuenta la arquitectura real de la conexión entre la red telefónica y el proveedor de acceso a internet, siendo el bucle local la única parte analógica de la transmisión, el ruido de cuantificación sólo está presente en la parte del enlace ascendente.

\subsubsection{Módem V90}

Estándar doméstico desde 1998, el principio de funcionamiento se basa en que el ISP tiene una conexión digital con la central local. Por lo tanto se asume que solo existe una conversión analógico-digital en el camino hasta el ISP.

	\begin{center}
		\includegraphics[scale=0.2]{images/ModemV90}
	\end{center}

\subsubsection{Comunicaciones de Facsímil a través de la RTB}

Los terminales de facsímil siguen estándares de la ITU-T. Los faxes se dividen en varios grupos según si el escaneado es digital y su velocidad de transmisión.

\hrulefill

\subsection{Bucle de Abonado}

Medio físico que une la red telefónica con el terminal de abonado. En la RDI este acceso es analógico y basado en cable de pares, representa gran parte del coste de una red. Existen tecnologías alternativas pero sigue siendo el predominante en usuarios domésticos. Va a seguir implantada durante mucho teimpo $\rightarrow$ demanda de servicio de voz únicamente.

Distintos medios de acceso por pares:

	\begin{center}
		\includegraphics[scale=0.2]{images/BucleIntro}
	\end{center}

\subsubsection{Distribución entre la central y el abonado}

El interfaz con la red telefónica: La central local

	\begin{center}
		\includegraphics[scale=0.2]{images/CentralLocal}
	\end{center}

Esquema de distribución de cable de pares

	\begin{center}
		\includegraphics[scale=0.2]{images/Distribucion}
	\end{center}	

\subsubsection{Bridged - Taps}

Método de cableado para redes telefónicas. Un par de cables ``aparecerá'' en distintas localizaciones de terminales permitiendo a la compañía telefónica asignar ese par a cualquier cliente que esté cerca de ese terminal. Una vez ese cliente se desconecta, el par se convierte en disponible para cualquiera de los terminales. Esta conexión es una ``T'' en el cable, no tiene bobina híbrida o un componente de adaptación de impedancias produciendo reflexiones de la señal en el cable.

	\begin{center}
		\includegraphics[scale=0.2]{images/Taps}
	\end{center}

\subsubsection{Características de los cables }

Los cables se clasifican según su galga (diámetro), según su categoría se establece su velocidad de transmisión y su aplicación varía. La distancia máxima del bucle del abonado está limitada por \textbf{limitación óhmica} y \textbf{limitación por pérdidas (atenuación)}.

\subsubsection{Centrales locales: Interfaz con el bucle de abonado}

Central Local

	\begin{center}
		\includegraphics[scale=0.2]{images/CentralFuncional}
	\end{center}
	
La central local tiene como \textbf{funciones} el \textbf{ mantenimiento} en la que supervisa las líneas de abonado y los enlaces, \textbf{funciones de operación} en las que maneja datos administrativos y datos estadísticos y puede añadir \textbf{servicios de valor añadido} si así lo desea.

	\begin{center}
		\includegraphics[scale=0.2]{images/CentralFuncional2}
	\end{center}

El interfaz con la red telefónica: La Central Local

	\begin{center}
		\includegraphics[scale=0.2]{images/InterfazRed}
	\end{center}

\subsubsection{Perturbaciones en el bucle de abonado}

La \textbf{bobina híbrida} permite separar los canales de ida y vuelta. Las conexiones entre abonados son a dos hilos y las conexiones entre centrales son a cuatro. Al ser elementos ideales, en las redes telefónicas se ocasionan los problemas de ECO y CANTO.

	\begin{center}
		\includegraphics[scale=0.2]{images/BobinaHibrida}
	\end{center}
	
\textbf{Pérdidas de retorno}:

	\begin{equation*}
		A_{R} =  20 \cdot \log \left( \left| \frac{1}{\rho} \right| \right) = 20 \cdot \log \left| \frac{Z_{L} + Z_{E}}{Z_{E} - Z_{L}} \right|
	\end{equation*}

\textbf{Atenuación Transhíbrida}:

	\begin{equation}
		A_{TH} = 2 \alpha + A_{R}
	\end{equation}
	
\textbf{Estudio de la Inestabilidad}:	
	
Pérdida entre extremos a 2 hilos: $T = 2 \alpha + L - G$

Margen de estabilidad $S = \frac{M}{2}$. El bucle a cuatro hilos será estable si no tiene ganancia, es decir, si $M = 0$. Habitualmente $S = 3 dB$.	
	
En la bobina híbrida aparecen dos tipos de \textbf{eco}:

	\begin{itemize}
		\item \textbf{Eco del hablante}: el hablante escucha un eco de su propia voz
		\item \textbf{Eco del oyente}: el oyente escucha un eco de la señal que escucha
	\end{itemize}
	
Para solucionar estos problemas se utilizan dos componentes. \textbf{NES} (\textit{Network Echo Supressor}) anula la línea de transmisión cuando se detecta la señal recibida. Es equivalente a una comunicación semi-dúplex.
	
	\begin{center}
		\includegraphics[scale=0.2]{images/NES}
	\end{center}
	
El \textbf{NEC} (\textit{Network Echo Canceller}) elimina en la rama de transmisión la señal de eco calculada a partir de la rama de recepción.

	\begin{center}
		\includegraphics[scale=0.2]{images/NEC}
	\end{center}

\textbf{Interfaces PSTN e ISDN en una central local}. ISDN es Inegrated Service Data Network y PSTN Public Switched Telephone Network.

	\begin{center}
		\includegraphics[scale=0.35]{images/ISDN}
	\end{center}

\textbf{Interfaz V1} acceso básico para la RDSI, sus funciones básicas son: $2B + D$ ($2 \cdot 64 kbps + 16 kbps$ canalización), temporización y sincronismo de trama, activación y desactivación del terminal de línea, operación y mantenimiento, alimentación... \textbf{V2 Interfaz para la conexión con concentradores}: 2048 kbit/s, $30B + D$. \textbf{V3 similar a V2 pero pensado para interfaz con centralitas (PABX)}: $30B + D$ a 2048 kbit/s y también $23B+D$ a 1544 kbit/s. \textbf{V4}: Interfaz con redes privadas no especificadas por la ITU-T. \textbf{V5}: Interfaz genérico entre la red de acceso y la central local, especifica los interfaces básicos para el acceso analógico y acceso RDSI. Incluye el control de los canales de acceso y funciones de señalización. Se utiliza en DLCs.

\subsubsection{Perturbaciones en el bucle de abonado}

Existen diversas fuentes de perturbación para las señales que viajan por el cable de pares en el bucle de abonado.

	\begin{itemize}
		\item \textbf{Distorsión de canal}: Causado por la variación no lineal en frecuencia de las velocidades de transmisión en el cable de pares.
		\item \text{Atenuación}: Causado por el efecto Joule y por pérdida de energía por radiación electromagnética.
		\item \textbf{Diafonía}: Causado por el acoplamiento de señales entre pares.
		\item \textbf{Ruido}: Térmico (movimiento pseudoaleatorio de electrones) o Impulsivo (equipos y ruido humano).
		\item \textbf{Interferencia Electromagnética}: Acoplamiento de señales de radiofrecuencia en el cable de pares.
		\item \textbf{Eco}: Causado por desequilibrios en la bobina híbrida en el paso de dos hilos a cuatro hilos.
	\end{itemize}

\subsubsection{Atenuación}

Cualquier señal al propagarse por un conductor no ideal pierde potencia. La pérdida en decibelios en el bucle de abonado es proporcional a la distancia y a la atenuación del cable de pares. La atenuacion se da en $\frac{dB}{Km}$ o $\frac{dB}{100 m}$ y depende fundamentalmente de las características del cable (conductor) y su grosor. La atenuación no es constante con la frecuencia. En el cable de pares aumenta rápidamente la atenuación kilométrica a frecuencias altas. Para la transmisión de la señal vocal no es un problema en los bucles de abonado típicos, el problema surge en la transmisión de datos utilizando dicho cable.

\subsubsection{DLC (Digital Loop Carriers)}

El bucle de abonado es el medio físico que une la central local con el terminal de abonado. Existen varios medios para unir estos abonados con la central: Reducción del número de cables (concentración), acortamiento del bucle de abonado, posibilidad de equipos de intemperie, densidades de población baja, reemplazar planta exterior obsoleta, introducción de nuevos servicios en puntos donde las centrales no los soportan. Los medios físicos no tienen por qué ser cables de pares, pueden ser de acceso por radio, fibra óptica...

	\begin{center}
		\includegraphics[scale=0.2]{images/DLC1}
	\end{center}
	
	\begin{center}
		\includegraphics[scale=0.2]{images/DLC2}
	\end{center}

\textbf{Protocolo V.5}: Definido para el interfaz con la central local.

	\begin{itemize}
		\item \textbf{V.5-1}: Enlace digital de 2Mbps, necesita software necesario para la transmisión de los eventos de línea en los 30 puertos asociados a la interfaz. Los nodos extremos intercambias mensajes referentes a los protocolos.
		\item \textbf{V.5-2}: Conjunto de enlaces de 2Mbps, hasta 16 enlaces. La asignación se realiza llamada a llamada (concentración), los extremos intercambian información.
	\end{itemize}
	
	\begin{center}
		\includegraphics[scale=0.2]{images/DLCs}
	\end{center}
	
\hrulefill

\subsection{ADSL}

La DSL (\textit{Digital Subscriber Line}) es una familia de tecnlologías que proporcionan el acceso a internet mediante la transmisión de datos digitales a través de los cables de una red telefónica local. El ADSL (\textit{Asymmetric Digital Subscriber Line}), la capacidad de descarga  y de subida no coinciden. Diseñado para que la capacidad de bajada sea mayor que la de subida.

\subsubsection{Arquitectura ADSL}

	\begin{center}
		\includegraphics[scale=0.2]{images/ADSL}
	\end{center}
	
	\begin{itemize}
		\item \textbf{ATU-R}: Módem digital que se instala en el domicilio del usuario y que está conectado a la línea telefónica y a la computadora, los datos son modulados mediante DMT.
		\item \textbf{Splitter}: Filtro que divide las frecuencias bajas en una banda de 0-4kHz y las altas frecuencias en una banda mayor a 4kHz. Al filtrar las bajas frecuencias se separa la voz y eliminan las interferencias provocadas sobre la transmisión de datos del usuario; de la misma forma al filtrar las altas frecuencias se separan los datos y permite que la transmisión de datos no afecte a la de voz.
		\item \textbf{ATU-C}: Módem digital ubicado en el lado de la central, debe de utilizar el mismo tipo de modulación que el ATU-R. Generalmente es una tarjeta que se insertará en el equipo que concentra y mulitplexa el tráfico de datos o DSLAM (\textit{Digital Subscriber Line Access Multiplexer}).
		\item \textbf{Multiplexor ADSL (DSLAM)}: Chasis que agrupa gran número de tarjetas, cada una consta de varios módems ATU-C y además concentra (multiplexa/demultiplexa) el tráfico de todos los enlaces ADSL hacia una red WAN. Realiza funciones de nivel de enlace (protocolo ATM sobre ADSL) entre el módem de usuario y el de central.
	\end{itemize}
	
\subsubsection{Problemas y Soluciones}

Uno de los mayores problemas es la atenuación de los cables, la diafonía, interferencias y ruido. Los errores se detectan y corrigen mediante FEC (\textit{Forward Error Coding}). Los bits se entrelazan para reducir los errores no corregibles (más tramas con errores, menos errores por trama). Los módems ADSL permiten definir circuitos virtuales especiales para aplicaciones sensibles al retardo. Los datos se transportan en tramas y supertramas (conjunto de 68 símbolos DMT), con 2 tipos de flujos: rápido y lento.

\subsubsection{ADS Lite - ADSL Spliterless}

Condiciones para la implantación óptima: Que no sea necesaria la instalación de un splitter por parte del proveedor de servicios, que el usuario pueda instalar el módem fácilmente, que no sea necesario re-cablear la línea, que exista compatibilidad espectral con otros servicios, velocidades adecuadas para el segmento del mercado al que se va a dar servicios, distancias en el bucle de abonado adecuadas para las velocidades a las que se va a dar servicio.\\

A partir de 1997 con un 10\% de las capacidades de ADSL es suficiente para dar internet a alta velocidad. Una reducción en la velocidad de servicio de ADSL se traduciría en una reducción de complejidad del hardware. Se suprime el splitter y permite modulación 256-QAM como máximo.

Para velocidades mayores se utiliza el VDSL que tiene un rango de frecuencias en el cable más amplio (30 MHz). HDSL pretende sustituir los sistemas T1 y E1. Ofrece un canal simétrico de 2Mbps y alcanza como máximo unos 4km. Se usa actualmente para líneas punto a punto de 2Mbps. tiene mayor alcance sin repetidores y al usar menor rango de frecuencias hay menos interferencias.

\hrulefill

\subsection{Sistemas MIC (PCM)}

MIC (\textit{Modulación poro Impulsos Codificados}). Diseñado originalmente para la transmisión digital de señales de voz con el objetivo de aprovechar las ventajas de los sistemas digitales. Estudio de: muestreo, cuantificación, codificación de canal, codificación de línea, multiplexación por división en el tiempo, señalización y control.

	\begin{center}
		\includegraphics[scale=0.2]{images/SistemasMIC}
	\end{center}	
	
\subsubsection{Muestreo}

La frecuencia de muestro es la máxima frecuencia utilizable de la señal analógica (voz). En telefonía, el ancho de banda de voz es 300-3400Hz. En función del teorema de Nyquist se muestrea con $F_{S} = 2 \cdot F_{max}$ pero por cuestiones prácticas la frecuencia seleccionada es de 8kHz. El valor de la frecuencia de muestreo fija la longitud de la trama en la multiplexación TDM, $125\mu s$.

\subsubsection{Cuantificación en sistemas MIC}

El proceso de cuantificación en sistemas MIC consiste en dividir el rango de valores de entrada en un número finito de intervalos y asignar un único valor de salida a cada intervalo. Hay que tener en cuenta los factores de:

	\begin{itemize}
		\item Valores analógicos máximos y mínimos de entrada: rango dinámico de entrada del cuantificador.
		\item Número de intervalos de cuantificación: Depende del número de bits de cuantificación e influye en el error de cuantificación.
		\item Tipo de cuantificador: Lineal o logarítmico.
	\end{itemize}

\subsubsection{Niveles de entrada al cuantificador}

Los valores extremos se conocen como valores virtuales de decisión. El nivel de máxima amplitud de una senoide sin recorte de crestas (nivel de sobrecarga). La UIT-T recomienda como nivel de sobrecarga como la amplitud de una señal de 3.14 dBm0 en el codificador situado tras la bobina híbrida (2h/4h). Las señales de entrada se normalizan a ese valor:  $P(dBm0) = 3.14 + 20 \log (V)$.

\subsubsection{Número de bits de cuantificación}

El parámetro de partida para fijar el número de bits es la relación señal a ruido de cuantificación $>60-70dB$, un valor habitual para los sistemas telefónicos. El objetivo es no cuantificar valores de señal menores que el ruido, en el caso de utilizar un cuantificador uniforme utilizar como número de bits 12. El cuantificador uniforme es óptimo para señales equiprobables en el ranngo de niveles de entrada. En la señal de voz los valores instantáneos de tensión siguen una distribución gamma: los valores bajos son muy probables y los altos poco probables. La potencia de ruido de cuantificación dependerá más de la amplitud del ruido de cuantificación en los niveles bajos. Será más eficiente utilizar un cuantificador adaptado al comportamiento estadístico de los valores de tensión de la señal vocal: Los intervalos pequeños cerca del origen tienen amplitud del error de cuantificación pequeña y estadísticamente contribuyen mucho a la potencia del error y los intervalos de cuantificación amplios tienen amplitud grande del error de cuantificación pero estadísticamente contribuyen poco a la potencia de error de cuantificación. Un cuantificador que cumple estas condiciones es el cuantificador logarítmico, con 8 bits se consigue una $(\frac{S}{N})_{Q}$ igual a la que se consigue con un cuantificador uniforme de 12 bits (sólo para señales de voz). La gananacia obtenida se conoce como ganancia de compansión y es de 24 dB: $G_{compansion} = 6 dB \cdot 4 bits = 24 dB$.

En realidad, el cuantificador logarítmico se construye con un cuantificador uniforme, un compansor y un expansor.

La UIT-T recomienda el uso de dos funciones logarítmicas para la cuantificación MIC:

	\begin{itemize}
		\item \textbf{Ley A}: Europa, es una función continua y se aproxima linealmente por tramo dando lugar a la ley A de 13 segmentos, $A = 87,6$. En el interior de cada segmento se hace una cuantificación uniforme con 4 bits: 16 subintervalos por segmento. La palabra MIC queda definida por 8 bits: 1 bit de signo, 3 bits por segmento y 4 bits por subintervalo dentro de cada segmento.
		\item \textbf{Ley $\mu$}: Estados Unidos, Japón y Canadá.
	\end{itemize}

\subsubsection{Sistemas MIC 30 + 2}

Hasta ahora se ha visto cómo digitalizar la señal vocal, para optimizar el uso de los recursos (enlaces) se multiplexan las señales vocales. En el caso de los sistemas MIC, esta multiplexación se realiza por división en el tiempo. De esta forma, los canales vocales correspondientes a 30 conversaciones se mltiplexan en el tiempo junto con información de alineación de trama y bits de servicio y señalización formando una trama MIC 30+2. Esta trama está compuesta por 32 intervalos temporales, cada uno de ellos correspondiente a una palabra MIC de una conversación, la duración de cada trama es de $125 \mu s$ (fijado por la frecuencia de muestreo $8kHz$. La velocidad de salida es de $32 intervalos \cdot 8 bits / intervalo \cdot 8 kHz = 2048 kbps$. Los intervalos se identifican como $I0 - I31$ y la estructura de la trama es:
	
	\begin{center}
		\includegraphics[scale=0.2]{images/TramaMIC}
	\end{center}

Hasta ahora (a falta de determinar la señales eléctricas que circulan por el medio físico, codificación de línea), un enlace a través de sistemas MIC 30+2:

	\begin{center}
		\includegraphics[scale=0.2]{images/EnlaceMIC}
	\end{center}
	
\subsubsection{Codificación de Línea}

Consiste en traducir los $0$ y $1$ a niveles de tensión en la línea de transmisión. LAs señales que se utilizan para ello han de cumplir unas condiciones:

	\begin{itemize}
		\item Ha de permitir la transmisión de señales en canales con bloqueo de continua. El código no ha de tener componente contínua, la densidad espectral de potencia debería tener un nulo en el origen y hay problemas con secuencias de ceros o unos consecutivos.
		\item Ha de contener información de temporización para extraer en el receptor la señal de reloj. Suficiente número de transiciones y hay problemas con secuencias de ceros o unos consecutivos.
		\item En algunos casos al introducir redundancia permitirán detectar errores.
	\end{itemize}
	
	\begin{itemize}
		\item \textbf{Código AMI} (\textit{Aternate Mark Inversion}): El cero se codifica con 0 voltios, el uno se codifica con $V^{+}$ o $V^{-}$ de forma alternada. Las secuencias largas de ceros dan problema con la información de reloj.
		\item \textbf{Código HDB3} (\textit{High Density Bipolar}): Soluciona el problema de secuencias largas de ceros de AMI. No están permitidas secuencias de más de tres ceros seguidos. Si hay 4 ceros consecutivos el 4º se codifica como un 1 pero el nivel alto o bajo opuesto al que corresponda. Si entre el bit $V$ que se va a insertar y el bit $V$ precedente hay un número par de 1s, el primer 0 también se codifica como si fuera un 1 con el nivel alto o bajo que le corresponda.
		\item \textbf{Código HDB4}: Igual que HDB3 pero con 4 ceros consecutivos
		\item \textbf{Códigos de bloque}: Una secuencia de unos y ceros se sustituye por una secuencia de impulsos que puede ser binaria, ternaria o cuaternaria. 3B4B, 3 símbolos binarios se sustituyen por 4 símbolos binarios.
	\end{itemize}
	
\subsubsection{Eficiencia y Redundancia de Código}


Los códigos de línea mejoran las propiedades de la señal transmitida (frente al nivel de continua y atenuación en baja frecuencia y frente al transporte de la información de la señal de reloj). Estas ventajas se consiguen a costa de aumentar el bitrate del canald e transmisión que será por lo general mayor que el bitrate de fuente.

Redundancia:

	\begin{equation*}
		R = \frac{r_{canal} - r_{fuente}}{r_{canal}}
	\end{equation*}
	
Eficiencia de código:

	\begin{equation*}
		\eta = 1 - R = \frac{r_{fuente}}{r_{canal}}
	\end{equation*}
	
\subsubsection{Regeneradores}

Se reparten a lo largo de la línea de transmisión para tratar cualquier perturbación. Están dispuestos a intervalos regulares, operan a nivel de bit.

\subsubsection{Transmisión a 64 Kbps}

Transmisión a 64 Kbps \textbf{codireccional} se utiliza un par por cada sentido de la transmisión con código 1B4T. Transmisión a 64 Kbps \textbf{contradireccional}, las señales de temporización se envían desde un mismo extremo en los dos sentidos (TX/RX), dos pares (uno para los datos y otro para la temporización por dirección, código AMI (100\% datos / 50\% reloj.

\subsubsection{ADPCM}

\textbf{DPCM}: La base es codificar la diferencia entre muestras consecutivas. Las ventajas de este codificador estriban en la gran correlación entre muestras consecutivas de la señal vocal. En los decodificadores DPCM la diferencia se codifica con 4 bits. La calidad es inferior a PCM. 

El codificador ADPCM funciona a varias velocidades (16-32 kbps). Completa el codificador DPCM incluyendo un bloque de predicción de la muestra actual. La predicción se compara con la muestra actual $\rightarrow$ error de predicción. El error de predicción se cuantifica y envía. Normalmente, la entrada al codificador es una señal MIC (codificada con Ley A o Ley $\mu$ por lo que previamente hay que convertir la codificación logarítmica en codificación uniforme).

\textbf{ADPCM G.721}
	
	\begin{center}
		\includegraphics[scale=0.2]{images/ADPCM1}
	\end{center}
	
	\begin{center}
		\includegraphics[scale=0.2]{images/ADPCM2}
	\end{center}
	
\hrulefill

\subsection{Redes de Fibra}

El acceso a través de fibra óptica es una alternativa más. Existen varias aproximaciones que se basan en criterios económicos y de necesidad de ancho de banda. FTT, Fiber to the X (cabinet, curb, building, home...). En su mayoría es una red óptica pasiva. 

Analogía: Acceso con cable de pares
	
	\begin{center}
		\includegraphics[scale=0.2]{images/Analogia}
	\end{center}

Escenarios posibles de FTTx:

	\begin{itemize}
		\item \textbf{Sistemas basados en la Jerarquía Digital Síncrona (SDH)}: Anillos de acceso (DLCs o unidades remotas de usuario provistas con equipos de ADM), anillos de acceso NGDLC (New Generation Digital Loop Carriers) con gran número de usuarios atendidos, necesitan poco control por la central, ofrecen gran cantidad de servicios e integran DLCs y DSLAM (separación de los flujos de datos). El tramo final sigue siendo en la mayoría cable de pares.
		\item \textbf{Sistemas basados en la Jerarquía Digital Plesiócrona}
		\item \textbf{SDH hasta los negocios}
	\end{itemize}

PDH sobre fibra hasta el negocio

	\begin{center}
		\includegraphics[scale=0.2]{images/PDHSobreFibra}
	\end{center}

SDH hasta el negocio:

	\begin{center}
		\includegraphics[scale=0.2]{images/SDHNegocio}
	\end{center}
	
\subsubsection{ATM Passive Optical Networks (APON)}

Sistemas basados en ATM (\textit{Asynchronous Transfer Mode}), la distancia máxima depende del número de salidas de distribución (limitado por atenuación), el máximo número de salidas del splitter es 64. Habitualmente en el sentido descendente (difusión), para el sentido ascendente TDM con sincronismo complejo (sistema de gestión específico) pero tiene problemas para ecualizar los tiempos de cada usuario.
	
	\begin{center}
		\includegraphics[scale=0.2]{images/APON}
	\end{center}

\hrulefill

\subsection{HFC}


Las \textbf{redes de cable} nacen ante la necesidad de resolver la mala calidad de las señales recibidas en poblaciones alejadas de los grandes núcleos urbanos. La arquitectura típica era árbol-rama. A partir de los años 80 a parecen las redes HFC (\textit{Hybrid Iber Coax}) que es una combinación de fibra óptica y cable coaxial. Surge un cambio en la estructura de la red, ahora es una combinación de estrella con árbol-rama. En un primer tramo, las fibras ópticas llevan la inforamción desd ela cabecera hasta unos nodos finales donde se realiza la conversión óptico/eléctrico con una topología de estrella y a partir de los cuales se despliega la típica red de coaxial con estructura de árbol-rama hasta los usuarios.

La red de fibra tiene una topología en estrella, los anillos son sólo un método para introducir redundancia (una fibra de subida y otra de bajada por cada lado del anillo) y conseguir una red más robusta ante averías. Entre cada nodo primario y nodo final hay 4 fibras dedicadas (2 de subida y 2 de bajada).

La topología HFC mejora la calidad gracias a que con la fibra disminuyen las interferencias y el número de amplificadores. Lo último también mejora la fiabilidad. Necesita menor alimentación, proporcionando mayor sencillez y menor coste. Aumenta la capacidad por acercar las fibras dedicadas a los usuarios. Permite dirigir algunos servicios por la parte en la estrella de la topología frente a la difusión de las redes antiguas. La tendencia es acercar más la fibra al usuario: FTTF $\rightarrow$ FTTC $\rightarrow$ FTTB $\rightarrow$ FTTH.

La \textbf{red HFC de Euskaltel} responde a la estructura de una red FTTC, es bidireccional y comparte la misma infraestructura que la red de transporte de Euskaltel destinada a los servicios de telefonía y datos. Se compone de: cabecera, red de contribución, red troncal, red de distribución y red de acometida de los abonados.

	\begin{center}
		\includegraphics[scale=0.2]{images/Euskaltel}
	\end{center}
	
La \textbf{red troncal}, para que la red no tenga degradaciones importantes la red ha de ser de fibra óptica. Realiza el transporte de las señales desde las cabeceras territoriales hasta los nodos finales a través de los tres niveles de red, habiendo rutas, equipamiento y fibras reduntantes. La red troncal primaria está constituida por anillos de fibra óptica que enlaza cada cabecera territorial con los nodos primarios de su provincia. La red troncal secundaria está constituida por anillos de fibra óptica que unen los nodos primarios con los secundarios. La red troncal terciaria está formada por anillos de fibra óptica que unen los nodos ópticos entre si y con los nodos secundarios.

Elementos de la red troncal:

	\begin{center}
		\includegraphics[scale=0.2]{images/RedTroncal}
	\end{center}
	
Elementos de la red de distribución:

	\begin{center}
		\includegraphics[scale=0.2]{images/RedDistribucion}
	\end{center}

La red de abonado une el tap situado en la entrada de los edificios con la toma final en la vivienda del usuario.

	\begin{center}
		\includegraphics[scale=0.2]{images/Esquema}
	\end{center}

\hrulefill

\subsection{Acceso Via Radio}

Soluciona los problemas de realizar obras para instalar una red de bucle de abonado. Es de bajo coste, rápido despliegue, accesible en zonas remotas, no necesita gran inversión inicial, el crecimiento es adaptado a la demanda, bajos costes de mantenimiento, retorno de inversión rápido...

\subsubsection{Características de los sistemas PMP}

Tiene esctructura celular, optimiza el espectro y necesita baja potencia. Establece la conexión entre subscriptores y central a través de radioenlaces digitales bidireccionale (necesita visibilidad radioeléctrica entre el cliente y la estación base de red).

\subsubsection{Estructura de la Red}

	\begin{center}
		\includegraphics[scale=0.2]{images/EstructuraRed}
	\end{center}
	
Algunos sistemas tienen estructura celular, éstas dependen de la orografía y de la densidad de clientes. Se reutilizan frecuencias dentro de una celda a otra. La utilización de polarizaciones diferentes permite optimizar más aún el espectro asignado.

	\begin{center}
		\includegraphics[scale=0.2]{images/Espectro}
	\end{center}

\subsubsection{Eficiencia}

Estos sistemas utilizan asignación dinámica de anchos de banca. Consiste en gestionar el espectro dinámicamente para asignar recursos a los usuarios. Permite comportir parte del espectro entre varios clientes según sus necesidades. Generalmente replican sistemas de conmutación ATM o Frame Relay.

	\begin{center}
		\includegraphics[scale=0.2]{images/Eficiencia}
	\end{center}
	
\subsubsection{Modulaciones y Ancho de Banda}

La capacidad de estos sistemas depende del ancho de banda disponible por el operador y la eficiencia espectral de la modulación empleada. Factor de utilización de frecuencias (doble uso de la banda asignada dentro de una celda con polarizaciones cruzadas), factor de compartición del espectro entre usuarios.

\subsubsection{MMDS}

Multichannel Multipoint Distribution System fue pensado originalmente como un servicio de difusión. Opera en la banda de los 3.5 GHz, las versiones posteriores son digitales con posibilidad de retorno, necesita visibilidad entre emisor-receptor, es sensible al multitrayecto, no se ve atenuado por la lluvia.

\subsubsection{LMDS}

Local Multipoint Distribution system utiliza la banda de frecuencias entre 24-40 GHz, emplea la modulación QPSK o 16 QAM. Sistema celular basado en células pequeñas, emplea transmisores de baja potencia, afectado por la atenuación atmosférica y en ocasiones no es necesario visión directa debido a las reflexiones.


%%%%%%%%%%%%%%%%%%%%%%%%%%%%%%%%%%%%%%%%%%%%%%%%%%%%%%%%%%%%%%%%%
% Tema 5

\hrulefill

\section{5. Redes Troncales y de Transporte}

\hrulefill

\subsection{PDH}

\subsubsection{Introducción}

La tecnología PDH (\textit{Plesiochronous Digital Hierarchy}) permite multiplexar en tramas de orden superior afluentes MIC 30+2. Optimiza los recursos en los enlaces transmitiendo el mayor número de canales vocales posibles con el mínimo de recursos. Al ser sistemas digitales, la multiplexación es TDM. PDH consiste en una red en la que los relojes con ``casi'' síncronos, tendrán la misma velocidad pero estarán sujetos a una tolerancia. Existen varios estándares. Es coherente con la diferencia en el proceso de creación de la trama básica MIC.\\

Las jerarquías digitales son la herramienta de transporte de las redes telefónicas digitales.

	\begin{center}
		\includegraphics[scale=0.2]{images/Jerarquia}
	\end{center}
	
La UIT especifica los niveles de multiplexación posible y la estructura de las tramas multiplexadas.

	\begin{center}
		\includegraphics[scale=0.2]{images/NivelesMUX}
	\end{center}

Las señales de entrada a un multiplexor son \textbf{tributarias} o \textbf{afluentes}. La multiplexación se realiza a nivel del bit.

\subsubsection{Velocidades PDH}

	\begin{center}
		\includegraphics[scale=0.2]{images/VelocidadPDH}
	\end{center}

	\begin{center}
		\includegraphics[scale=0.2]{images/CaracteristicasPDH}
	\end{center}	

\subsubsection{Justificación o Relleno}

Los equipos de la red PDH no están sincronizados, utilizan distintos relojes con la misma velocidad pero con tolerancias sobre la velocidad nominal. Esto supone un problema para la multiplexación.

	\begin{center}
		\includegraphics[scale=0.2]{images/JustificacionPDH}
	\end{center}	

Las tramas contemplan espacios para bits de relleno que se pueden utilizar como bits de información si es necesario $\rightarrow$ Justificación positiva
	
	\begin{center}
		\includegraphics[scale=0.2]{images/Relleno}
	\end{center}	

La técnica de justificación tiene tres variantes: positiva, negativa (supresión de impulsos sobre los afluentes), positiva-nula-negativa (usada en la SDH). Si se ha utilizado justificación en la trama es necesario indicarlo, hay bits de control de justificación sobre qué afluente se ha realizado.

\subsubsection{Estructura de la trama G.742 - 8 Mbps}
	
	\begin{center}
		\includegraphics[scale=0.2]{images/TramaG742}
	\end{center}	

\subsubsection{Estructura de las tramas G.751 - 34 Mbps}

	\begin{center}
		\includegraphics[scale=0.2]{images/TramaG751}
	\end{center}
	
\subsubsection{Estructura de las tramas G.751 - 140 Mbps}

	\begin{center}
		\includegraphics[scale=0.2]{images/TramaG7512}
	\end{center}		

\subsubsection{Velocidades de trama de señales PDH}

\textbf{Tasa Nominal de Justificación}: $\theta = \frac{F_{J}}{F_{A}}$\\
	\quad $F_{A}$: Velocidad de justificación cuando el afluente llega a velocidad nominal\\
	\quad $F_{J}$: Frecuencia Nominal de Justificación

\textbf{Frecuencia de Redundancia}\\
	\quad Frecuencia a la que se insertan los bits de control \\
	\quad Tasa de redundancia $R$
	
\textbf{Relación Nominal de Justificación}: Relación entre la velocidad nominal de justificación  y la velocidad máxima de justificación posible. $g = \frac{F_{J}}{F_{J}^{MAX}}$

\textbf{Frecuencia del Múltiplex}: $F_{M} = N \cdot F_{A} * (1 + \theta) (1 + R)$


\subsubsection{Tipos de Equipos en redes PDH}

\textbf{Terminales de línea}

\textbf{Multiplexores}

	\begin{center}
		\includegraphics[scale=0.2]{images/Electrico}
	\end{center}	

\textbf{Radio}

	\begin{center}
		\includegraphics[scale=0.2]{images/Radio}
	\end{center}

\textbf{Optico}

	\begin{center}
		\includegraphics[scale=0.2]{images/Optico}
	\end{center}

\textbf{Multiplexores de acceso: canales de usuario / 2M}

	\begin{center}
		\includegraphics[scale=0.2]{images/Mux}
	\end{center}

	\begin{center}
		\includegraphics[scale=0.2]{images/MuxTerminal}
	\end{center}

\textbf{Multiplexores de Inserción y Extracción}

	\begin{center}
		\includegraphics[scale=0.2]{images/MuxExtra}
	\end{center}

Tipos de conexiones:

	\begin{itemize}
	\item Punto a punto
		\begin{center}
			\includegraphics[scale=0.2]{images/P2P}
		\end{center}
	\item Punto a Multipunto
		\begin{center}
			\includegraphics[scale=0.2]{images/P2M}
		\end{center}
	\item Punto a Multipunto
		\begin{center}
			\includegraphics[scale=0.2]{images/Uni}
		\end{center}
	\item Protección de circuitos: sistemas propietarios
	\end{itemize}

\textbf{Cross connect}

	\begin{center}
			\includegraphics[scale=0.2]{images/Cross}
		\end{center}

\hrulefill

\subsection{SDH}

\subsubsection{Introducción}

El antecedente a las redes SDH (Sinchronous Digital Hierarchy) son los sistemas PDH. Multiplexación de sistemas MIC 30+2, creación de múltiplex de orden superior mediante multiplexación ``casi'' síncrona, los equipos utilizan relojes no sincronizados $\rightarrow$ para poder realizar correctamente la multiplexación se requiere de técnicas de justificación. Tienen una serie de \textbf{problemas}: \textbf{No existe compatibilidad en los estándares utilizados en diferentes países} y los equipos de adaptación son costosos, \textbf{las capacidades máximas son limitadas}, \textbf{problemas para extraer flujos} ya que para extraer un flujo de nivel inferior directamente de un flujo de nivel superior hay que demultiplexar todos los flujos, \textbf{no está pensada para el uso de fibra óptica como medio de transmisión}, \textbf{carece de herramientas de gestión y no prevé sistemas de tolerancia a fallos en los enlaces} limitando el ancho de banda disponible para la señalización y la gestión de la red siendo los sistemas de gestión y mantenimiento propietarios.

Extracción de una tributaria de 2 Mbps en un múltiplex de 139 Mbps

	\begin{center}
		\includegraphics[scale=0.2]{images/Extraccion}
	\end{center}

Estos problemas los resuelve SDH. Da lugar a un estándar ANSI que se denominó SONET.

	\begin{center}
		\includegraphics[scale=0.2]{images/AddDrop}
	\end{center}


\subsubsection{Velocidades SDH-SONET}

Aquí irían las tablas

	\begin{center}
		\includegraphics[scale=0.2]{images/EjemploTrans}
	\end{center}

\begin{itemize}
	\item \textbf{Regeneradores}: Regeneradores de señal utilizados en enlaces de larga distancia, donde las perturbaciones del medio de transmisión limitan el alcance del enlace.
	\item \textbf{Multiplexores terminales}: Permiten combinar señales de entrada PDH o SDH en señales STM-N de mayor velocidad.
	\item \textbf{Multiplexores ADM} (Add and Drop Multiplexer): Permiten intercalar o extraer tramas de un determinado múltiplex. Permiten la formación de anillos.
	\item \textbf{Digital Cross-Connect}: Permiten la interconexión de diferentes secciones o anillos de la red SDH, tienen varias entradas y varias salidas.
\end{itemize}

\subsubsection{Topologías de Red}

\begin{itemize}
	\item Punto a Punto
		\begin{center}
			\includegraphics[scale=0.2]{images/SDHP2P}
		\end{center}
	\item Lineal / Punto a Multipunto
		\begin{center}
			\includegraphics[scale=0.2]{images/SDHMulti}
		\end{center}
	\item Anillo
		\begin{center}
			\includegraphics[scale=0.2]{images/SDHAnillo}
		\end{center}
	\item Anillo: Comunicación full duplex con una fibra
		\begin{center}
			\includegraphics[scale=0.2]{images/SDHAnillo2}
		\end{center}
\end{itemize}

\subsubsection{Arquitectura SDH}

\subsubsection{Estructura STM-1/STM-N/STM-Nc}

\subsubsection{Tara de Sección}

\subsubsection{Contenedores y Punteros}

\subsubsection{Multiplexación y Mapeado de señales PDH sobre SDH}

\subsubsection{Gestión de red SDH}

\subsubsection{Protección en SDH}

\subsubsection{Sincronismo}

\subsubsection{Arquitectura de un operador}


%%%%%%%%%%%%%%%%%%%%%
TEMA 5

\subsubsection{Tipos de Señalización}

\textbf{Señalización de abonado}: Abonado-Central y Central-Abonado. Tiene que ser sencilla para permitir la retrocompatibilidad y permitir terminales sencillos y baratos.

\textbf{Señalización interna de central}: Es dependiente del fabricante. Envía información de señales y control entre los equipos de la central. La señalización en la red fue en sus comienzos una esxtensión de este tipo de señalización.

\textbf{Señalización entre centrales}: Elige los enlaces que se van a utilizar para la llamada y se encarga de liberarlos una vez se termine. Antes, consulta el \textbf{estado de enlace} para comprobar si está disponible. Hay que enviar las llamadas desde un extremo de la red hasta otro mediante el \textbf{rutado de llamadas}. \textbf{Envío de numeración entre centrales}. La información se envía \textbf{enlace a enlace}. \textbf{Extremo a extremo}, un nodo maestro dirige la señalización, conoce la red y va a organizar la comunicación, permite enviar menos información por la red.

\textbf{Señalización por canal asociado}: Cada canal vocal tiene asociado un canal de comunicación propio para la señalización. Típica de los sistemas analógicos. Este canal de señalización está asignado durante la comunicación. Normalmente, un canal físico de señalización está asociado al canal vocal. El canal físico (medio de transmisión) se comparte por los usuarios pero aún así sigue siendo señalización por canal asociado. Conocido como CAS. El equipo de transmisión está activo or una mínima parte del tiempo total de utilización del canal. El equipo receptor está funcionando permanentemente pero su actividad es mínima. No es posible la señalización extremo a extremo sin establecer un circuito. Por la eficiencia, no es posible consultar bases de datos porque solo hay un canal de señalización activo con un canal de voz activo.

	\begin{center}
		\includegraphics[scale = 0.2]{SenalizacionAsociado}
	\end{center}

\textbf{Señalización por canal común}: Un único canal de señalización que es compartido por todos los usuarios. La señalización de cada usuario (canal vocal) se envía cuando se necesita. En algunos casos el canal físico de señalización y el de tráfico están separados pero no tiene por qué ser así. Conocido como CCS. Al suprimir los terminales de señalización se ahorra en los enlaces. Aumenta el vocabulario de señalización implementando más servicios. Incremento en velocidad en el establecimiento de llamadas. Aumento en la fiabilidad mediante el empleo de métodos mas eficac es de detección y corrección de errores. Permite el acceso a bases de datos. La señalización extremo a extremo es posible antes de establecer una comunicación.

	\begin{center}
		\includegraphics[scale = 0.2]{SenalizacionComun}
	\end{center}

Realizando una llamada a cobro revertido.

	\begin{center}
		\includegraphics[scale = 0.2]{CobroRevertido}
	\end{center}


\subsubsection{Evolución de los Sistemas de Señalización}

\textbf{CAS} previo a 1970 con señalización en banda. \textbf{CCIS} (\textit{Common Channel Interoffice Signaling}), se utiliza SS6, un sistema de canal común. \textbf{SS7}, origen en los años 1980, la señalización se transporta sobre redes separadas del tráfico de voz y tiene una pila de protocolos.

\subsubsection{SS7}

Estándar de la UIT-T, diseñado específicamente para atender requerimientos avanzados de redes telefónicas digitales (RDSI). Basado en una red de datos independiente de la red de tráfico. Teniendo nodos, enlaces y protocolos propios. Su arquitectura es independiente de la(s) red(es) que soporta. 

	\begin{center}
		\includegraphics[scale = 0.2]{RedTrafico}
	\end{center}

\textbf{Arquitectura de la Red SS7} está compuesta por nodos y enlaces. Los nodos se conocen como Signaling Points y tiene un código (SPC) que lo identifica.

	\begin{center}
		\includegraphics[scale = 0.2]{SS7}
	\end{center}
	
	\begin{center}
		\includegraphics[scale = 0.2]{SP}
	\end{center}
	
	\begin{itemize}
		\item SSP (Service Switching Point): Genera y recibe mensajes de señalización.
		\item STP (Service Transfer Point): Conmutador de paquetes de señalización. No genera ni termina mensajes y físicamente puede ser el mismo equipo que un SSP o separado.
		\item SCP (Service Control Point): Nodos para aplicaciones.
	\end{itemize}

Los SCPs y STPs suelen estar redundados por motivos de seguridad y disponibilidad. La red de SSPs (enlaces de tráfico) puede tener cualquier topografía.

	\begin{center}
		\includegraphics[scale = 0.2]{RedSSP}
	\end{center}

	\begin{center}
		\includegraphics[scale = 0.2]{PlanoSS7}
	\end{center}

Una central puede tener varios SCPs (en varios planos)

	\begin{center}
		\includegraphics[scale = 0.2]{NivelesSS7}
	\end{center}

	




%\vfill

%\end{multicols}

\end{document}