\documentclass[10pt,portrait, twocolumn]{article}
\usepackage{multicol}
\usepackage{calc}
\usepackage[portrait]{geometry}
\usepackage{amsmath,amsthm,amsfonts,amssymb}
\usepackage{times}
\usepackage{color,graphicx,overpic}
\graphicspath{ {images/} }
\usepackage{hyperref}
\usepackage{pgfplots}
\usepackage{esint}
\usepackage{bm}
\usepackage{tikz}
\usepackage{relsize}
\usepackage{datetime}
\usepackage[utf8] {inputenc}
\usepackage[spanish, activeacute] {babel}
\usepackage{IEEEtrantools}
\usepackage{framed}

\usepackage{pdflscape}

\usepackage{draftwatermark}
\SetWatermarkText{Javier de Martín}
\SetWatermarkScale{0.8}

% This sets page margins to .5 inch if using letter paper, and to 1cm
% if using A4 paper. (This probably isn't strictly necessary.)
% If using another size paper, use default 1cm margins.
\geometry{top=.5cm,left=.5cm,right=.5cm,bottom=.5cm}
    
\pgfplotsset{
    dirac/.style={
        mark=triangle*,
        mark options={scale=2},
        ycomb,
        scatter,
        visualization depends on={y/abs(y)-1 \as \sign},
        scatter/@pre marker code/.code={\scope[rotate=90*\sign,yshift=-2pt]}
    }
}

% Turn off header and footer
\pagestyle{empty}

% Redefine section commands to use less space
\makeatletter
\renewcommand{\section}{\@startsection{section}{1}{0mm}%
                                {-1ex plus -.5ex minus -.2ex}%
                                {0.5ex plus .2ex}%x
                                {\normalfont\large\bfseries}}
\renewcommand{\subsection}{\@startsection{subsection}{2}{0mm}%
                                {-1explus -.5ex minus -.2ex}%
                                {0.5ex plus .2ex}%
                                {\normalfont\normalsize\bfseries}}
\renewcommand{\subsubsection}{\@startsection{subsubsection}{3}{0mm}%
                                {-1ex plus -.5ex minus -.2ex}%
                                {1ex plus .2ex}%
                                {\normalfont\small\bfseries}}
\makeatother

\newcommand{\Lagr}{\mathcal{L}}

% Define BibTeX command
\def\BibTeX{{\rm B\kern-.05em{\sc i\kern-.025em b}\kern-.08em
    T\kern-.1667em\lower.7ex\hbox{E}\kern-.125emX}}

% Don't print section numbers
\setcounter{secnumdepth}{0}


\setlength{\parindent}{0pt}
\setlength{\parskip}{0pt plus 0.5ex}

%My Environments
\newtheorem{example}[section]{Example}
% ---------------------------------------------------------------

\begin{document}

\begin{landscape}

\raggedright
\footnotesize
\begin{multicols}{3}


% multicol parameters
% These lengths are set only within the two main columns
%\setlength{\columnseprule}{0.25pt}
\setlength{\premulticols}{1pt}
\setlength{\postmulticols}{1pt}
\setlength{\multicolsep}{1pt}
\setlength{\columnsep}{2pt}

\begin{framed}
	\begin{center}
    	\Large{\underline{Sistemas de Telecomunicación}} \\
    	\scriptsize{3º Ingeniería de Telecomunicaciones | UPV/EHU}\\
     	%Actualizado por última vez el \today \\
     	"\textsl{Under-promise and over-deliver}." \\
     	%\hspace{5 pt} \\
     	\small{\textbf{Javier de Martín -- 2016}}
	\end{center}
\end{framed}

%
% Cheatsheet code below 
%                                                      

\section{\underline{Unidades Logarítmicas}}

%\begin{center}
%	\begin{tikzpicture}
%		\node [draw, align=center] (1) at (0,0) {Fuente};
%		\node [draw, align=center] (2) at (0,-1) {Procesado en TX};
%	
%		\node [draw, align=center] (3) at (2,-1) {Red};
%		
%		\node [draw, align=center] (4) at (4,-1) {Procesado en RX};
%		\node [draw, align=center] (5) at (4,0) {Presentación};
%	
%		\path (1) edge node [right] {} (2);
%		\path (2) edge node [above] {} (3);
%		\path (3) edge node [above] {} (4);
%		\path (4) edge node [left] {} (5);
%	\end{tikzpicture}	
%\end{center}

\textbf{dB} es una unidad que describe una \textbf{relación} entre magnitudes.

\begin{IEEEeqnarray*}{rCl}
	L(dB) & = & 10 \cdot \log_{10} \left( \frac{P_2}{P_1} \right) \\ & = & 20 \cdot \log_{10} \left( \frac{V_2}{V_1} \right) + 10 \cdot \log_{10} \left( \frac{R_1	}{R_2} \right)
\end{IEEEeqnarray*}

\subsection{Unidades Derivadas del dB}

\begin{itemize}
	\item $dBm$: Potencia de la señal en un punto cualquiera de un circuito referida a una potencia de $1mW$.
		
		\begin{equation*}
			L(dBm) = 10 \cdot \log_{10} \left( \frac{P(mW)}{1mW} \right)
		\end{equation*}
	\item $dBW$: Potencia de la señal referida a una potencia de $1W$.
		
		\begin{equation*}
			L(dBW) = 10 \cdot \log_{10} \left( \frac{P(W)}{1W} \right)
		\end{equation*}
		
	\item $dBmV$: Nivel de un voltaje comparado con $1mV$ sobre una carga de $75\Omega$.
		
		\begin{equation*}
			L(dbmV) = 20 \cdot \log_{10} \left( \frac{V(mV)}{1mV} \right)
		\end{equation*}
		
	\item $dBV$: Nivel de un voltaje comparado con $0,775 V$ (tensión eficaz) sobre una carga de $600\Omega$.
		
		\begin{equation*}
			L(dBm) = dBV + 10 \cdot \log_{10} \left( \frac{600}{R} \right)
		\end{equation*}
\end{itemize}

Estas medidas están relacionadas por:

	\begin{equation*}
		L(dBm) = L(dBV) + 10 \cdot \log_{10} \left( \frac{600}{R} \right)
	\end{equation*}
	
	\begin{IEEEeqnarray*}{rCl}
		L(dBV) & = & L(dBm) - 10 \cdot \log_{10} \left( \frac{600}{R} \right) \\
			    & = & 20 \cdot \log_{10} \left( \frac{V}{0,775 (V)} \right)
	\end{IEEEeqnarray*}
	

\subsection{Niveles}

\begin{itemize}
	\item $dBr$: Expresa el nivel relativo en un punto con respecto a otro punto; es una medida en $dB$ sin sufijo, $r$ se incluye para denotar que se trata de un valor relativo a un cierto punto de referencia.
	
		\begin{equation*}
			L(dBr) = 10 \log_{10} \frac{P}{P_{ref}}
		\end{equation*}
	
	\item $dBm0$: Indica la potencia en $dBm$ presente en el punto de nivel relativo cero.
		
		\begin{equation*}
			L(dBm0) = L_A(dBm) - L_A(dBr)
		\end{equation*}
\end{itemize}

Si se emite un tono de prueba ($0dBm$) $\rightarrow L_A (dBm) = L_A (dBr)$.

	\begin{IEEEeqnarray*}{rCl}
		dBm \pm dBm = dB \\
		dBm \pm dB = dBm
	\end{IEEEeqnarray*}	

\section{\underline{Perturbaciones y Medios de TX}}

\subsection{Distorsión Lineal}

%\begin{center}
%	\begin{tikzpicture}
%		\draw (0,0) -- node[above left] {$x(t)$} (0.6,0);
%		\node [draw, align=center] at (0.95,0) {$h(t)$};	
%		\draw (1.3,0) -- node[above right] {$y(t) = k \cdot x(t - t_0)$} (1.9,0);
%	\end{tikzpicture}	
%\end{center}

%Hay distorsión no lineal si en la banda de trabajo:

\begin{itemize}
	\item de amplitud: $k \neq cte$, $k = k(f)$
	\item de fase: $t_o \neq cte$, $t_0 = t_o(f)$
\end{itemize}

\subsection{Distorsión No Lineal o Armónica}

%No se puede utilizar la respuesta en frecuencia $H(f)$, se utilizará la \textbf{característica de transferencia}.

\begin{IEEEeqnarray*}{rCl}
	y(t) = f(x(t)) \underbrace{=}_\textrm{\tiny{Serie de Taylor}}  \overbrace{a_0}^\textrm{C.C} & + & \overbrace{a_1 x(t)}^\textrm{\tiny{Término Lineal}} + \overbrace{a_2 x^2(t)}^\textrm{\tiny{Término Cuadrático}} + \\ & + & \overbrace{a_3 x^3(t)}^\textrm{\tiny{Término Cúbico}} + ... + a_n x^n(t)
\end{IEEEeqnarray*}

%\begin{center}
%	\begin{tikzpicture}[scale = 0.5]
%	\begin{axis}[axis lines = middle,xmin=-0.1,xmax = 3.2,ymin=-0.3,ymax=3, xticklabels={,,}, yticklabels={,,}] %,grid=both]
%		\addplot +[dirac] coordinates {(1, 2)};
%		
%		% Valores del eje X
%		\node at (axis cs:1, 0) [anchor=north] {$f_0$};
%		\node at (axis cs:0, 3) [anchor=north west] {$X(f)$};
%		\node at (axis cs:3.2, 0) [anchor=south east] {$f$};
%		
%	\end{axis}
%\end{tikzpicture}
%\hspace{2pt}
%$\rightarrow$
%\begin{tikzpicture}[scale = 0.5]
%	\begin{axis}[axis lines = middle,xmin=-0.1,xmax = 3.2, ymin=-0.3,ymax=3, xticklabels={,,}, yticklabels={,,}] %,grid=both]
%		\addplot +[dirac] coordinates {(1, 2)};
%		%\node at (axis cs:1,2) [anchor=north east] {test};
%		\addplot +[orange, dirac] coordinates {(2, 1.6)};
%		\node at (axis cs:2, 1.6) [anchor=north east] {$V_{d_3}$};
%		\addplot +[orange, dirac] coordinates {(3, 1.3)};
%		\node at (axis cs:3, 1.3) [anchor=north east] {$V_{d_3}$};
%		
%		% Valores del eje X
%		\node at (axis cs:1, 0) [anchor=north] {$f_0$};
%		\node at (axis cs:2, 0) [anchor=north] {$ 2 \cdot f_0$};
%		\node at (axis cs:3, 0) [anchor=north] {$3 \cdot f_0$};
%		\node at (axis cs:0, 3) [anchor=north west] {$Y(f)$};
%		\node at (axis cs:3.2, 0) [anchor=south east] {$f$};
%		
%		% Nota: "Amplitud de distorsión
%		\draw [<-] (axis cs:2,1.7)-- +(20pt,15pt) node[above] {Amplitud de distorsión};
%		\draw [<-] (axis cs:3,1.4)-- +(-20pt,30pt) node[above] {}; %{Amplitud de distorsión};
%	\end{axis}
%\end{tikzpicture}
%\end{center}

El grado del polinomio a la salida del sistema no lineal indica cuántas frecuencias nuevas van a ser generadas por dicho sistema.

\begin{itemize}
	\item \textbf{Coeficiente de Distorsión del Armónico n-ésimo $d_n$}:
		\begin{IEEEeqnarray*}{rCl}
			d_n & = & \frac{V_{d_n}}{V_1} \hspace{5pt} n = 2, 3, ..., k \\
			D_n & = & 20 \cdot \log_{10} \frac{V_{d_n}}{V_1}	
		\end{IEEEeqnarray*}
	\item \textbf{Atenuación del Armónico n-ésimo $A_n$}:
		\begin{equation*}
			A_n = 20 \cdot \log_10 \frac{V_1}{V_{d_n}} = -D_n
		\end{equation*}
	\item \textbf{Coeficiente de Distorsión Total $d$}:
		\begin{equation*}
			d = \sqrt{\sum_{n>1} d_n^2}
		\end{equation*}
	\item \textbf{Total Harmonic Distortion(THD)}:
		\begin{equation*}
			THD(\%) = \frac{1}{V_1} \sqrt{\sum_{n>1} V_{d_n}^2} \cdot 100 \%
		\end{equation*}
\end{itemize}

Si $V_1$ aumenta $\Delta (dB) \rightarrow V_{d_n}$ aumenta $n \cdot \Delta (dB)$. 


\subsection{Intermodulación}

\begin{center}
	\begin{tikzpicture}
		\draw (0,0) -- node[above left] {$x(t)$} (0.6,0);
		\node [draw, align=center] at (0.95,0) {$h(t)$};	
		\draw (1.3,0) -- node[above right] {$y(t) = a_0 + a_1 \cdot x(t)+ ... + a_n \cdot x^n(t)$} (1.9,0);
	\end{tikzpicture}	
\end{center}

A la salida del sistema aparecen nuevas frecuencias:

\begin{itemize}
	\item Armónicos: $2 \cdot f_1$, $2 \cdot f_2$, $3 \cdot f_1$, $3 \cdot f_2$, ..., $n \cdot f_1$, $n \cdot f_2$
	\item Combinación lineal de las frecuencias de x(t):
		\begin{itemize}
			\item Segundo orden: $f_1 + f_2$, $f_1 - f_2$, ...
			\item Tercer orden: $2 f_1 + f_2$, $f_1 - 2 f_2$, ... 
		\end{itemize} 
\end{itemize}

\begin{center}
	\begin{tikzpicture}[scale = 0.5]
	\begin{axis}[axis lines = middle,xmin=-0.1,xmax = 3.2,ymin=-0.3,ymax=3, xticklabels={,,}, yticklabels={,,}] %,grid=both]
		\addplot +[dirac, blue] coordinates {(1, 2)};
			\node at (axis cs:1, 2) [anchor=north east] {$V_1$};
		\addplot +[dirac, blue] coordinates {(1.5, 2)};
			\node at (axis cs:1.5, 2) [anchor=north east] {$V_2$};
		
		% Valores del eje X
		\node at (axis cs:1, 0) [anchor=north] {$f_1$};
		\node at (axis cs:1.5, 0) [anchor=north] {$f_2$};
		
		\node at (axis cs:0, 3) [anchor=north west] {$x(t)$};
		\node at (axis cs:3.2, 0) [anchor=south east] {$f$};
		
	\end{axis}
\end{tikzpicture}
\hspace{2pt}
$\rightarrow$
\begin{tikzpicture}[scale = 0.5]
	\begin{axis}[axis lines = middle,xmin=-0.1,xmax = 5, ymin=-0.4,ymax=3, xticklabels={,,}, yticklabels={,,}] %,grid=both]
		\addplot +[dirac, blue] coordinates {(1.5, 2)};
			\node at (axis cs:1.5, 2) [anchor=north east] {$V_1$};
			\node at (axis cs:1.5, 0) [anchor=north] {$f_1$};
		\addplot +[dirac, blue] coordinates {(2.75, 2)};
			\node at (axis cs:2.75, 2) [anchor=north east] {$V_2$};
			\node at (axis cs:2.75, 0) [anchor=north] {$f_2$};
		
			% Armónicos de f1 y f2
		\addplot +[dirac, green] coordinates {(3.75, 1)};
			\node at (axis cs:3.75, 1) [anchor=north east] {$V_{d_2}$};
			\node at (axis cs:3.75, 0) [anchor=north] {\tiny $2f_1$};
		\addplot +[dirac, green] coordinates {(4.75, 1)};
			\node at (axis cs:4.75, 1) [anchor=north east] {$V_{d_2}$};
			\node at (axis cs:4.75, 0) [anchor=north] {\tiny $2f_2$};
			
%		\addplot +[dirac, red] coordinates {(3, 1)};
%			\node at (axis cs:3, 1) [anchor=north west] {$V_{d_3}$};
%			\node at (axis cs:3, -0.2) [anchor=north] {\tiny $2f_2$};
%		\addplot +[dirac, red] coordinates {(4.5, 1)};
%			\node at (axis cs:4.5, 1) [anchor=north west] {$V_{d_4}$};
%			\node at (axis cs:4.5, 0) [anchor=north] {\tiny $3f_2$};
%		
			% Productos de intermodulación de Orden 2	
		\addplot +[dirac, black] coordinates {(0.5, 1.5)}; % f2 - f1
			\node at (axis cs:0.5, 1.5) [anchor=north east] {$V_{i_2}$};
			\node at (axis cs:0.35, -0.1) [anchor=north, rotate = 0] {\tiny $f_2 - f_1$};
		\addplot +[dirac, black] coordinates {(4.25, 1.5)}; % f1 + f2
			\node at (axis cs:4.25, 1.5) [anchor=north west] {$V_{i_2}$};
			\node at (axis cs:4.25, -0.25) [anchor=south, rotate = 0] {\tiny $f_1 + f_2$};
			
			% Productos de intermodulacion de Orden 3
		\addplot +[dirac, red] coordinates {(2.25, 1.5)}; % 2f2 - f1
			\node at (axis cs:2.25, 1.5) [anchor=north east] {$V_{i_3}$};
			\node at (axis cs:2.15, -0.15) [anchor=north, rotate = 0] {\tiny $2f_2 - f_1$};
		\addplot +[dirac, red] coordinates {(3.25, 1.5)}; % 2f1 + f2
			\node at (axis cs:3.25, 1.5) [anchor=north east] {$V_{i_3}$};	
			\node at (axis cs:3.15, -0.15) [anchor=north, rotate = 30] {\tiny $2f_1 + f_2$};
%		
%		% Valores del eje X
%		\node at (axis cs:1, 0) [anchor=north] {\tiny $f_1$};
%		\node at (axis cs:1.5, 0) [anchor=north] {\tiny $f_2$};
%		
%		\node at (axis cs:2, 0) [anchor=north] {\tiny $ 2 \cdot f_0$};
%		\node at (axis cs:3, 0) [anchor=north] {\tiny $3 \cdot f_0$};
%		\node at (axis cs:0, 3) [anchor=north west] {$y(t)$};
%		\node at (axis cs:5, 0) [anchor=south east] {$f$};
	
	\end{axis}
\end{tikzpicture}
\end{center}

\begin{center}
	Intermodulación orden $n$ $>$ Armónico orden $n$
\end{center}

\begin{itemize}
	\item \textbf{Coeficiente de Intermodulación enésimo ($i_n$)}:
		\begin{IEEEeqnarray*}{rCl}
			i_n & = & \frac{V_{d_1}}{V_1} = n \cdot d_n	\\
			I_n & = & 20 \cdot \log_{10} \frac{V_{i_n}}{V_1} = 20 \cdot \log_{10} n \cdot \frac{V_{d_1}}{V_1} = D_n + 20 \cdot \log_{10} n
		\end{IEEEeqnarray*}
\end{itemize}

Si $x(t)$ cambia y ahora tiene $\Delta (dB)$ menos:

	\begin{IEEEeqnarray*}{rCl}
		D'_n & = & D_n + (n - 1) \cdot \Delta \\
		I_N & = & D_n + 20 \cdot \log_{10} n \\
		I'_n & = & D'_n + 20 \cdot \log_{10} n = D_n + (n - 1) \Delta + 20 \cdot \log_{10} n = \\
		     & = & I_n + (n-1)\cdot \Delta	
	\end{IEEEeqnarray*}


\subsection{Diafonía}

\begin{center}
	\begin{tikzpicture}
		\node [draw, align=center, fill={rgb:black,1;white,2}, text = white] at (0,0) {$F_1$};
		\node [draw, align=center, fill={rgb:black,1;white,2}, text = white] at (0,-1) {$F_2$};	
	
		\node [draw, align=center, fill={rgb:black,1;white,2}, text = white] at (2,0) {$P_1$};
		\node [draw, align=center, fill={rgb:black,1;white,2}, text = white] at (2,-1) {$P_2$};
	
		\draw (0.25,0) -- node[above left] {} (1.75,0);
		\draw (0.25,-1) -- node[above left] {} (1.75,-1);
		\draw [blue,  -to, thick] (0.5,0) -- (1.5,0) node [right] {};
		\draw [red,  -to, thick] (0.5,0) .. controls (1.10,0) and (0.7,-1) .. (1.5,-1);
		\draw [red,  -to, thick] (0.5,0) .. controls (1.05,0) and (0.88,-1) .. (0.5,-1);
	\end{tikzpicture}
\end{center}

El circuito \textbf{perturbador} es el circuito en el que se genera la perturbación y el circuito \textbf{perturbado} es en el que se recibe la diafonía.\\

\textbf{Clasificación} de la diafonía:

\begin{itemize}
	\item Según como sea percibida la señal perturbadora en el circuito perturbado:
		\begin{itemize}
			\item Inteligible
			\item Ininteligible
		\end{itemize}
	\item Según el número de circuitos que atraviesa la señal perturbadora:
		\begin{itemize}
			\item Directa: No se atraviesan circuitos intermedios
				
%				\begin{center}
%					\begin{tikzpicture}
%						\node [draw, align=center, fill={rgb:black,1;white,2}, text = white] at (0,0) {$F_1$};
%						\node [draw, align=center, fill={rgb:black,1;white,2}, text = white] at (0,-1) {$F_2$};	
%	
%						\node [draw, align=center, fill={rgb:black,1;white,2}, text = white] at (2,0) {$P_1$};
%						\node [draw, align=center, fill={rgb:black,1;white,2}, text = white] at (2,-1) {$P_2$};
%	
%						\draw (0.25,0) -- node[above left] {} (1.75,0);
%						\draw (0.25,-1) -- node[above left] {} (1.75,-1);
%						\draw [red,  -to, thick] (0.5,0) .. controls (1.10,0) and (0.7,-1) .. (1.5,-1);
%					\end{tikzpicture}
%				\end{center}
			
			\item Indirecta: Se atraviesan uno o más circuitos intermedios
			
				\begin{itemize}
					\item Transversal
					
%						\begin{center}
%							\begin{tikzpicture}[scale = 0.7]
%								\node [draw, align=center, fill={rgb:black,1;white,2}, text = white] at (0,0) {$F_1$};
%								\node [draw, align=center, fill={rgb:black,1;white,2}, text = white] at (0,-1) {$F_2$};
%								\node [draw, align=center, fill={rgb:black,1;white,2}, text = white] at (0,-2) {$F_3$};		
%	
%								\node [draw, align=center, fill={rgb:black,1;white,2}, text = white] at (2,0) {$P_1$};
%								\node [draw, align=center, fill={rgb:black,1;white,2}, text = white] at (2,-1) {$P_2$};
%								\node [draw, align=center, fill={rgb:black,1;white,2}, text = white] at (2,-2) {$P_3$};
%	
%								\draw (0.25,0) -- node[above left] {} (1.75,0);
%								\draw (0.25,-1) -- node[above left] {} (1.75,-1);
%								\draw (0.25,-2) -- node[above left] {} (1.75,-2);
%								\draw [red,  -to, thick] (0.5,0) .. controls (1.10,0) and (0.7,-2) .. (1.5,-2);
%							\end{tikzpicture}
%						\end{center}
					
					\item Longitudinal
					
%						\begin{center}
%							\begin{tikzpicture}[scale = 0.7]
%								\node [draw, align=center, fill={rgb:black,1;white,2}, text = white] at (0,0) {$F_1$};
%								\node [draw, align=center, fill={rgb:black,1;white,2}, text = white] at (0,-1) {$F_2$};
%								\node [draw, align=center, fill={rgb:black,1;white,2}, text = white] at (0,-2) {$F_3$};		
%	
%								\node [draw, align=center, fill={rgb:black,1;white,2}, text = white] at (2,0) {$P_1$};
%								\node [draw, align=center, fill={rgb:black,1;white,2}, text = white] at (2,-1) {$P_2$};
%								\node [draw, align=center, fill={rgb:black,1;white,2}, text = white] at (2,-2) {$P_3$};
%	
%								\draw (0.36,0) -- node[above left] {} (1.64,0);
%								\draw (0.36,-1) -- node[above left] {} (1.64,-1);
%								\draw (0.36,-2) -- node[above left] {} (1.64,-2);
%								
%								\draw [red, thick] (0.36,0) -- (0.44,0);
%								\draw [red, thick] (0.44,0) .. controls (0.53,-0.05) and (0.58, -0.95) .. (0.7,-1);
%								\draw [red, thick] (0.7,-1) -- (0.95,-1);
%								\draw [red, thick] (0.95,-1) .. controls (1.1,-1.05) and (1.25, -1.9) .. (1.5,-2);
%								\draw [red, -to, thick] (1.5,-2) -- (1.64,-2);
%							\end{tikzpicture}
%						\end{center}

						\end{itemize}
					\item Según el extremo que recibe la perturbación
					
						\begin{itemize}
							\item Paradiafonía: Perturbación recibida en el mismo extremo que se genera la señal, conocida como \textit{NEXT} (Near End Cross Talk).
							
%								\begin{center}
%%									\begin{tikzpicture}[scale = 0.7]
%										\node [draw, align=center, fill={rgb:black,1;white,2}, text = white] at (0,0) {$F_1$};
%										\node [draw, align=center, fill={rgb:black,1;white,2}, text = white] at (0,-1) {$F_2$};	
%		
%										\node [draw, align=center, fill={rgb:black,1;white,2}, text = white] at (2,0) {$P_1$};
%										\node [draw, align=center, fill={rgb:black,1;white,2}, text = white] at (2,-1) {$P_2$};
%	
%										\draw (0.36,0) -- node[above left] {} (1.64,0);
%										\draw (0.36,-1) -- node[above left] {} (1.64,-1);
%										\draw [red,  -to, thick] (0.35,0) .. controls (1.10,0) and (1.1,-1) .. (0.35,-1);
%									\end{tikzpicture}
%								\end{center}
							
							\item Telediafonía: Recibida en el extremo opuesto. Conocida como \textit{FEXT} (Far End Cross Talk).
							
%								\begin{center}
%									\begin{tikzpicture}[scale = 0.7]
%										\node [draw, align=center, fill={rgb:black,1;white,2}, text = white] at (0,0) {$F_1$};
%										\node [draw, align=center, fill={rgb:black,1;white,2}, text = white] at (0,-1) {$F_2$};	
%	
%										\node [draw, align=center, fill={rgb:black,1;white,2}, text = white] at (2,0) {$P_1$};
%										\node [draw, align=center, fill={rgb:black,1;white,2}, text = white] at (2,-1) {$P_2$};
%	
%										\draw (0.36,0) -- node[above left] {} (1.64,0);
%										\draw (0.36,-1) -- node[above left] {} (1.64,-1);
%										\draw [red,  -to, thick] (0.5,0) .. controls (1.10,0) and (0.7,-1) .. (1.5,-1);
%									\end{tikzpicture}
%							\end{center}
				\end{itemize}
		\end{itemize}
\end{itemize}

Parámetros de medida de la diafonía:

\begin{itemize}
	\item $P_1$: Potencia de la señal en un punto del circuito perturbador.
	\item $P_2$: Potencia de la señal perturbada medida en un punto equivalente del circuito perturbado.
\end{itemize}

\begin{itemize}
	\item Relación de Diafonía ($R_d$):
		\begin{equation*}
			R_d = 10 \log_{10} \left( \frac{P_2}{P_1} \right)
		\end{equation*}
	\item Atenuación de Diafonía ($A_d$):
		\begin{equation*}
			A_d = 10 \log_{10} \left( \frac{P_1}{P_2} \right) = - R_d
		\end{equation*}
	\item Cross Talk Unit (CU):
		\begin{equation*}
			CU = 20 \log_{10} \left( \frac{V_2}{V_1} \cdot 10^6 \right) = 120 - A_d
		\end{equation*}
\end{itemize}


\begin{center}
	\begin{tikzpicture}[scale = 0.3]
	
	\draw[red, very thick] (0.5,5) -- (6, 3.5) node[right] {Telediafonía};;
	\draw[green, very thick] (0.5,3) -- (6, 2) node[right] {Paradiafonía};
	
	\draw[gray, dashed, thick] (0.75,5) -- (0.75, 0.5);
	\draw[gray, dashed, thick] (5.75,3.5) -- (5.75, 0.5);
	
	\begin{axis}[axis lines = middle,xmin=-0.1,xmax = 3.2,ymin=-0.3,ymax=3, xticklabels={,,}, yticklabels={,,}] ,grid=both]
		
		%TERMINAR AQUI
		
			% Ejes
		\node at (axis cs:0, 3) [anchor=north west] {$A_d$};
		\node at (axis cs:3.2, 0) [anchor=south east] {$f$};
		
		\draw[red, very thick] (0,0) -- (2.5,1.5);% -- (0,1);
		\draw (0,0) -- (2,2);
		
	\end{axis}
	\end{tikzpicture}
\end{center}

\subsection{Ruido}

\subsubsection{Ruido Térmico}

\begin{IEEEeqnarray*}{rCl}
	n & = & k \cdot t \cdot B \\
	N(dBm) & = & 10 \cdot \log_{10} (ktB) + 30
\end{IEEEeqnarray*}

\begin{itemize}
	\item $k$: Constante de Boltzmann ($1.38 \cdot 10^{-23} W/K/Hz$)
	\item $t$ (Kelvin): Temperatura
	\item $b$ (Hz): Ancho de banda
\end{itemize}

\subsubsection{Ruido en un Cuadripolo}

\begin{center}
	\begin{tikzpicture}[scale = 0.7]		
		\draw [draw, align=center, fill={rgb:black,1;white,2}, text = white] (0,0) rectangle (2,2) node[pos=.5] {Cuadripolo}; 
	
		\draw (-0.5,0.5) -- node[above left] {$n_e$} (0,0.5);
		\draw (-0.5,1.5) -- node[above left] {$S_e$} (0,1.5);
				
		\draw (2,0.5) -- node[above right] {$S_s = S_e \cdot g$} (2.5,0.5);
		\draw (2,1.5) -- node[above right] {$n_s = n_e \cdot g + n_{interno}$} (2.5,1.5);
	\end{tikzpicture}
\end{center}

Parámetros de caracterización del ruido:

\begin{itemize}
	\item \textbf{Temperatura Equivalente de Ruido ($T_{eq}$)}: Temperatura a la que tendría que estar la entrada del circuito para que a la salida se vea el mismo ruido que se produce suponiendo que el cuadripolo es ideal.
	
		\begin{equation*}
			n_{int} = k \cdot t_{eq} \cdot b \cdot g
		\end{equation*}
	
	\item \textbf{Factor de Ruido en un Cuadripolo ($f$)}: Cociente entre la potencia de ruido a la salida comparada con la potencia de ruido que habría a la salida si la entrada estuviera a temperatura estándar y el cuadripolo no añadiera ruido térmico.
		
		\begin{IEEEeqnarray*}{rCl}
			f & = & \frac{n_s}{k \cdot t_o \cdot b \cdot g} = 1 + \frac{t_{eq}}{t_o} \hspace{20px} f= \frac{\left( \frac{S}{N} \right)_e}{\left( \frac{S}{N} \right)_s} \\	
			F & = & 10 \cdot \log_{10} (f) = \left( \frac{S}{N} \right)_e - \left( \frac{S}{N} \right)_s
		\end{IEEEeqnarray*}
\end{itemize}

Relación entre $t_{eq}$ y $f$:

	\begin{equation*}
		t_{eq} = t_0 \cdot (f - 1) \hspace{20px} f = 1 + \frac{t_{eq}}{t_o}
	\end{equation*}

\subsubsection{Asociación de Cuadripolos}

\begin{center}
	\begin{tikzpicture}[scale = 0.65]
		
		\draw [draw, align=center, fill={rgb:black,1;white,2}, text = white] (0,0) rectangle (2,2) node[pos=.5] {$g_1$ $f_1$ $t_{eq_1}$}; 
	
		\draw (-0.5,0.5) -- node[above left] {} (0,0.5);
		\draw (-0.5,0.5) -- (-0.5,0.25);
		\draw (-0.575,0.25) -- (-0.425,0.25) -- (-0.425,-0.25) -- (-0.575,-0.25) -- (-0.575,0.25); % Resistencia
		\draw (-0.5,-0.25) -- (-0.5,-0.5);
		\draw (-0.5,-0.25) -- (-0.5,-0.5);
		\draw (-0.625,-0.5) -- (-0.375,-0.5);
		\draw (-0.5,-0.25) node[above left] {$t_{eq_1}$} (-0.5,0);

		\draw (-1.5,1.5) -- node[above,xshift=-0.4cm] {$n_e = k \cdot t_e \cdot b$} (0,1.5);
		
				
		%\draw (2,1.5) -- node[above right] {$k \cdot t_e \cdot b \cdot g_1$} (2.5,1.5);
		\draw (2,1.5) -- node[above right] {} (4,1.5);
		
		
		\draw [draw, align=center, fill={rgb:black,1;white,2}, text = white] (3.5,0) rectangle (5.5,2) node[pos=.5] {$g_2$ $f_2$ $t_{eq_2}$}; 
	
		\draw (3,0.5) -- node[above left] {} (3.5,0.5);
		\draw (3,0.5) -- (3,0.25);
		\draw (3.075,0.25) -- (2.925,0.25) -- (2.925,-0.25) -- (3.075,-0.25) -- (3.075,0.25); % Resistencia
		\draw (3,-0.25) -- (3,-0.5);
		\draw (3,-0.25) -- (3,-0.5);
		\draw (3.125,-0.5) -- (2.875,-0.5);
		\draw (3,-0.25) node[above left] {$t_{eq_2}$} (3,0);
		
		\draw (5.5,1.5) -- node[above right] {} (7.5,1.5);
		
		\draw [draw, align=center, fill={rgb:black,1;white,2}, text = white] (7,0) rectangle (9,2) node[pos=.5] {$g_3$ $f_3$ $t_{eq_3}$}; 
	
		\draw (6.5,0.5) -- node[above left] {} (7,0.5);
		\draw (6.5,0.5) -- (6.5,0.25);
		\draw (6.575,0.25) -- (6.425,0.25) -- (6.425,-0.25) -- (6.575,-0.25) -- (6.575,0.25); % Resistencia
		\draw (6.5,-0.25) -- (6.5,-0.5);
		\draw (6.5,-0.25) -- (6.5,-0.5);
		\draw (6.625,-0.5) -- (6.375,-0.5);
		\draw (6.5,-0.25) node[above left] {$t_{eq_3}$} (6.5,0);
		
		\draw (9,1.5) -- node[above right] {} (9.5,1.5);

	\end{tikzpicture}
\end{center}

\begin{equation*}
	n_s = k \cdot b \cdot g_1 \cdot g_2 \cdot g_3 \left( t_o + t_{eq_1} + \frac{t_{eq_2}}{g_1} + \frac{t_{eq_3}}{g_1 \cdot g_2} \right)	
\end{equation*}

Fórmula de Friis:

	\begin{equation*}
		f_T = f_1 + \frac{f_2 - 1}{g_1} + \frac{f_3 - 1}{g_1 \cdot g_2} + ... + \frac{f_n - 1}{g_1 \cdot g_2 \cdot ... \cdot g_{n-1}}
	\end{equation*}

\section{\underline{Tráfico}}

\subsection{Tráfico Telefónico}

El \textbf{tráfico} es una medida del conjunto de peticiones de uso y de ocupación de los recursos de un determinado sistema de telecomunicaciones.

\begin{itemize}
	\item \textbf{Ritmo de afluencia de las llamadas} ($\lambda$, $\frac{\mbox{Número de Llamadas}}{\mbox{Tiempo}}$)
	\item \textbf{Tiempo medio de duración} de las llamadas ($T_m$)
	\item \textbf{Volumen de Tráfico}: Tiempo de ocupación de los recursos, para $N$ circuitos:
		\begin{equation*}
			V(N) = \sum_i V_i
		\end{equation*}
		Se mide en:
		\begin{itemize}
			\item LLR: Llamadas reducidas - 120 segundos $\rightarrow 1(E) = 30 \frac{LLR}{H}$
			\item CCS: \textit{Century Call Seconds} - 100 segundos $\rightarrow 1(E) = 30 \frac{LLR}{H}$
		\end{itemize}
	\item \textbf{Intensidad de Tráfico (\textit{A})}: Volumen a lo largo de un periodo de observación, se mide en \textit{Erlangs}.
		\begin{equation*}
			A = \frac{t_{\mbox{ocupación}}}{t_{\mbox{observación}}} = \lambda \cdot t_{medio}
		\end{equation*}
	\item \textbf{Tiempo de Observación para las medidas del tráfico (\textit{A})} El tráfico depende tanto de la duración como de la distribución de llegada de las llamadass
\end{itemize}

\subsection{Bloqueo - Llamadas Perdidas - GoS - Disponibilidad}

\begin{itemize}
	\item \textbf{Tráfico Ofrecido ($A_O$)}: Tráfico que soportaría la red si fuera capaz de servir todas las solicitudes de servicio.
	\item \textbf{Tráfico Bloqueado ($A_B$)}: Tráfico rechazado por ocupación de todos los circuitos $B \cdot A_O$.
	\item \textbf{Tráfico Cursado ($A_C$)}: Tráfico servido por la red $A_O (1-B)$.
\end{itemize}

\begin{itemize}
	\item En un sistema sin pérdidas: $A_O = A_C$.
	\item En un sistema con pérdidas: $A_O = A_C + A_B$.
	\item Con $N$ circuitos o servidores, $\rho = \frac{A}{N}$ será el tráfico, ofrecido/cursado, por circuito o servidor.
\end{itemize}

Un \textbf{conmutador tiene disponibilidad total} cuando cada entrada tiene acceso a cada una de las salidas.

\subsection{Distribuciones Estadísticas para Fuentes de Tráfico}

\begin{itemize}
	\item Duración de llamada constante: redes de conmutación de paquetes
	\item Duración de llamadas exponencial negativa: conversación telefónica
\end{itemize}

%\subsection{Modelos de Gestión de Llamadas Bloqueadas}
%
%\begin{itemize}
%	\item \textbf{Lost Calls Held (\textit{LCH})}: Práctica norteamericana, la llamada se pierde y el usuario volverá a intentarlo de forma inmediata. El segundo intento está estadísticamente relacionado con el primero.
%	\item \textbf{Lost Calls Cleared (\textit{LCC})}: Práctica europea, la llamada se pierde y el usuario dejará pasar cierto tiempo antes de volver a intentarlo. El segundo intento está considerado como una petición aleatoria más.
%	\item \textbf{Lost Calls Delayed (\textit{LCD})}: La llamada no se pierde, existe una cola de espera hasta que se libere algún acceso.
%	\item \textbf{Lost Calls Retried (\textit{LCR})}: Variación de \textit{LCC}, es un caso especial
%\end{itemize}

\subsection{Modelo de Llamadas Perdidas Despejadas}

\subsubsection{Modelo LLC $\rightarrow$ Erlang-B}

Distribución Erlang B para el cáclulo de la \textbf{probabilidad de bloqueo}

	\begin{equation*}
		B(N,A) = \frac{\frac{A^N}{N!}}{\sum_{i=0}^N \frac{A^i}{i!}}
	\end{equation*}

\qquad \textit{$B(N,A)$: Probabilidad de Bloqueo}\\
\qquad \textit{$N$: Número de órganos} \\
\qquad \textit{$A$: Tráfico ofrecido}

\subsubsection{Sistemas con Retardo - LCD $\rightarrow$ Erlang-C}

Las solicitudes de servicio que encuentran todos los servidores ocupados son puestas en una cola. Los servidores verán un ritmo constante de llegadas. Parámetros:

\begin{itemize}
	\item Tiempo de Servicio o Tiempo de Ocupación ($T_O$).
	\item Tiempo de Espera ($T_w$).
	\item Tiempo total en el sistema ($T_s = T_m + T_w$).
\end{itemize}

%Terminología de Colas
%
%\begin{enumerate}
%	\item \textbf{Input Specification}:
%		\begin{itemize}
%			\item G: General (no assumptions)
%			\item M: Purely random
%		\end{itemize}
%	\item \textbf{Service Time Distribution}:
%		\begin{itemize}
%			\item G: General (no assumptions)
%			\item M: Negative Exponential
%			\item D: Constant
%		\end{itemize}
%	\item \textbf{Number of Sources}:
%		\begin{itemize}
%			\item M: Finite
%			\item  : Indinite
%		\end{itemize}
%	\item \textbf{Queue Length}:
%		\begin{itemize}
%			\item L: Finite Length
%			\item  : Infinite Length
%		\end{itemize}
%\end{enumerate}

\subsubsection{Sistemas M/M/N (Erlang-C)}

Llegadas aleatorias, tiempo de servicio exponencial y $N$ servidores.

	\begin{equation*}
		p(t_w > t) = C(N,A) \cdot e^{- \frac{(N-A)t}{T_m}}
	\end{equation*}
	
	\begin{equation*}
		T_w = \frac{C(N,A) \cdot T_m}{N - A}
	\end{equation*}
	
Número medio de usuarios en cola:

	\begin{equation*}
		u_w = \lambda \cdot T_w
	\end{equation*}

%\begin{equation*}
%	C(N,A) = \frac{N \cdot E_B}{[ N - A (1 - E_B) ]}
%\end{equation*}
%
%\begin{itemize}
%	\item $N$: Números de servidores del sistema
%	\item $A$: Tráfico ofrecido al sistema
%\end{itemize}

\subsubsection{Sistemas M/M/1}

Llegadas aleatorias, tiempo de servicio exponencial y $1$ servidor.

	\begin{equation*}
		C(N,A) = A = \rho
	\end{equation*}

	\begin{equation*}
		p(t_w > t) = A \cdot e^{- \frac{(1-A)t}{T_m}}
	\end{equation*}
	
	\begin{equation*}
		T_w = \frac{\rho \cdot T_m}{1 - \rho}
	\end{equation*}

\subsubsection{Sistemas M/D/1}

Llegadas aleatorias, tiempos de servicio fijos y 1 servidor.

	\begin{equation*}
		T_w = \frac{\rho \cdot T_m}{2 \cdot (1 - \rho)}
	\end{equation*}

	\begin{equation*}
		p(t_w > 0) = A = \rho
	\end{equation*}
	
	
\end{multicols}
\end{landscape}


\begin{framed}
	\begin{center}
    	\Large{\underline{Sistemas de Telecomunicación}} \\
    	\scriptsize{3º Ingeniería de Telecomunicaciones | UPV/EHU}\\
     	%Actualizado por última vez el \today \\
     	"\textsl{Under-promise and over-deliver}." \\
     	%\hspace{5 pt} \\
     	\small{\textbf{Javier de Martín -- 2016}}
	\end{center}
\end{framed}

%%%%%%%%%%%%%%%%%%%%%%%%%%%%%%%%%%%%%%%%%%%%%%%%%%%%%%%%%%%%%%%%%
% Tema 4

\hrulefill

\section{4. Redes de Acceso}

\hrulefill

``\textit{El objetivo es asegurar la comunicación oral entre los usuarios del servcio atendiendo a los estándares de la ITU que fijan las normas para obtener un servicio mínimo de calidad}''\\

Para proveer el servicio es necesario:

\begin{itemize}
	\item Red o conjunto de medios que posibiliten el \textbf{acceso a los usuarios del servicio}: terminales de abonado, medios de transmisión (bucle local) y centrales de acceso y conmutación (centrales locales - remotas).
	\item Red o conjunto de \textbf{medios que interconectan los medios de acceso de los usuarios} para proveer conectividad total entre los usuarios: Redes de transporte y centros de conmutación.
\end{itemize}

La \textbf{red} se define como un método de interconexión de centrales para poder proveer conectividad total. En telefonía hay tres métodos utilizados para interconectar centrales:

	\begin{itemize}
		\item Malla (todos con todos): Eficiente en zonas con usuarios concentrados cerca de los nodos y si el tráfico de usuarios es alto, el número de enlaces es alto, $\frac{N(N-1)}{2}$, y la escalabilidad es compleja.
		\item Estrella (a través de un centro de y estrella doble (varias estrellas a través de un centrogg de tránsito de segundo orden): Menor número de conexiones ($N$), necesita nodos intermedios de conmutación, óptimo en lugares con poco tráfico, permite dar acceso a zonas aisladas y su escalabilidad es sencilla.
		\item Estrella doble: Varias estrellas a través de un centro de tránsito de segundo orden.
	\end{itemize}

\subsection{RDI: Red Digital Integrada}

	\begin{center}
		\includegraphics[scale=0.3]{images/RDII}
	\end{center}

La \textbf{red de acceso} son centrales digitales remotas, compuestas por:

	\begin{itemize}
		\item  \textbf{Concentradores}: son equipos de conmutación para dar acceso a abonados alejados de la central, las funciones de conmutación las realiza la central autónoma, el camino telefónico de una llamada local pasa por la central autónoma.

				\begin{center}
					\includegraphics[scale=0.2]{images/RedAcceso}
				\end{center}
		
		\item \textbf{URA (Unidades Remotas de Abonados)}: equipos con más capacidad de accesos que el concentrador, están conectados a la central autónoma mediante sistemas MIC \footnote{Modulación por Impulsos Codificados} por cable o bien enlaces de fibra, su funcionamiento es más autónomo que en el caso de los concentradores y tienen capacidad de conmutación para los enlaces locales.

			\begin{center}
				\includegraphics[scale=0.2]{images/RDI}
			\end{center}
	\end{itemize}

		
\hrulefill		

\subsection{Terminal Telefónico}

	\begin{center}
		\includegraphics[scale=0.3]{images/TerminalTelefonico}
	\end{center}

El primer terminal telefónico era un sistema basado en \textbf{un transmisor, un receptor} y una batería dispuestos en serie y funcionaba por el principio de resistencia variable. El problema era la potencia del transmisor ya que no existían amplificadores. En la actualidad el transmisor es un micrófono de carbón (no lineal y muy sensible).

El \textbf{bucle de abonado} es una conexión física con la central telefónica, inicialmente era un único hilo (aprovechando la tierra) y actualmente son dos hilos (cable de pares). Nació como líneas dedicadas entre dos usuarios, posteriormente fueron líneas de conmutación manual (centralitas) y finalmente son centrales de conmutación (strowger \footnote{Conmutador basado en telerruptores})


La \textbf{bobina de inducción} era inicialmente un transmisor y un receptor en serie. Tenían \textbf{problemas} como la impedancia del altavoz, difícil adaptación de impedancias y la corriente continua que circulaba a través del receptor reducía su eficiencia. Como \textbf{solución} se utiliza la bobina de inducción, se aíslan los TX y RX facilitando la adaptación y evitando que circule corriente continua por el altavoz.

%\subsubsection{El Efecto Local (Sidetone)}

Sigue habiendo un \textbf{problema}, las señales del micrófono se escuchan en el auricular, \textbf{efecto local (\textit{sidetone})}. Hay un tercer camino de alta ganancia entre el altavoz y auricular que hace que el usuario se escuche a sí mismo demasiado alto y tienda a hablar más bajo. La adaptación no es perfecta ($Z_{L}$ variable), pero \textbf{un cierto nivel de sidetone es beneficioso}.

%\subsubsection{Alimentación y llamada}

La \textbf{alimentación} se realiza desde la central local con $48V (CC)$. El consumo de corriente del terminal permite identificar y dar servicio a los usuarios (estados: \textit{on-hook}, colgado, y \textit{off-hook}, descolgado). El \textbf{timbre} está conectado en paralelo y es de alta impedancia, la señal que se envía a la central es de $75 V_{rms}$ y $50Hz$. 

%\subsubsection{Marcación}

La \textbf{marcación} puede realizarse de dos formas: por pulsos/decamétrica o DTMF (\textit{Dual Tone Multifrecuency}).

\subsubsection{Otros tipos de terminales: Módems}

	\begin{center}
		\includegraphics[scale=0.3]{images/SpecsModem}
	\end{center}

La ITU define dos tipos de módems:

\begin{itemize}
	\item \textbf{Modems Digitales}: Envía señales G.711 y recibe señales V.34 codificadas con el estándar V.34. Se conecta a una red con conmutación digital con un interfaz digital.
	\item \textbf{Modems Analógicos}: Generan señales V.34 y reciben señales G.711 que han sido decodificadas en una central local de abonado telefónico y preparadas para su envío a través de un bucle de abonado.
\end{itemize}

\begin{center}
	{\scriptsize G.711 - Pulse Code Modulation (PCM) of voice frequencies\\
	V.34 - A modem operating (up to 33.600 bit/s) for use in 2-wire analog PSTN}
\end{center}

\textbf{Funciones} de un modem:

	\begin{enumerate}
		\item \textbf{Circuitos para compatibilizar la señalización RDI}: Descolgado, marcado y detección de tono marcado.
		\item \textbf{Circuitos de envío de la señal de datos} en la banda de frecuencias vocal
		\item \textbf{Transmisión a cuatro hilos equivalentes}: Diferentes portadoras en cada sentido e inclusión de filtros para minimizar interferencias.
	\end{enumerate}
	
	\begin{center}
		\includegraphics[scale=0.3]{images/FuncModem}
	\end{center}
	
%\subsubsection{Operación de un Modem}

\textbf{?`Cómo opera un módem?} El módem receptor espera en answer mode, en el otro extremo espera en call mode y se simula el off-hook escuchando el tono de invitación a marcar y se envían pulsos o tonos para marcar. El modem en modo answer detecta las señales de llamada, simula el off-hook y envía una portadora. El modem en modo llamada envía una portadora y se intercambian los datos ambos.

%\subsubsection{Módem V.34}

El \textbf{módem V.34} permite mayor número de bits por símbolo y permite modulación QAM transmitiendo información en la amplitud y en la fase, incluye códigos de corrección y permite compresión. Estos módems se diseñaron para optimizar la situación en la que ambos lados de la comunicación eran analógicos.

%	\begin{center}
%		\includegraphics[scale=0.2]{images/ModemV34}
%	\end{center}

	\begin{figure}[!ht]
 		\centering
  		 \includegraphics[scale = 0.25]{images/ModemV34}
		\caption{Módem V34}
	\end{figure}

Este tipo de módems está \textbf{limitado} por el ruido de cuantificación. El estándar V34 no tiene en cuenta la arquitectura real de la conexión entre la red telefónica y el proveedor de acceso a internet, siendo el bucle local la única parte analógica de la transmisión, el ruido de cuantificación sólo está presente en la parte del enlace ascendente.

%\subsubsection{Módem V90}

El \textbf{módem V90} es el estándar doméstico desde 1998, el principio de funcionamiento se basa en que el ISP tiene una conexión digital con la central local. Por lo tanto se asume que solo existe una conversión analógico-digital en el camino hasta el ISP.

%	\begin{center}
%		\includegraphics[scale=0.2]{images/ModemV90}
%	\end{center}

	\begin{figure}[!ht]
 		\centering
  		 \includegraphics[scale = 0.25]{images/ModemV90}
		\caption{Módem V90}
	\end{figure}

Los \textbf{terminales de facsímil} siguen estándares de la ITU-T. Los faxes se dividen en varios grupos según si el escaneado es digital y su velocidad de transmisión.

\hrulefill

\subsection{Bucle de Abonado}

Medio físico que une la red telefónica con el terminal de abonado. En la RDI este acceso es analógico y basado en cable de pares, representa gran parte del coste de una red. Existen tecnologías alternativas pero sigue siendo el predominante en usuarios domésticos. Va a seguir implantada durante mucho teimpo $\rightarrow$ demanda de servicio de voz únicamente.

Distintos medios de acceso por pares:

	\begin{center}
		\includegraphics[scale=0.2]{images/BucleIntro}
	\end{center}

El \textbf{interfaz con la red telefónica}: La central local

	\begin{center}
		\includegraphics[scale=0.2]{images/CentralLocal}
	\end{center}

Esquema de \textbf{distribución de cable de pares}

	\begin{center}
		\includegraphics[scale=0.2]{images/Distribucion}
	\end{center}	

%\subsubsection{Bridged - Taps}

Los \textbf{bridged - taps} son un método de cableado para redes telefónicas. Un par de cables ``aparecerá'' en distintas localizaciones de terminales permitiendo a la compañía telefónica asignar ese par a cualquier cliente que esté cerca de ese terminal. Una vez ese cliente se desconecta, el par se convierte en disponible para cualquiera de los terminales. Esta conexión es una ``T'' en el cable, no tiene bobina híbrida o un componente de adaptación de impedancias produciendo reflexiones de la señal en el cable.

	\begin{center}
		\includegraphics[scale=0.2]{images/Taps}
	\end{center}

%\subsubsection{Características de los cables }

Los \textbf{cables se clasifican} según su galga (diámetro), según su categoría se establece su velocidad de transmisión y su aplicación varía. La distancia máxima del bucle del abonado está limitada por \textbf{limitación óhmica} y \textbf{limitación por pérdidas (atenuación)}.

\subsubsection{Centrales locales: Interfaz con el bucle de abonado}

Central Local

	\begin{center}
		\includegraphics[scale=0.2]{images/CentralFuncional}
	\end{center}
	
La central local tiene como \textbf{funciones} 

	\begin{itemize}
		\item  \textbf{ mantenimiento} en la que supervisa las líneas de abonado y los enlaces
		\item  \textbf{funciones de operación} en las que maneja datos administrativos y datos estadísticos y puede añadir \textbf{servicios de valor añadido} si así lo desea
	\end{itemize}

	\begin{center}
		\includegraphics[scale=0.2]{images/CentralFuncional2}
	\end{center}

El interfaz con la red telefónica: La Central Local

	\begin{center}
		\includegraphics[scale=0.2]{images/InterfazRed}
	\end{center}

%\subsubsection{Perturbaciones en el bucle de abonado}

La \textbf{bobina híbrida} permite separar los canales de ida y vuelta. Las conexiones entre abonados son a dos hilos y las conexiones entre centrales son a cuatro. Al ser elementos ideales, en las redes telefónicas se ocasionan los problemas de ECO y CANTO.
%
%	\begin{center}
%		\includegraphics[scale=0.2]{images/BobinaHibrida}
%	\end{center}
	
	\begin{figure}[!ht]
 		\centering
  		 \includegraphics[scale = 0.25]{images/BobinaHibrida}
		\caption{Bobina híbrida}
	\end{figure}
	
\textbf{Pérdidas de retorno}:

	\begin{equation*}
		A_{R} =  20 \cdot \log \left( \left| \frac{1}{\rho} \right| \right) = 20 \cdot \log \left| \frac{Z_{L} + Z_{E}}{Z_{E} - Z_{L}} \right|
	\end{equation*}

\textbf{Atenuación Transhíbrida}:

	\begin{equation*}
		A_{TH} = 2 \alpha + A_{R}
	\end{equation*}
	
\textbf{Estudio de la Inestabilidad}:	
	
Pérdida entre extremos a 2 hilos: $T = 2 \alpha + L - G$

Margen de estabilidad $S = \frac{M}{2}$. El bucle a cuatro hilos será estable si no tiene ganancia, es decir, si $M = 0$. Habitualmente $S = 3 dB$.	
	
En la bobina híbrida aparecen dos tipos de \textbf{eco}:

	\begin{itemize}
		\item \textbf{Eco del hablante}: el hablante escucha un eco de su propia voz
		\item \textbf{Eco del oyente}: el oyente escucha un eco de la señal que escucha
	\end{itemize}
	
Para solucionar estos problemas se utilizan dos componentes. \textbf{NES} (\textit{Network Echo Supressor}) que anula la línea de transmisión cuando se detecta la señal recibida. Es equivalente a una comunicación semi-dúplex.
	
%	\begin{center}
%		\includegraphics[scale=0.3]{images/NES}
%	\end{center}

	\begin{figure}[!ht]
 		\centering
  		 \includegraphics[scale = 0.25]{images/NES}
		\caption{NES}
	\end{figure}
	
El \textbf{NEC} (\textit{Network Echo Canceller}) elimina en la rama de transmisión la señal de eco calculada a partir de la rama de recepción.

%	\begin{center}
%		\includegraphics[scale=0.3]{images/NEC}
%	\end{center}

	\begin{figure}[!ht]
 		\centering
  		 \includegraphics[scale = 0.25]{images/NEC}
		\caption{NEC}
	\end{figure}

\textbf{Interfaces PSTN (\textit{Public Switched Telephone Network}) e ISDN (\textit{Integrated Service Data Network })en una central local}.

	\begin{center}
		\includegraphics[scale=0.35]{images/ISDN}
	\end{center}

\textbf{Interfaces}:

	\begin{itemize}
		\item \textbf{Interfaz V1}: acceso básico para la RDSI, sus funciones básicas son: $2B + D$ ($2 \cdot 64 kbps + 16 kbps$ canalización), temporización y sincronismo de trama, activación y desactivación del terminal de línea, operación y mantenimiento, alimentación... 
		\item \textbf{V2 Interfaz para la conexión con concentradores}: 2048 kbit/s, $30B + D$.
		\item  \textbf{V3 similar a V2 pero pensado para interfaz con centralitas (PABX)}: $30B + D$ a 2048 kbit/s y también $23B+D$ a 1544 kbit/s.
		\item  \textbf{V4}: Interfaz con redes privadas no especificadas por la ITU-T.
		\item  \textbf{V5}: Interfaz genérico entre la red de acceso y la central local, especifica los interfaces básicos para el acceso analógico y acceso RDSI. Incluye el control de los canales de acceso y funciones de señalización. Se utiliza en DLCs.
	\end{itemize}

Existen diversas fuentes de \textbf{perturbación en el bucle de abonado} para las señales que viajan por el cable de pares en el bucle de abonado.

	\begin{itemize}
		\item \textbf{Distorsión de canal}: Causado por la variación no lineal en frecuencia de las velocidades de transmisión en el cable de pares.
		\item \text{Atenuación}: Causado por el efecto Joule y por pérdida de energía por radiación electromagnética.
		\item \textbf{Diafonía o Crosstalk}: Causado por el acoplamiento de señales entre pares.
		\item \textbf{Ruido}: Térmico (movimiento pseudoaleatorio de electrones) o Impulsivo (equipos y ruido humano).
		\item \textbf{Interferencia Electromagnética}: Acoplamiento de señales de radiofrecuencia en el cable de pares.
		\item \textbf{Eco}: Causado por desequilibrios en la bobina híbrida en el paso de dos hilos a cuatro hilos.
	\end{itemize}

\textbf{Atenuación de diafonía}. Cualquier señal al propagarse por un conductor no ideal pierde potencia. La pérdida en decibelios en el bucle de abonado es proporcional a la distancia y a la atenuación del cable de pares. La atenuación se da en $\frac{dB}{Km}$ o $\frac{dB}{100 m}$ y depende fundamentalmente de las características del cable (conductor) y su grosor. La atenuación no es constante con la frecuencia. En el cable de pares aumenta rápidamente la atenuación kilométrica a frecuencias altas. Para la transmisión de la señal vocal no es un problema en los bucles de abonado típicos, el problema surge en la transmisión de datos utilizando dicho cable.

%\subsubsection{DLC (Digital Loop Carriers)}

Un \textbf{DLC (\textit{Digital Loop Carrier})} es un sistema que utiliza transmisión digital para extender el rango del bucle local más de lo que podría llegar el par de cobre. El bucle de abonado es el medio físico que une la central local con el terminal de abonado. Existen varios medios para unir estos abonados con la central: Reducción del número de cables (concentración), acortamiento del bucle de abonado, posibilidad de equipos de intemperie, densidades de población baja, reemplazar planta exterior obsoleta, introducción de nuevos servicios en puntos donde las centrales no los soportan. Los medios físicos no tienen por qué ser cables de pares, pueden ser de acceso por radio, fibra óptica...

	\begin{center}
		\includegraphics[scale=0.4]{images/DLC1}
	\end{center}
	
	\begin{center}
		\includegraphics[scale=0.4]{images/DLC2}
	\end{center}

Los DLCs utilizan el \textbf{Protocolo V.5}: Definido para el interfaz con la central local.

	\begin{itemize}
		\item \textbf{V.5-1}: Enlace digital de 2Mbps, necesita software necesario para la transmisión de los eventos de línea en los 30 puertos asociados a la interfaz. Los nodos extremos intercambias mensajes referentes a los protocolos.
		\item \textbf{V.5-2}: Conjunto de enlaces de 2Mbps, hasta 16 enlaces. La asignación se realiza llamada a llamada (concentración), los extremos intercambian información.
	\end{itemize}
	
	\begin{center}
		\includegraphics[scale=0.3]{images/DLCs}
	\end{center}
	
\hrulefill

\subsection{ADSL}

La DSL (\textit{Digital Subscriber Line}) es una familia de tecnlologías que proporcionan el acceso a internet mediante la transmisión de datos digitales a través de los cables de una red telefónica local. El ADSL (\textit{Asymmetric Digital Subscriber Line}), la capacidad de descarga  y de subida no coinciden. Diseñado para que la capacidad de bajada sea mayor que la de subida.

\subsubsection{Arquitectura ADSL}

	\begin{center}
		\includegraphics[scale=0.3]{images/ADSL}
	\end{center}
	
	\begin{itemize}
		\item \textbf{ATU-R}: Módem digital que se instala en el domicilio del usuario y que está conectado a la línea telefónica y a la computadora, los datos son modulados mediante DMT.
		\item \textbf{Splitter}: Filtro que divide las frecuencias bajas en una banda de $0-4kHz$ y las altas frecuencias en una banda mayor a 4kHz. Al filtrar las bajas frecuencias se separa la voz y eliminan las interferencias provocadas sobre la transmisión de datos del usuario; de la misma forma al filtrar las altas frecuencias se separan los datos y permite que la transmisión de datos no afecte a la de voz.
		\item \textbf{ATU-C}: Módem digital ubicado en el lado de la central, debe de utilizar el mismo tipo de modulación que el ATU-R. Generalmente es una tarjeta que se insertará en el equipo que concentra y mulitplexa el tráfico de datos o DSLAM (\textit{Digital Subscriber Line Access Multiplexer}).
		\item \textbf{Multiplexor ADSL (DSLAM)}: Chasis que agrupa gran número de tarjetas, cada una consta de varios módems ATU-C y además concentra (multiplexa/demultiplexa) el tráfico de todos los enlaces ADSL hacia una red WAN. Realiza funciones de nivel de enlace (protocolo ATM sobre ADSL) entre el módem de usuario y el de central.
	\end{itemize}
	
Cuando se pasa de ADSL a ADSL Lite la instalación se hace en la centralita por lo que se puede quitar el splitter de la casa del cliente.
	
\subsubsection{Problemas y Soluciones}

Uno de los mayores problemas es la atenuación de los cables, la diafonía, interferencias y ruido. Los errores se detectan y corrigen mediante FEC (\textit{Forward Error Coding}). Los bits se entrelazan para reducir los errores no corregibles (más tramas con errores, menos errores por trama). Los módems ADSL permiten definir circuitos virtuales especiales para aplicaciones sensibles al retardo. Los datos se transportan en tramas y supertramas (conjunto de 68 símbolos DMT), con 2 tipos de flujos: rápido y lento.

\subsubsection{ADS Lite - ADSL Spliterless}

Condiciones para la implantación óptima: Que no sea necesaria la instalación de un splitter por parte del proveedor de servicios, que el usuario pueda instalar el módem fácilmente, que no sea necesario re-cablear la línea, que exista compatibilidad espectral con otros servicios, velocidades adecuadas para el segmento del mercado al que se va a dar servicios, distancias en el bucle de abonado adecuadas para las velocidades a las que se va a dar servicio.\\

A partir de 1997 con un 10\% de las capacidades de ADSL es suficiente para dar internet a alta velocidad. Una reducción en la velocidad de servicio de ADSL se traduciría en una reducción de complejidad del hardware. Se suprime el splitter y permite modulación 256-QAM como máximo.

Para velocidades mayores se utiliza el VDSL que tiene un rango de frecuencias en el cable más amplio (30 MHz). HDSL pretende sustituir los sistemas T1 y E1. Ofrece un canal simétrico de 2Mbps y alcanza como máximo unos 4km. Se usa actualmente para líneas punto a punto de 2Mbps. tiene mayor alcance sin repetidores y al usar menor rango de frecuencias hay menos interferencias.

\hrulefill

\subsection{Sistemas MIC/PCM}

MIC (\textit{Modulación por Impulsos Codificados}) o PCM (\textit{Pulse Coded Modulation}). Diseñado originalmente para la transmisión digital de señales de voz con el objetivo de aprovechar las ventajas de los sistemas digitales. Estudio de: muestreo, cuantificación, codificación de canal, codificación de línea, multiplexación por división en el tiempo, señalización y control.

	\begin{center}
		\includegraphics[scale=0.45]{images/SistemasMIC}
	\end{center}	
	
%\subsubsection{Muestreo}

La frecuencia de \textbf{muestreo} es la máxima frecuencia utilizable de la señal analógica (voz). En telefonía, el ancho de banda de voz es 300-3400Hz. En función del teorema de Nyquist se muestrea con $F_{S} = 2 \cdot F_{max}$ pero por cuestiones prácticas la frecuencia seleccionada es de 8kHz. El valor de la frecuencia de muestreo fija la longitud de la trama en la multiplexación TDM, $125\mu s$.

%\subsubsection{Cuantificación en sistemas MIC}

El proceso de \textbf{cuantificación} en sistemas MIC consiste en dividir el rango de valores de entrada en un número finito de intervalos y asignar un único valor de salida a cada intervalo. Hay que tener en cuenta los factores de:

	\begin{itemize}
		\item Valores analógicos máximos y mínimos de entrada: rango dinámico de entrada del cuantificador.
		\item Número de intervalos de cuantificación: Depende del número de bits de cuantificación e influye en el error de cuantificación.
		\item Tipo de cuantificador: Lineal o logarítmico.
	\end{itemize}

%\subsubsection{Niveles de entrada al cuantificador}

\textbf{Niveles de entrada al cuantificador}, los valores extremos se conocen como valores virtuales de decisión. El nivel de máxima amplitud de una senoide sin recorte de crestas (nivel de sobrecarga). La UIT-T recomienda como nivel de sobrecarga como la amplitud de una señal de $3.14 dBm0$ en el codificador situado tras la bobina híbrida (2h/4h). Las señales de entrada se normalizan a ese valor:  $P(dBm0) = 3.14 + 20 \log (V)$.

%\subsubsection{Número de bits de cuantificación}

\textbf{Número de bits de cuantificación}, el parámetro de partida para fijar el número de bits es la relación señal a ruido de cuantificación $>60-70dB$, un valor habitual para los sistemas telefónicos. El objetivo es no cuantificar valores de señal menores que el ruido, en el caso de utilizar un cuantificador uniforme utilizar como número de bits 12. El cuantificador uniforme es óptimo para señales equiprobables en el ranngo de niveles de entrada. En la señal de voz los valores instantáneos de tensión siguen una distribución gamma: los valores bajos son muy probables y los altos poco probables. La potencia de ruido de cuantificación dependerá más de la amplitud del ruido de cuantificación en los niveles bajos. Será más eficiente utilizar un cuantificador adaptado al comportamiento estadístico de los valores de tensión de la señal vocal: Los intervalos pequeños cerca del origen tienen amplitud del error de cuantificación pequeña y estadísticamente contribuyen mucho a la potencia del error y los intervalos de cuantificación amplios tienen amplitud grande del error de cuantificación pero estadísticamente contribuyen poco a la potencia de error de cuantificación. Un cuantificador que cumple estas condiciones es el cuantificador logarítmico, con 8 bits se consigue una $(\frac{S}{N})_{Q}$ igual a la que se consigue con un cuantificador uniforme de 12 bits (sólo para señales de voz). La gananacia obtenida se conoce como ganancia de compansión y es de 24 dB: $G_{compansion} = 6 dB \cdot 4 bits = 24 dB$.

En realidad, \textbf{el cuantificador logarítmico se construye con un cuantificador uniforme, un compansor y un expansor}. La UIT-T recomienda el uso de dos funciones logarítmicas para la cuantificación MIC:

	\begin{itemize}
		\item \textbf{Ley A}: Europa, es una función continua y se aproxima linealmente por tramo dando lugar a la ley A de 13 segmentos (realmente son 16 segmentos), $A = 87,6$. En el interior de cada segmento se hace una cuantificación uniforme con 4 bits: 16 subintervalos por segmento. La \textbf{palabra MIC} queda definida por 8 bits: 1 bit de signo, 3 bits por segmento (identificación de los 8 intervalos de decisión) y 4 bits por subintervalo dentro de cada segmento (escalón del tramo lineal).
		\item \textbf{Ley $\mu$}: Estados Unidos, Japón y Canadá.
	\end{itemize}


%\subsubsection{Sistemas MIC 30 + 2}

Hasta ahora se ha visto cómo digitalizar la señal vocal, para optimizar el uso de los recursos (enlaces) se multiplexan las señales vocales. En el caso de los sistemas MIC, esta multiplexación se realiza por división en el tiempo. De esta forma, los canales vocales correspondientes a 30 conversaciones se multiplexan en el tiempo junto con información de alineación de trama y bits de servicio y señalización formando una trama \textbf{MIC 30+2}, compuesta por 32 intervalos temporales, la duración de cada trama es de $125 \mu s$ (fijado por la frecuencia de muestreo $8kHz$. La velocidad de salida es de $32 intervalos \cdot 8 bits / intervalo \cdot 8 kHz = 2048 kbps$. Los intervalos se identifican como $I0 - I31$ y la estructura de la trama es:
	
%	\begin{center}
%		\includegraphics[scale=0.4]{images/TramaMIC}
%	\end{center}

	\begin{figure}[!ht]
 		\centering
  		 \includegraphics[scale = 0.4]{images/TramaMIC}
		\caption{Estructura de Multitrama MIC 30+2}
	\end{figure}

Los 32 bytes de la trama tienen los siguientes significados:

	\begin{itemize}
		\item \texttt{I0}: Información de alineamiento de trama, si no se detecta la secuencia \texttt{X0011011} se ha realizado mal el conteo y se activa la alarma.
		\item \texttt{I16}: Información de señalización (4 bits), por tanto, en cada trama sólo se puede enviar información de dos canales. Se utiliza una multitrama para señalizar 30 canales. Todos los bits que van dentro de IX, en esta sección, son palabras MIC.
	\end{itemize}

Un enlace a través de sistemas MIC 30+2 quedaría de la siguiente manera:

%	\begin{center}
%		\includegraphics[scale=0.4]{images/EnlaceMIC}
%	\end{center}
	
	\begin{figure}[!ht]
 		\centering
  		 \includegraphics[scale = 0.4]{images/EnlaceMIC}
		\caption{Enlace a través de un sistema MIC 30+2}
	\end{figure}
	
%\subsubsection{Codificación de Línea}

La \textbf{codificación de línea} consiste en traducir los $0$ y $1$ a niveles de tensión en la línea de transmisión. Las señales que se utilizan para ello han de cumplir unas condiciones:

	\begin{itemize}
		\item Ha de permitir la transmisión de señales en canales con bloqueo de continua. El código no ha de tener componente continua, la densidad espectral de potencia debería tener un nulo en el origen y hay problemas con secuencias de ceros o unos consecutivos.
		\item Ha de contener información de temporización para extraer en el receptor la señal de reloj. Suficiente número de transiciones y hay problemas con secuencias de ceros o unos consecutivos.
		\item En algunos casos al introducir redundancia permitirán detectar errores.
	\end{itemize}
	
	\begin{itemize}
		\item \textbf{Código AMI} (\textit{Aternate Mark Inversion}): El cero se codifica con 0 voltios, el uno se codifica con $V^{+}$ o $V^{-}$ de forma alternada. Las secuencias largas de ceros dan problema con la información de reloj.
		\item \textbf{Código HDB3} (\textit{High Density Bipolar}): Soluciona el problema de secuencias largas de ceros de AMI. No están permitidas secuencias de más de tres ceros seguidos. Si hay 4 ceros consecutivos el 4º se codifica como un 1 pero el nivel alto o bajo opuesto al que corresponda. Si entre el bit $V$ que se va a insertar y el bit $V$ precedente hay un número par de 1s, el primer 0 también se codifica como si fuera un 1 con el nivel alto o bajo que le corresponda.
		\item \textbf{Código HDB4}: Igual que HDB3 pero con 4 ceros consecutivos.
		\item \textbf{Códigos de bloque}: Una secuencia de unos y ceros se sustituye por una secuencia de impulsos que puede ser binaria, ternaria o cuaternaria. Por ejemplo, 3B4B, 3 símbolos binarios se sustituyen por 4 símbolos binarios.
	\end{itemize}
	
%\subsubsection{Eficiencia y Redundancia de Código}

Los códigos de línea mejoran las propiedades de la señal transmitida (frente al nivel de continua y atenuación en baja frecuencia y frente al transporte de la información de la señal de reloj). Estas ventajas se consiguen a costa de aumentar el bitrate del canal de transmisión que será por lo general mayor que el bitrate de fuente. \textbf{Redundancia}:

	\begin{equation*}
		R = \frac{r_{canal} - r_{fuente}}{r_{canal}}
	\end{equation*}
	
\textbf{Eficiencia de código}:

	\begin{equation*}
		\eta = 1 - R = \frac{r_{fuente}}{r_{canal}}
	\end{equation*}
	
%\subsubsection{Regeneradores}

Los \textbf{regeneradores} se reparten a lo largo de la línea de transmisión para tratar cualquier perturbación. Están dispuestos a intervalos regulares y operan a nivel de bit.

%\subsubsection{Transmisión a 64 Kbps}

\textbf{Transmisión a 64 Kbps codireccional} se utiliza un par por cada sentido de la transmisión con código 1B4T. \textbf{Transmisión a 64 Kbps contradireccional}, las señales de temporización se envían desde un mismo extremo en los dos sentidos (TX/RX), dos pares (uno para los datos y otro para la temporización por dirección, código AMI (100\% datos / 50\% reloj.

%\subsubsection{ADPCM}

Existen otros métodos de codificación de la señal de voz. En \textbf{DPCM} (\textit{Differential Pulse Code Modulation}) la base es codificar la diferencia entre muestras consecutivas. Las ventajas de este codificador estriban en la gran correlación entre muestras consecutivas de la señal vocal. En los decodificadores DPCM la diferencia se codifica con 4 bits siendo la calidad inferior a PCM. 

El codificador \textbf{ADPCM} (\textit{Adaptative Differential Pulse Code Modulation}) funciona a varias velocidades ($16-32 kbps$). Completa el codificador DPCM incluyendo un bloque de predicción de la muestra actual. La predicción se compara con la muestra actual $\rightarrow$ error de predicción. El error de predicción se cuantifica y envía. Normalmente, la entrada al codificador es una señal MIC (codificada con Ley A o Ley $\mu$ por lo que previamente hay que convertir la codificación logarítmica en codificación uniforme).

\textbf{ADPCM G.721}
	
%	\begin{center}
%		\includegraphics[scale=0.4]{images/ADPCM1}
%	\end{center}
	
	\begin{figure}[!ht]
 		\centering
  		 \includegraphics[scale = 0.4]{images/ADPCM1}
		\caption{CodificadorADPCM G.721}
	\end{figure}	
	
	\begin{enumerate}
		\item Obtiene el error de predicción
		\item Cuantificación con 15 niveles (4 bits) $\rightarrow$ señal $32 kbps$
		\item Se obtiene el error de predicción
		\item Este error se combina con la versión reconstruida de la señal de entrada
		\item Ambos se introducen al predictor adaptativo que obtiene una estimación de la señal de entrada
		\item La estimación se usa para obtener el error de predicción
	\end{enumerate}
	
%	\begin{center}
%		\includegraphics[scale=0.4]{images/ADPCM2}
%	\end{center}

	\begin{figure}[!ht]
 		\centering
  		 \includegraphics[scale = 0.4]{images/ADPCM2}
		\caption{Enlace Decodificador ADPCM G.721}
	\end{figure}
	
Los bloques son análogos a los equivalentes en el codificador con la salvedad del ajuste de codificación síncrono. Este bloque intenta eliminar la distorsión acumulativa que se produce en enlaces en tándem ADPCM-PCM-ADPCM síncronos.	
	
\hrulefill

\subsection{Redes de Fibra}

El acceso a través de fibra óptica es una alternativa más. Existen varias aproximaciones que se basan en criterios económicos y de necesidad de ancho de banda. FTT, Fiber to the X (cabinet, curb, building, home...). En su mayoría es una red óptica pasiva. 

Analogía: Acceso con cable de pares
	
%	\begin{center}
%		\includegraphics[scale=0.3]{images/Analogia}
%	\end{center}

	\begin{figure}[!ht]
 		\centering
  		 \includegraphics[scale = 0.25]{images/Analogia}
		\caption{Analogía del acceso con cable de pares con la fibra}
	\end{figure}	

Escenarios posibles de FTTx:

	\begin{itemize}
		\item \textbf{Sistemas basados en la Jerarquía Digital Síncrona (SDH)}: Anillos de acceso (DLCs o unidades remotas de usuario provistas con equipos de ADM), anillos de acceso NGDLC (New Generation Digital Loop Carriers) con gran número de usuarios atendidos, necesitan poco control por la central, ofrecen gran cantidad de servicios e integran DLCs y DSLAM (separación de los flujos de datos). El tramo final sigue siendo en la mayoría cable de pares.
		\item \textbf{Sistemas basados en la Jerarquía Digital Plesiócrona}
		\item \textbf{SDH hasta los negocios}
	\end{itemize}

%PDH sobre fibra hasta el negocio

%	\begin{center}
%		\includegraphics[scale=0.2]{images/PDHSobreFibra}
%	\end{center}

	\begin{figure}[!ht]
 		\centering
  		 \includegraphics[scale = 0.4]{images/PDHSobreFibra}
		\caption{PDH sobre fibra hasta el negocio}
	\end{figure}

%SDH hasta el negocio:

%	\begin{center}
%		\includegraphics[scale=0.2]{images/SDHNegocio}
%	\end{center}

	\begin{figure}[!ht]
 		\centering
  		 \includegraphics[scale = 0.4]{images/SDHNegocio}
		\caption{SDH hasta el negocio}
	\end{figure}
	
%\subsubsection{ATM Passive Optical Networks (APON)}

\textbf{ATM Passive Optical Networks (APON)}.Sistemas basados en ATM (\textit{Asynchronous Transfer Mode}), la distancia máxima depende del número de salidas de distribución (limitado por atenuación), el máximo número de salidas del \textit{splitter} es 64. Habitualmente en el sentido descendente (difusión), para el sentido ascendente TDM con sincronismo complejo (sistema de gestión específico) pero tiene problemas para ecualizar los tiempos de cada usuario.
	
%	\begin{center}
%		\includegraphics[scale=0.2]{images/APON}
%	\end{center}

	\begin{figure}[!ht]
 		\centering
  		 \includegraphics[scale = 0.35]{images/APON}
		\caption{Passive Optical Networks (APON)}
	\end{figure}

\hrulefill

\subsection{HFC}

Las \textbf{redes de cable} nacen ante la necesidad de resolver la mala calidad de las señales recibidas en poblaciones alejadas de los grandes núcleos urbanos. La arquitectura típica era árbol-rama. A partir de los años 80 a parecen las redes \textbf{HFC} (\textit{Hybrid Iber Coax}) que son redes de telecomunicaciones por cable que combinan la fibra óptica y el cable coaxial como soporte de transmisión de las señales. Se compone de cuatro partes: cabecera, red troncal, red de distribución y red de acometida a los abonados. Es bidireccional, sentido descendente del operador al usuario y ascendente del usuario al operador. En un primer tramo, las fibras ópticas llevan la información desde la cabecera hasta unos nodos finales donde se realiza la conversión óptico/eléctrico con una topología de estrella y a partir de los cuales se despliega la típica red de coaxial con estructura de árbol-rama hasta los usuarios.

	\begin{figure}[!ht]
 		\centering
  		 \includegraphics[scale = 0.35]{images/TopologiaHFC}
		\caption{Topología de una red HFC}
	\end{figure}	

La red de fibra tiene una topología en estrella, los anillos son sólo un método para introducir redundancia (una fibra de subida y otra de bajada por cada lado del anillo) y conseguir una red más robusta ante averías. Entre cada nodo primario y nodo final hay 4 fibras dedicadas (2 de subida y 2 de bajada).

La topología HFC tiene varias \textbf{ventajas}, como la mejora de la calidad gracias a que con la fibra disminuyen las interferencias y el número de amplificadores. Lo último también mejora la fiabilidad. Necesita menor alimentación, proporcionando mayor sencillez y menor coste. Aumenta la capacidad por acercar las fibras dedicadas a los usuarios. Permite dirigir algunos servicios por la parte en la estrella de la topología frente a la difusión de las redes antiguas. La tendencia es acercar más la fibra al usuario: FTTF (Feeder) $\rightarrow$ FTTC (Curb) $\rightarrow$ FTTB (Building) $\rightarrow$ FTTH (Home).\\

	\begin{figure}[!ht]
 		\centering
  		 \includegraphics[scale = 0.4]{images/Euskaltel}
		\caption{Red HFC de Euskaltel}
	\end{figure}	

La \textbf{red HFC de Euskaltel} responde a la estructura de una red FTTC, es bidireccional y comparte la misma infraestructura que la red de transporte de Euskaltel destinada a los servicios de telefonía y datos. Se compone de: cabecera, red de contribución, red troncal, red de distribución y red de acometida de los abonados.\\

%	\begin{center}
%		\includegraphics[scale=0.4]{images/Euskaltel}
%	\end{center}
	
La \textbf{red troncal}, para que la red no tenga degradaciones importantes la red ha de ser de fibra óptica. Realiza el transporte de las señales desde las cabeceras territoriales hasta los nodos finales a través de los tres niveles de red, habiendo rutas, equipamiento y fibras reduntantes. La red troncal primaria está constituida por anillos de fibra óptica que enlaza cada cabecera territorial con los nodos primarios de su provincia. La red troncal secundaria está constituida por anillos de fibra óptica que unen los nodos primarios con los secundarios. La red troncal terciaria está formada por anillos de fibra óptica que unen los nodos ópticos entre si y con los nodos secundarios. Los \textbf{elementos} de la red troncal son:

	\begin{center}
		\includegraphics[scale=0.4]{images/RedTroncal}
	\end{center}
	
Elementos de la \textbf{red de distribución} (cable coaxial):

	\begin{center}
		\includegraphics[scale=0.4]{images/RedDistribucion}
	\end{center}

La \textbf{red de abonado} une el tap situado en la entrada de los edificios con la toma final en la vivienda del usuario.

%	\begin{center}
%		\includegraphics[scale=0.4]{images/Esquema}
%	\end{center}
	
	\begin{figure}[!ht]
 		\centering
  		 \includegraphics[scale = 0.4]{images/Esquema}
		\caption{Esquema de Red, parte descendente}
	\end{figure}	

\hrulefill

\subsection{Acceso Via Radio}

Soluciona los problemas de realizar obras para instalar una red de bucle de abonado. Es de bajo coste, rápido despliegue, accesible en zonas remotas, no necesita gran inversión inicial, el crecimiento es adaptado a la demanda, bajos costes de mantenimiento, retorno de inversión rápido...

\subsubsection{Características de los sistemas PMP}

Los sistemas \textbf{PMP} (\textit{Point to Multi-Point}) tienen esctructura celular, optimiza el espectro y necesita baja potencia. Establece la conexión entre subscriptores y central a través de radioenlaces digitales bidireccionale (necesita visibilidad radioeléctrica entre el cliente y la estación base de red).\\

Algunos sistemas tienen \textbf{estructura} celular, éstas dependen de la orografía y de la densidad de clientes. Se reutilizan frecuencias dentro de una celda a otra. La utilización de polarizaciones diferentes permite optimizar más aún el espectro asignado.

%	\begin{center}
%		\includegraphics[scale=0.2]{images/Espectro}
%	\end{center}

	\begin{figure}[!ht]
 		\centering
  		 \includegraphics[scale = 0.4]{images/Espectro}
		\caption{Antena sectorial}
	\end{figure}	

%\subsubsection{Eficiencia}

Para maximizar la \textbf{eficiencia} se utilizan asignación dinámica de anchos de banda. Consiste en gestionar el espectro dinámicamente para asignar recursos a los usuarios. Permite comportir parte del espectro entre varios clientes según sus necesidades. Generalmente replican sistemas de conmutación ATM o Frame Relay.

	\begin{center}
		\includegraphics[scale=0.3]{images/Eficiencia}
	\end{center}
	
%\subsubsection{Modulaciones y Ancho de Banda}

La \textbf{capacidad} de estos sistemas depende del ancho de banda disponible por el operador y la eficiencia espectral de la modulación empleada. Factor de utilización de frecuencias (doble uso de la banda asignada dentro de una celda con polarizaciones cruzadas), factor de compartición del espectro entre usuarios.\\

%\subsubsection{MMDS}

\textbf{MMDS} (\textit{Multichannel Multipoint Distribution System}) fue pensado originalmente como un servicio de difusión. Opera en la banda de los $3.5 GHz$, las versiones posteriores son digitales con posibilidad de retorno, necesita visibilidad entre emisor-receptor, es sensible al multitrayecto, no se ve atenuado por la lluvia.

%\subsubsection{LMDS}

\textbf{LMDS} (\textit{Local Multipoint Distribution System}) utiliza la banda de frecuencias entre $24-40 GHz$, emplea la modulación QPSK o 16 QAM. Sistema celular basado en células pequeñas, emplea transmisores de baja potencia, afectado por la atenuación atmosférica y en ocasiones no es necesario visión directa debido a las reflexiones.


%%%%%%%%%%%%%%%%%%%%%%%%%%%%%%%%%%%%%%%%%%%%%%%%%%%%%%%%%%%%%%%%%
% Tema 5

\newpage

\hrulefill

\section{5. Redes Troncales y de Transporte}

\hrulefill

\subsection{PDH}

La tecnología \textbf{PDH} (\textit{Plesiochronous \footnote{Forma de sincronización en una red digital en la que los equipos se sincronizan mediante relojes separados de similar precisión.} Digital Hierarchy}) permite multiplexar en tramas de orden superior afluentes MIC 30+2. Optimiza los recursos en los enlaces transmitiendo el mayor número de canales vocales posibles con el mínimo de recursos. Al ser sistemas digitales, la multiplexación es TDM. PDH consiste en una red en la que los relojes con ``casi'' síncronos, tendrán la misma velocidad pero estarán sujetos a una tolerancia. Existen varios estándares. Es coherente con la diferencia en el proceso de creación de la trama básica MIC.\\

	\begin{figure}[!ht]
 		\centering
  		 \includegraphics[scale = 0.3]{images/Jerarquia}
		\caption{Jerarquías Digitales}
	\end{figure}

Las jerarquías digitales son la herramienta de transporte de las redes telefónicas digitales.

%	\begin{center}
%		\includegraphics[scale=0.2]{images/Jerarquia}
%	\end{center}
	
La UIT especifica los niveles de multiplexación posible y la estructura de las tramas multiplexadas.

%	\begin{center}
%		\includegraphics[scale=0.2]{images/NivelesMUX}
%	\end{center}
	
	\begin{figure}[!ht]
 		\centering
  		 \includegraphics[scale = 0.3]{images/NivelesMUX}
		\caption{Niveles de multiplexación}
	\end{figure}

Las señales de entrada a un multiplexor son \textbf{tributarias} o \textbf{afluentes}. La multiplexación se realiza a nivel del bit.

%\subsubsection{Velocidades PDH}

	\begin{figure}[!ht]
 		\centering
  		 \includegraphics[scale = 0.35]{images/VelocidadPDH}
		\caption{Velocidades PDH}
	\end{figure}

%	\begin{center}
%		\includegraphics[scale=0.2]{images/CaracteristicasPDH}
%	\end{center}	

%\subsubsection{Justificación o Relleno}

Los equipos de la red PDH no están sincronizados, utilizan distintos relojes con la misma velocidad pero con tolerancias sobre la velocidad nominal. Esto supone un problema para la multiplexación. A cada señal tributaria se le añaden unos bits que son de \textbf{relleno} o \textbf{justificación} para que el extremo receptor pueda distinguir los bits que son de información y los de relleno. Con esto se consiguen salvar las diferencias de frecuencia que puedan tener los distintos tributarios. A los tributarios \footnote{Número $N$ de señales numéricas que recibe un equipo multiplicador digital} más lentos es necesario añadirles más bits de relleno que a los rápido.

	\begin{center}
		\includegraphics[scale=0.35]{images/JustificacionPDH}
	\end{center}	

Las tramas contemplan espacios para bits de relleno que se pueden utilizar como bits de información si es necesario $\rightarrow$ Justificación positiva
	
%	\begin{center}
%		\includegraphics[scale=0.4]{images/Relleno}
%	\end{center}	
	
	\begin{figure}[!ht]
 		\centering
  		 \includegraphics[scale = 0.35]{images/Relleno}
		\caption{Justificación de señales plesiócronas}
	\end{figure}
	

La técnica de justificación tiene tres variantes: 

	\begin{itemize}
		\item Positiva: Las tramas contemplan espacios para bits de relleno que se pueden utilizar como bits de información si es necesario.
		\item Negativa: Supresión de impulsos sobre los afluentes
		\item Positiva-Nula-Negativa: Utilizado en SDH
	\end{itemize}
	
Si se ha utilizado justificación en la trama es necesario indicarlo, para ello hay bits de control de justificación sobre qué afluente se ha realizado.

%\subsubsection{Estructura de la trama G.742 - 8 Mbps}
	
%	\begin{center}
%		\includegraphics[scale=0.3]{images/TramaG742}
%	\end{center}	

	\begin{figure}[!ht]
 		\centering
  		 \includegraphics[scale = 0.3]{images/TramaG742}
		\caption{Estructura de la trama G.742 - 8 Mbps}
	\end{figure}
	
Al multiplexar 4 tramas de 2 Mbps el resultado no es $8192 = 4 \cdot 1024$ sino más ya que se añade información extra. Cada trama tiene longitud diferente ya que la multiplexación se realiza a nivel de bit.
	
%\subsubsection{Estructura de las tramas G.751 - 34 Mbps}

%	\begin{center}
%		\includegraphics[scale=0.3]{images/TramaG751}
%	\end{center}
	
	\begin{figure}[!ht]
 		\centering
  		 \includegraphics[scale = 0.3]{images/TramaG751}
		\caption{Estructura de la trama G.751 - 34 Mbps}
	\end{figure}
	
%\subsubsection{Estructura de las tramas G.751 - 140 Mbps}

%	\begin{center}
%		\includegraphics[scale=0.3]{images/TramaG7512}
%	\end{center}	
	
	\begin{figure}[!ht]
 		\centering
  		 \includegraphics[scale = 0.3]{images/TramaG7512}
		\caption{Estructura de la trama G.751 - 140 Mbps}
	\end{figure}	

\textbf{Velocidades de trama de señales PDH}

	\begin{itemize}
		\item \textbf{Tasa Nominal de Justificación}: $\theta = \frac{F_{J}}{F_{A}}$\\
				\quad $F_{A}$: Velocidad de justificación cuando el afluente llega a velocidad nominal\\
				\quad $F_{J}$: Frecuencia Nominal de Justificación
		\item \textbf{Frecuencia de Redundancia}\\
			\quad Frecuencia a la que se insertan los bits de control \\
			\quad Tasa de redundancia $R$
		\item \textbf{Relación Nominal de Justificación}: Relación entre la velocidad nominal de justificación  y la velocidad máxima de justificación posible. $g = \frac{F_{J}}{F_{J}^{MAX}}$
		\item \textbf{Frecuencia del Múltiplex}: $F_{M} = N \cdot F_{A} * (1 + \theta) (1 + R)$
	\end{itemize}


Existen distintos \textbf{tipos de equipos en redes PDH}

	\begin{itemize}
		\item Terminales de línea: Elécricos, radio y ópticos
		\item Multiplexores
		\item Multiplexores de acceso
		\item Multiplexores de Inserción y Extracción
		\item Cross-Connect
		\item Equipos integrados
	\end{itemize}

Tipos de conexiones:

	\begin{itemize}
	\item Punto a punto
		\begin{center}
			\includegraphics[scale=0.2]{images/P2P}
		\end{center}
	\item Punto a Multipunto
		\begin{center}
			\includegraphics[scale=0.2]{images/P2M}
		\end{center}
	\item Punto a Multipunto
		\begin{center}
			\includegraphics[scale=0.2]{images/Uni}
		\end{center}
	\item Protección de circuitos: sistemas propietarios
	\end{itemize}

\textbf{Cross connect}

	\begin{center}
			\includegraphics[scale=0.2]{images/Cross}
		\end{center}

\hrulefill

\subsection{SDH}

El antecedente a las redes \textbf{SDH} (Sinchronous Digital Hierarchy) son los sistemas PDH. Multiplexación de sistemas MIC 30+2, creación de múltiplex de orden superior mediante multiplexación ``casi'' síncrona, los equipos utilizan relojes no sincronizados $\rightarrow$ para poder realizar correctamente la multiplexación se requiere de técnicas de justificación.\\

Las redes PDH tienen una serie de \textbf{problemas}: \textbf{No existe compatibilidad en los estándares utilizados en diferentes países} y los equipos de adaptación son costosos, \textbf{las capacidades máximas son limitadas}, \textbf{problemas para extraer flujos} ya que para extraer un flujo de nivel inferior directamente de un flujo de nivel superior hay que demultiplexar todos los flujos, \textbf{no está pensada para el uso de fibra óptica como medio de transmisión}, \textbf{carece de herramientas de gestión y no prevé sistemas de tolerancia a fallos en los enlaces} limitando el ancho de banda disponible para la señalización y la gestión de la red siendo los sistemas de gestión y mantenimiento propietarios.

%	\begin{center}
%		\includegraphics[scale=0.2]{images/Extraccion}
%	\end{center}

	\begin{figure}[!ht]
 		\centering
  		 \includegraphics[scale = 0.3]{images/Extraccion}
		\caption{Extracción de una tributaria de 2 Mbps en un múltiplex de 139 Mbps}
	\end{figure}	

Los problemas de PDH los resuelve SDH. Da lugar a un estándar ANSI que se denominó SONET (\textit{Synchronous Optical NETwork}). SDH compatibiliza los afluentes PDH y pretende relevarlo al uso de enlaces de menor capacidad, permite velocidades de hasta $10 Gbps$, permite una extracción sencilla de flujos inferiores, compatible con sistemas de diferentes continentes (SONET $\rightarrow$ SDH), crea un reloj de referencia estándar para toda la red

%	\begin{center}
%		\includegraphics[scale=0.2]{images/AddDrop}
%	\end{center}

	\begin{figure}[!ht]
 		\centering
  		 \includegraphics[scale = 0.3]{images/AddDrop}
		\caption{Add and Drop Multiplexers}
	\end{figure}

ADM se encarga de extraer o insertar señales tributarias plesiócronas o síncronas de cualquiera de las dos señales agregadas que recibe (una en cada sentido de la transmisión), .

%\subsubsection{Velocidades SDH-SONET}


%	\begin{center}
%		\includegraphics[scale=0.2]{images/EjemploTrans}
%	\end{center}
	
	\begin{figure}[!ht]
 		\centering
  		 \includegraphics[scale = 0.3]{images/EjemploTrans}
		\caption{Ejemplo de transporte PDH sobre SONET/SDH}
	\end{figure}

Equipos en las redes SDH:

\begin{itemize}
	\item \textbf{Regeneradores}: Regeneradores de señal utilizados en enlaces de larga distancia, donde las perturbaciones del medio de transmisión limitan el alcance del enlace.
	\item \textbf{Multiplexores terminales}: Permiten combinar señales de entrada PDH o SDH en señales STM-N de mayor velocidad.
	\item \textbf{Multiplexores ADM} (Add and Drop Multiplexer): Permiten intercalar o extraer tramas de un determinado múltiplex. Permiten la formación de anillos.
	\item \textbf{Digital Cross-Connect}: Permiten la interconexión de diferentes secciones o anillos de la red SDH, tienen varias entradas y varias salidas.
\end{itemize}

\textbf{Topologías de Red}

\begin{itemize}
	\item Punto a Punto
		\begin{center}
			\includegraphics[scale=0.2]{images/SDHP2P}
		\end{center}
	\item Lineal / Punto a Multipunto
		\begin{center}
			\includegraphics[scale=0.2]{images/SDHMulti}
		\end{center}
	\item Anillo
		\begin{center}
			\includegraphics[scale=0.2]{images/SDHAnillo}
		\end{center}
	\item Anillo: Comunicación full duplex con una fibra
		\begin{center}
			\includegraphics[scale=0.2]{images/SDHAnillo2}
		\end{center}
	\item Anillo múltiple
		\begin{center}
			\includegraphics[scale=0.2]{images/AnilloMultiple}
		\end{center}
\end{itemize}

	\begin{figure}[!ht]
 		\centering
  		 \includegraphics[scale = 0.4]{images/AveriaSDH}
		\caption{Funcionamiento normal y avería en una red SDH}
	\end{figure}
	
	\begin{figure}[!ht]
 		\centering
  		 \includegraphics[scale = 0.4]{images/Enlace}
		\caption{Enlaces en una red SONET-SDH}
	\end{figure}

	\begin{itemize}
		\item Sección: Unión directa entre dos equipos cualesquiera
		\item Línea: Unión entre dos ADMs contiguos
		\item Ruta: Unión entre dos equipos finales
	\end{itemize}
	
	\begin{figure}[!ht]
 		\centering
  		 \includegraphics[scale = 0.4]{images/Arquitectura}
		\caption{Arquitectura de Capas de SONET-SDH}
	\end{figure}
	
SONET/SDH divide la capa física en cuatro subcapas:

	\begin{itemize}
		\item Fotónica: transmisión de la señal y las fibras
		\item Sección: Interconexión de equipos contiguos
		\item Línte: Multiplexación/Demultiplexación de circuitos entre dos ADMs
		\item Rutas: Comunicación extremo a extremo
	\end{itemize}
	
El bloque básico de la construcción de SDH es el módulo STM-1 \footnote{Synchronous Transport Module}, consiste en una matriz de $9x270 bytes$. La transmisión se realiza por las filas de arriba a abajo y de izquierda a derecha, cada trama se envía en 125$\mu$s. A partir de este elemento básico se construyen por multiplexación síncrona byte a byte las velocidades (múltiplex) de orden superior STM-N.
	
	\begin{figure}[!ht]
 		\centering
  		 \includegraphics[scale = 0.4]{images/TramaSMT}
		\caption{Estructura de la trama STM-1}
	\end{figure}
	
\textbf{SOH} (\textit{Section Over-Head}) incluye: datos para la alineación de trama, chequeo de paridad, bytes para transportar datos, señalización y punteros. La carga útil (tráfico) contendrá los datos apropiadamente dichos.

	\begin{figure}[!ht]
 		\centering
  		 \includegraphics[scale = 0.4]{images/STM4}
		\caption{Estructura de la trama STM-4}
	\end{figure}
	
La tara de sección no se multiplexa, se vuelve a generar. La trama STM-1 se multiplexa byte a byte.
	
	\begin{figure}[!ht]
 		\centering
  		 \includegraphics[scale = 0.4]{images/STM16}
		\caption{Estructura de la trama STM-16}
	\end{figure}
		
El multiplexado de orden STM-N se realiza por bloques de $N$ bytes.\\

Para tramas de orden superior, \textbf{STM-N} se admiten afluentes cuya carga (bitrate) no exceda la capacidad de carga útil del módulo de transporte STM-1. Esta estructura limita a los afluentes a una velocidad máxima de 155 Mbps, no es suficiente para algunas aplicaciones. Para resolver este problema se prevee \textbf{STM-Nc}, STM-N concatenados. La estructura es idéntica a un STM-N con la salvedad de un flag que indica que dichas tramas no han de ser demultiplexadas en un múltiplex de STM-M, $m < N$.
	
	\begin{figure}[!ht]
 		\centering
  		 \includegraphics[scale = 0.5]{images/SOH}
		\caption{Tara de Sección SOH}
	\end{figure}
	
\textbf{Tara de sección de regeneración} se desensambla, evalúa y regenera en cada regenerador. \textbf{Tara de sección de multiplexión} se mantiene intacta hasta el multiplexor por donde se desensambla con la carga útil.
	
	\begin{itemize}
		\item A1-A2: Alineamiento de trama
		\item B1-B2: Monitorización de la calidad, bytes de paridad
		\item D1-D3: Canal de gestión de red
		\item D4-D12: Canal de gestión de red
		\item E1-E2: Telefonía de servicio
		\item F1: Mantenimiento
		\item J0 (C1): Identificador de tramo
		\item K1-K2: Control de la conmutación de automática de protección
		\item S1: Indicador de la calidad de reloj empleado en la transmisión
		\item M1: Reconocimiento de errores de transmisión. Detector en el extremo distante.
		\item X: Reservado para uso futuro
	\end{itemize}
	
		\begin{figure}[!ht]
 		\centering
  		 \includegraphics[scale = 0.4]{images/VC}
		\caption{Contenedores Virtuales y Punteros}
	\end{figure}
	
	
Para ser transportados en tramas STM-1, los datos se introducen en \textbf{contenedores virtuales}. Los contenedores se introducen en la carga útil del módulo STM-1 y son transportados a $8 \cdot 10^{3}$ tramas por segundo. Los contenedores no tienen una posición fija definida en la trama STM-1, para poder transmitirse correctamente en la tara de sección existe un puntero que indica el comienzo de un contenedor.
	
	\begin{figure}[!ht]
 		\centering
  		 \includegraphics[scale = 0.4]{images/VCPointer}
		\caption{Add/Drop de Contenedores Virtuales}
	\end{figure}
	
	\begin{figure}[!ht]
 		\centering
  		 \includegraphics[scale = 0.4]{images/VC2}
		\caption{Valores de los punteros dentro de un contenedor virtual}
	\end{figure}
	
	\begin{figure}[!ht]
 		\centering
  		 \includegraphics[scale = 0.4]{images/VCPosit}
		\caption{Bits de justificación positiva en un contenedor virtual}
	\end{figure}
	
	\begin{figure}[!ht]
 		\centering
  		 \includegraphics[scale = 0.4]{images/VCNeg}
		\caption{Bits de justificación negativa en un contenedor virtual}
	\end{figure}
	
Los contenedores virtuales están formados por dos unidades administrativas: tara de trayecto y un contenedor. El conjunto de un contenedor virtual con el puntero indica su posición en la trama STM-1 forman una unidad administrativa. Este proceso permite transportar una señal PDH de 139 Mbps.

	\begin{figure}[!ht]
 		\centering
  		 \includegraphics[scale = 0.4]{images/VC3}
		\caption{Transporte señal PDH 139 Mbps}
	\end{figure}

	\begin{figure}[!ht]
 		\centering
  		 \includegraphics[scale = 0.4]{images/Resumen}
		\caption{Resumen}
	\end{figure}
	
	\begin{figure}[!ht]
 		\centering
  		 \includegraphics[scale = 0.4]{images/Contenedores}
		\caption{Tipos de contenedores}
	\end{figure}

Hay varios \textbf{tipos de contenedores}, si a los contenedores anteriores se les añade la tara de trayecto el resultado es el contenedor virtual:

	\begin{itemize}
		\item Contenedores de orden superior: V-4 y V-3
		\item Contenedores de orden inferior: V2, V-1 y V11
	\end{itemize}

La tara de trayecto es diferente en función del tipo de contenedor, para los contenedores de orden superior es de 9 bytes (una columna), para los de orden inferior es de un solo byte ya que se usan más bytes como señalización de un conjunto de tres contenedores de orden inferior.

	\begin{figure}[!ht]
 		\centering
  		 \includegraphics[scale = 0.4]{images/POH}
		\caption{Tara de trayecto}
	\end{figure}
	
Además de poder insertar una señal de 140 Mbps en un STM-1 a través de un contenedor virtual de orden 4 se pueden \textbf{multiplexar otros tipos de contenedores} a través de:

	\begin{itemize}
		\item Unidades tributarias
		\item Grupos de unidades tributarias, multiplexando byte a byte unidades tributarias
	\end{itemize}
	
	\begin{figure}[!ht]
 		\centering
  		 \includegraphics[scale = 0.4]{images/VCMux}
		\caption{Multiplexación VC-4 y VC-3}
	\end{figure}



	\begin{figure}[!ht]
 		\centering
  		 \includegraphics[scale = 0.4]{images/PunterosMux}
		\caption{Punteros en la multiplexación VC-4 y VC-3}
	\end{figure}
	
	\begin{figure}[!ht]
 		\centering
  		 \includegraphics[scale = 0.4]{images/VC2Mux}
		\caption{Multiplexación VC-2}
	\end{figure}
	
	\begin{figure}[!ht]
 		\centering
  		 \includegraphics[scale = 0.4]{images/VC12Mux}
		\caption{Multiplexación VC-12}
	\end{figure}
	
\textbf{Mapeado de señales PDH en contenedores}, detalle de una fila del C-4 (20 bloques de 13 bytes)

	\begin{figure}[!ht]
 		\centering
  		 \includegraphics[scale = 0.4]{images/PDHCont}
		\caption{Bits de justificación negativa en un contenedor virtual}
	\end{figure}
	
	\begin{figure}[!ht]
 		\centering
  		 \includegraphics[scale = 0.4]{images/Mapeado}
		\caption{Mapeado Asíncrono de señales de 2 Mbps}
	\end{figure}
	
	\begin{figure}[!ht]
 		\centering
  		 \includegraphics[scale = 0.4]{images/Mapeado2}
		\caption{Mapeado Síncrono de señales de 2 Mbps}
	\end{figure}

\textbf{Funciones de gestión de red en SDH}:

	\begin{itemize}
		\item Anomalías: Errores de paridad e indicación de error remoto
		\item Defectos: Persistentes en el tiempo, de señal, de alineamiento de trama, de puntero...
		\item Fallos, incapacidad de una función de conseguir realizar una acción en un determinado límite de tiempo
		\item Alarmas: Señales de mantenimiento que permiten avisar en el trayecto a otros equipos de alguna anomalía, fallo o defecto
	\end{itemize}
	
	\begin{figure}[!ht]
 		\centering
  		 \includegraphics[scale = 0.4]{images/Defectos}
		\caption{Mapeado Síncrono de señales de 2 Mbps}
	\end{figure}

Existen \textbf{tipos de protección en SDH}, permite diseñar redes preparadas para la recuperación automática frente a fallos. Las taras permiten el envío de alarmas y mensajes de error que facilitan esa tarea. Hay varios tipos de protecciones:

	\begin{itemize}
		\item Protección de la Sección de Multiplexación (MSP)
		\item Protección de la Conexión de Subred (SNCP)
		\item Protección Add and Drop
	\end{itemize}
	
La protección MSP (protección de sección de multiplexación) tiene dos tipos: $1+1$ y $m:n$.

	\begin{figure}[!ht]
 		\centering
  		 \includegraphics[scale = 0.4]{images/MSP}
		\caption{Protección MSP Lineal}
	\end{figure}
	
	\begin{figure}[!ht]
 		\centering
  		 \includegraphics[scale = 0.4]{images/MS}
		\caption{Protección MSP Dedicada (MS-SPRING}
	\end{figure}
	
La \textbf{protección SNCP} es similar a la $1+1$ pero a nivel de camino de contenedor virtual.

	\begin{figure}[!ht]
 		\centering
  		 \includegraphics[scale = 0.4]{images/SNCP}
		\caption{Protección SNCP}
	\end{figure}
	
La \textbf{protección Drop and Continue} se utiliza para conectar subredes.

	\begin{figure}[!ht]
 		\centering
  		 \includegraphics[scale = 0.4]{images/Drop}
		\caption{Protección Drop and Continue}
	\end{figure}

	\begin{figure}[!ht]
 		\centering
  		 \includegraphics[scale = 0.4]{images/Sync}
		\caption{Sincronismo SDH}
	\end{figure}
	
	\begin{figure}[!ht]
 		\centering
  		 \includegraphics[scale = 0.4]{images/Final}
		\caption{Arquitectura final de un operador}
	\end{figure}

\hrulefill

\subsection{Señalización}

La señalización en una red telefónica es el lenguaje utilizado entre los terminales de abonado, centrales de conmutación y equipos de inteligencia de red para establecer una comunicación, proveer los medios de control para ofrecer servicios suplementarios, proveer mecanismos para realizar consultas a bases de datos... \textbf{Es la información que se intercambia entre terminales y equipos/nodos de la red para que los servicios de usuario puedan ser ofrecidos de forma satisfactoria}

\textbf{Tipos de Señalización}

	\begin{itemize}
		\item \textbf{Señalización de abonado}: Abonado-Central y Central-Abonado. Tiene que ser sencilla para permitir la retrocompatibilidad y permitir terminales sencillos y baratos.
		\item \textbf{Señalización interna de central}: Es dependiente del fabricante. Envía información de señales y control entre los equipos de la central. La señalización en la red fue en sus comienzos una esxtensión de este tipo de señalización.
		\item \textbf{Señalización entre centrales}: Elige los enlaces que se van a utilizar para la llamada y se encarga de liberarlos una vez se termine. Antes, consulta el \textbf{estado de enlace} para comprobar si está disponible. Hay que enviar las llamadas desde un extremo de la red hasta otro mediante el \textbf{rutado de llamadas}. \textbf{Envío de numeración entre centrales}. La información se envía \textbf{enlace a enlace}. \textbf{Extremo a extremo}, un nodo maestro dirige la señalización, conoce la red y va a organizar la comunicación, permite enviar menos información por la red.
		\item \textbf{Señalización por canal asociado}: Cada canal vocal tiene asociado un canal de comunicación propio para la señalización. Típica de los sistemas analógicos. Este canal de señalización está asignado durante la comunicación. Normalmente, un canal físico de señalización está asociado al canal vocal. El canal físico (medio de transmisión) se comparte por los usuarios pero aún así sigue siendo señalización por canal asociado. Conocido como CAS. El equipo de transmisión está activo or una mínima parte del tiempo total de utilización del canal. El equipo receptor está funcionando permanentemente pero su actividad es mínima. No es posible la señalización extremo a extremo sin establecer un circuito. Por la eficiencia, no es posible consultar bases de datos porque solo hay un canal de señalización activo con un canal de voz activo.
	\end{itemize}
	
	\begin{figure}[!ht]
 		\centering
  		 \includegraphics[scale = 0.3]{images/SenalizacionAsociado}
		\caption{Señalización por canal asociado}
	\end{figure}

\textbf{Señalización por canal común}: Un único canal de señalización que es compartido por todos los usuarios. La señalización de cada usuario (canal vocal) se envía cuando se necesita. En algunos casos el canal físico de señalización y el de tráfico están separados pero no tiene por qué ser así. Conocido como CCS. Al suprimir los terminales de señalización se ahorra en los enlaces. Aumenta el vocabulario de señalización implementando más servicios. Incremento en velocidad en el establecimiento de llamadas. Aumento en la fiabilidad mediante el empleo de métodos mas eficaces de detección y corrección de errores. Permite el acceso a bases de datos. La señalización extremo a extremo es posible antes de establecer una comunicación.
	
	\begin{figure}[!ht]
 		\centering
  		 \includegraphics[scale = 0.3]{images/SenalizacionComun}
		\caption{Señalización por Canal Común}
	\end{figure}

	\begin{figure}[!ht]
 		\centering
  		 \includegraphics[scale = 0.3]{images/CobroRevertido}
		\caption{Señalización por canal común y canal asociado en una llamada a cobro revertido automática}
	\end{figure}


\subsubsection{Evolución de los Sistemas de Señalización}

\textbf{CAS} previo a 1970 con señalización en banda. \textbf{CCIS} (\textit{Common Channel Interoffice Signaling}), se utiliza SS6, un sistema de canal común. \textbf{SS7}, origen en los años 1980, la señalización se transporta sobre redes separadas del tráfico de voz y tiene una pila de protocolos.


El \textbf{Sistema de Señalización Número 7 (SS7)} es el estándar de la UIT-T, diseñado específicamente para atender requerimientos avanzados de redes telefónicas digitales (RDSI). Basado en una red de datos independiente de la red de tráfico. Teniendo nodos, enlaces y protocolos propios. Su arquitectura es independiente de la(s) red(es) que soporta. 

\textbf{Arquitectura de la Red SS7} está compuesta por nodos y enlaces. Los nodos se conocen como Signaling Points y tiene un código (SPC) que lo identifica.

	\begin{center}
		\includegraphics[scale = 0.2]{SS7}
	\end{center}
	
	\begin{center}
		\includegraphics[scale = 0.2]{SP}
	\end{center}
	
	\begin{itemize}
		\item SSP (Service Switching Point): Genera y recibe mensajes de señalización.
		\item STP (Service Transfer Point): Conmutador de paquetes de señalización. No genera ni termina mensajes y físicamente puede ser el mismo equipo que un SSP o separado.
		\item SCP (Service Control Point): Nodos para aplicaciones.
	\end{itemize}

Los SCPs y STPs suelen estar redundados por motivos de seguridad y disponibilidad. La red de SSPs (enlaces de tráfico) puede tener cualquier topografía.

	\begin{center}
		\includegraphics[scale = 0.2]{RedSSP}
	\end{center}

	\begin{center}
		\includegraphics[scale = 0.2]{PlanoSS7}
	\end{center}

Una central puede tener varios SCPs (en varios planos)

	\begin{center}
		\includegraphics[scale = 0.2]{NivelesSS7}
	\end{center}

\textbf{Protocolos SS7}

	\begin{center}
		\includegraphics[scale = 0.3]{SS72}
	\end{center}

SS7 se diseña para minimizar los retardos en la señalización, para eso únicamente utiliza los niveles inferiores de OSI (Físico, Enlace y Red). Para gestionar los recursos de red se utiliza la señalización.
	
	\begin{figure}[!ht]
 		\centering
  		 \includegraphics[scale = 0.5]{images/SS7OSI}
		\caption{Relación entre OSI y los Protocolos SS7}
	\end{figure}

	\begin{itemize}
		\item \textbf{MTP} (\textit{Message Transfer Part}): Compuesta por los tres primeros niveles OSI. MTP Capa 1 (Enlace de datos de señalización $\rightarrow$ Capa Física), MTP Capa 2 (Enlace de señalización $\rightarrow$ CApa de Enlace), MTP Capa 3 (Funciones de la red de señalización: tratamiento de mensajes y gestión de red).
		\item \textbf{SCCP} (\textit{Signaling Connection Control Point}): Amplía las funcionalidades de las capas MTP y permite realizar comunicaciones orientadas a conexión: Fundamental para aplicaciones que no sean conmutación de circuitos.
	\end{itemize}

TCAP $\rightarrow$ Parte de la capa de aplicación: permite la ejecución de distintos procesos remotos:

	\begin{itemize}
		\item \textbf{TUP} (\textit{Telephone User Part}): En desuso, reemplazado por ISUP.
		\item \textbf{ISUP} (\textit{ISDN User Part}): Señalización para toma, mantenimiento y liberación de conexiones orientadas a conmutación de circuitos.
		\item \textbf{MAP} (\textit{Mobile User Part}): PAra transiciones entre centrales de la red de móviles y bases de datos.
		\item \textbf{INAP} (\textit{Intelligent Network Application Part}): Aplicaciones de inteligencia de red en redes fijas.
		\item \textbf{CAP} (\textit{CAMEL Application Part}): Extender funcionalidades de red inteligente en redes móviles donde MAP no es suficiente.
	\end{itemize}

\subsubsection{MTP 1}

Corresponde al nivel 1 OSI (físico), diseñado para ser utilizado en enlaces digitales full-duplex. Su velocidad típica de transmisión es 64 kbps pero permite superiores.

\subsubsection{MTP 2}

Corresponde al nivel 2 OSI (enlace), permite la comunicacion libre de errores entre dos SP unidos por un enlace. Como \textbf{funciones} permite: delimitación de mensajes mediante flags de inicio y fin, detección de errores, comprobación del orden de secuencia en los mensajes, ACKs y detección de fallo en los enlaces. Hay tres tipos de mensajes:

	\begin{itemize}
		\item \textbf{MSU} (\textit{Message Signal Unit}): Lleva la información de señalización (MTP 3).
		
			\begin{center}
				\includegraphics[scale = 0.3]{MSU}
			\end{center}
			
		\item \textbf{LSSU} (\textit{Link Signaling Signal Unit}): Lleva información sobre el estado del enlace de señalización.
		
			\begin{center}
				\includegraphics[scale = 0.35]{LSSU}
			\end{center}
		
		\item \textbf{FISU} (\textit{Fill-Inn Signal Unit}): Envían para monitorizar la calidad del enlace.
		
			\begin{center}
				\includegraphics[scale = 0.35]{FISU}
			\end{center}
 	\end{itemize}

\subsubsection{MTP 3}

Corresponde al nivel 3 OSI (red). Tiene las siguientes \textbf{funciones}: 

	\begin{itemize}
		\item Discriminación de mensajes (decide ruta o usuario final): Determina si un mensaje ha llegado a su nodo final. Examina el DPC: Si coincide con el SPC entrega el mensaje ala parte de distribución de mensajes, si no coincide lo entrega a la parte de encaminamiento.
		\item  Encaminamiento de mensajes (rutado): Utiliza la etiqueta de rutados y el SSF, el encaminamiento se basa en tablas de rutado, los encaminamiento son por defecto estáticos y en caso de fallo o congestión se utilizan enlaces alternativos.
		\item Distribución de mensajes (asignación a la parte del usuario final: Entrega el mensaje a la parte de usuario correspondiente. Utilizando MTP3 sólo son posibles 16 usuarios diferentes (4 bits).
		\item  gestión del tráfico de señalización y gestión de las rutas de señalización: Entre dos SP contiguos para advertir de que un enlace está congestionado o enfallo y activación y desactivación de enlaces.
		\item Gestión de las rutas de señalización: Informa a SP cercanos del estado del enlace (congestión).
	\end{itemize}

	\begin{center}
		\includegraphics[scale = 0.3]{MTP3E}
	\end{center}
	
	\begin{itemize}
		\item OPC: Origination Point Code
		\item DPC: Destination Point Code
		\item SLS: Bits para el rutado
		\item SI: Signaling Indicator (Indica el usuario final: ISUP, TUP, SCCP).
		\item SSF: Indica si el nodo es de nivel nacional, internacional o interno de un operador
	\end{itemize}
	
\textbf{Ejemplo}: Transmisión con MTP3
	
	\begin{center}
		\includegraphics[scale = 0.3]{MTP3Ej}
	\end{center}	
	
\subsubsection{SCCP}

\textit{Signaling Connection Control Part}. Solamente con MTP  no se aseguran transferencias de datos asociadas a conexión (necesarias en aplicaciones complejas de señalización), el espacio de direccionamiento de MTP es limitado. \textbf{Funcionalidades}:

	\begin{itemize}
		\item Permite dar servicio a partes de usuario de operación y mantenimiento.
		\item Completa MTP 3 hasta el nivel de red de la capa OSI e incluye algunas funciones de la capa 4 (transferencias orientadas a conexión).
		\item Ofrece servicios de red con o sin conexión
		\item Permite ampliar el espacio de direccionamiento de MTP3.
		\item Permite la señalización extremo a extremo no asociada a circuitos (activación remota del servicio de llamadas).
	\end{itemize}

El servicio se divide en varias \textbf{clases}:

	\begin{itemize}
		\item Clase 0: NOC, no garantiza el orden de entrega.
		\item Clase 1: NOC con garanatía de orden de entrega.
		\item Clase 2: OC básica.
		\item Clase 4: OC con control de flujo, detección de pérdida de tramas y retransmisión
	\end{itemize}
	
Se puede ampliar la capacidad de direccionamiento:

	\begin{itemize}
		\item SSN: Hasta 256 partes de usuario diferentes
		\item GTT: Global Title Translation
	\end{itemize}
	
	\begin{center}
		\includegraphics[scale = 0.3]{SCCP}
	\end{center}
	
\subsubsection{Partes de Usuario y Partes de Aplicación}

ISUP (\textit{ISDN User Part}) pensado para la red digital de servicios integrados RDSI. En la actualidad se utiliza en las redes de conmutación de circuitos que no son RDSI (RDI). Sus \textbf{funcionalidades} son: establecimiento, liberación y control de llamadas, voz RDSI, voz RTB y conexiones de datos e intercambio de mensajes extremo a extremo. Unos de los mensajes más importantes son:

	\begin{itemize}
		\item IAM (Initial Address Message)
		\item ACM (Address Complete Message)
		\item ANM (Answer Message)
		\item REL (Release)
		\item RLC (Release complete)
	\end{itemize}
	
\subsubsection{ISUP: Ejemplo de llamada en RDSI}
	
	\begin{center}
		\includegraphics[scale = 0.3]{ej1}
	\end{center}
	
	\begin{center}
		\includegraphics[scale = 0.3]{ej2}
	\end{center}	

	\begin{figure}[!ht]
 		\centering
  		 \includegraphics[scale = 0.15]{Rutado}
		\caption{Rutado MTP}
	\end{figure}	


\subsubsection{ISUP: Ejemplo de llamada en RTB}

	\begin{center}
		\includegraphics[scale = 0.3]{rtb1}
	\end{center}
	
	\begin{center}
		\includegraphics[scale = 0.3]{rtb2}
	\end{center}
	
	\begin{center}
		\includegraphics[scale = 0.3]{rtb3}
	\end{center}
	
	\begin{center}
		\includegraphics[scale = 0.3]{rtb4}
	\end{center}
	
	\begin{center}
		\includegraphics[scale = 0.3]{rtb5}
	\end{center}

\subsection{TCAP: Transaction Capabilities Application Part}

Da servicio a transacciones de usuario (inicia procesos en un nodo remoto): MAP, INAP, OPAM y CAMEL. Permite intercambiar \textbf{información no relacionada con circuitos}: servicios de red inteligente, localización de terminales móviles por consulta a base de datos y operación y mantenimiento del a red. \textbf{Permite dos tipos de transacciones}: En tiempo real (volumen pequeño y servicio de red sin conexión) y en tiempo diferido (volumen de datos grande, tiempo de respuesta indiferente y servicio de red con conexión).
	
\subsection{Ejemplo de aplicación INAP}
	
	\begin{center}
		\includegraphics[scale = 0.3]{INAP}
	\end{center}
	
	\begin{enumerate}
		\item El rutado de la llamada llega a la central
		\item Se identifica la petición como un proceso de RI
		\item SSP pide información al SCP (database)
		\item SCP ofrece la información necesaria
		\item Con el número de usuario la llamada sigue el proceso de rutado
	\end{enumerate}
	
\hrulefill

\subsection{Conmutación}

El servicio telefónico se ofrece a los usuarios de forma compartida:

	\begin{itemize}
		\item El número de usuarios es grande
		\item Los recursos de la red son limitados (economía)
		\item El único recurso disponible siempre es el bucle de abonado
		\item Para que el servicio sea óptimo: Cálculo del número de recursos en función del grado de servicio deseado.
	\end{itemize}
	
Los nodos de conmutación hacen posible el establecimiento entre usuarios de una red telefónica. La misión es proveer al usuario con un circuito para la comunicación telefónica bidireccional.

La conmutación hoy en día es digital basada en sistemas MIC, realiza operaciones básicas: cambiar intervalos TS (conmutación temporal) y cambiar sistema MIC (conmutación espacial).

	\begin{center}
		\includegraphics[scale = 0.3]{aaa}
	\end{center}

Los conmutadores digitales tienen las siguientes ventajas: tiempos de establecimiento de llamada menores, ocupan menor espacio en planta, posibilidad de construir centrales con muchos abonados y congestión interna muy baja.
	
\subsubsection{Estructura de un sistema de comunicación}
	
La estructura de un sistema de conmutación de circuitos digital es	
	
	\begin{center}
		\includegraphics[scale = 0.3]{estr}
	\end{center}
	
	\begin{itemize}
		\item \textbf{Interfaces}: Conectan la central con el exterior. Líneas de abonado analógicas y digitales, enlaces con otras centrales y enlaces con sistemas de explotación (locales y remotos).
		\item \textbf{Órganos internos}: Realizan algunas funciones necesarias para el establecimiento de conexiones así como O\&M. Emisores y receptores de tonos (señalización general), detección de eventos producidos por abonados (señalización) y órganos de prueba.
		\item \textbf{Red de conexión digital}: Establece y libera conexiones (conmutación de circuitos.
		\item \textbf{Unidad de control}: Supervisa el proceso de establecimiento, mantenimiento y liberación de llamada.
	\end{itemize}
	
\subsubsection{Redes de conexión espacial}

La conmutación espacial permite realizar conexiones entre un conjunto de N líneas de entrada con M líneas de salida. La estructura más simple es una matriz de barras NxM. La complejidad de esta estructura se mide en función del número de interconexiones (NxM), en el caso de una matriz de interconexión local: la matriz será cuadrada, la diagonal no tiene sentido y el número de conexiones será N(N-1).

	\begin{center}
		\includegraphics[scale = 0.3]{barras}
	\end{center}
	
Existe otra arquitectura, triangular, en el cual el control es más complicado.

	\begin{center}
		\includegraphics[scale = 0.3]{triangular}
	\end{center}
	
Las estructuras anteriores de conmutación tienen serios inconvenientes, como el dimensionado y que las conexiones dependen de la fiabilidad del enlace. La \textbf{ventaja} que tienen es que son arquitecturas no bloqueantes, si el usuario origen y destino están libras es posible establecer un circuito de forma independiente del estado del resto de la matriz. Se puede reducir la complejidad de la estructura a costa de añadir probabilidad de bloqueo utilizando \textbf{concenctración}: la conmutación seguirá siendo espacial realizándose en varias etapas, disminuye el número de conexiones y existe probabilidad que una solicitud de entrada por una entrada inactiva no pueda cursarse hacia una salida que está inactiva.

	\begin{center}
		\includegraphics[scale = 0.3]{concentracion}
	\end{center}
	
La comunicación $C$ hacia cualquiera de las salidas está bloqueada mientras $A-D$ y $B-E$ están activas, se produce en la primera etapa una concentración de $4:2$. Cuanto mayor sea el número de conmutadores en la segunda etapa, menor será la probabilidad de bloqueo. Existe un número de conmutadores en la segunda etapa que hace que el bloqueo sea imposible ($k = 2n - 1$).
	
	\begin{center}
		\includegraphics[scale = 0.3]{espacial}
	\end{center}
	
Los enlaces ocupados de la primera etapa provienen de la matriz $i$. Los enlaces ocupados de la segunda etapa provienen de otras matrices. Si hay una matriz más es posible la comunicación, aun así el número de puntos de conmutación es elevado. Es raro necesitar un grado de servicio del 100\%. 

Estas estructuras se pueden aplicar tanto a redes analógicas como digitales. En las digitales, la conmutación puede estar restringida a slots de tiempo. Una determinada conexión sólo necesita estar activa el tiempo correspondiente a los 8 bits de un canal MIC. La contrapartida es que todo ha de estar sincronizado. Aisladamente no se usa (los intervalos de salida tienen que coincidir con las posiciones en las tramas de entrada).

	\begin{center}
		\includegraphics[scale = 0.3]{ee}
	\end{center}
	
\subsubsection{Redes de conmutación temporal}

Las redes de conmutación temporal permiten cambiar la posición de una conversación dentro de una trama TDM (MIC).

	\begin{center}
		\includegraphics[scale = 0.3]{temporal}
	\end{center}

La conmutación se realiza escribiendo y leyendo de una misma memoria. Una posición de memoria se escribe y se lee durante el tiempo correspondiente a un intervalo temporal, por tanto, uno de los requerimientos es la rapidez de las memorias y otro será el número de palabras de la memoria + control.

Existen dos tipos en función de dónde se realiza el control:

	\begin{itemize}
		\item Conmutación por control de escritura: La escritura en la memoria se controla en función de la posición que ocupa cada canal en el MIC de salida, la lectura es secuencial.
		\item Conmutación por control de lectura: La lectura es secuencial y la lectura en memoria se controla en función de la posiciónq ue ocupa cada canal en el MIC de salida.
	\end{itemize}
	
Al igual que las redes de conmutación espacial, las redes de conmutación temporal no se utilizarán por sí solas sino en combinación con otras redes.

	\begin{center}
		\includegraphics[scale = 0.3]{temporal2}
	\end{center}
	
\subsubsection{Redes de conmutación espacial - temporal}

En la práctica, los conmutadores serán combinaciones de varias etapas. Éstas serán habitualmente combinación de etapas temporales y espaciales. Los objetivos son minimizar la probabilidad de bloqueo y la complejidad. Para ello hay múltiples alternativas:

	\begin{itemize}
		\item ST - Espacial Temporal
		\item TS - Temporal Espacial
		\item STS - Espacial Temporal Espacial
		\item TST - Temporal Espacial Temporal
		\item ...
	\end{itemize}
	
Un \textbf{conmutador temporal - espacial (TS)} es una etapa de conmutación temporal seguida de una etapa de conmutación espacial.	
	
	\begin{figure}[!ht]
 		\centering
  		 \includegraphics[scale = 0.4]{images/TS}
		\caption{Conmutación Temporal - Espacial}
	\end{figure}
	
Un \text{conmutador espacial - temporal (ST)} es una etapa de conmutación espacial seguida de una etapa de conmutación temporal
	
	\begin{figure}[!ht]
 		\centering
  		 \includegraphics[scale = 0.4]{images/ST}
		\caption{Conmutación Espacial - Temporal}
	\end{figure}

Un conmutador digital formado por una única etapa espacial es impracticable debido a la alta probabilidad de bloqueo. Un conmutador digital formado por una única etapa temporal puede utilizarse para una capacidad de 250 líneas. Los conmutadores de dos etapas ST o TS pueden utilizarse para conseguir una capacidad de conmutación baja-media. Las altas probabilidades de bloqueo obligan a aumentar el tamaño de los conmutadores temporales.\\

Una etapa de \textbf{conmutación temporal-espacial-temporal} es una red de tres etapas, el control de la red de conexión no asocia una ranura de tiempo en un canal de entrada o salida con una ranura particular de la etapa espacial. La ranura de tiempo de la trama de entrada puede ser conectada a una ranura de tiempo de la salida a través de cualquier ranura de la matriz espacial que esté libre en ese momento. El proceso de captura del camino de conexión se convierte en la tarea de encontrar una ranura disponible en cada extremo de la matriz espacial. En este tipo de conmutador puede producirse bloqueo si no existe ranura de tiempo disponible entre las etapas T. Este tipo de conmutadores ofrecen unas capacidades de conmutación media-alta con bajas probabilidades de bloqueo.

	\begin{figure}[!ht]
 		\centering
  		 \includegraphics[scale = 0.4]{images/TST}
		\caption{Conmutador TST}
	\end{figure}

Un \textbf{conmutador STS} se trata de una red de tres etapas. El conmutador espacial de entrada conecta el bus de entrada con el conmutador temporal durante la ranura de tiempo de entrada. El conmutador espacial de salida conecta el conmutador temporal con el bus de salida durante la ranura de tiempo de salida.

	\begin{figure}[!ht]
 		\centering
  		 \includegraphics[scale = 0.4]{images/STS}
		\caption{Conmutador STS}
	\end{figure}

	
	
	
%\vfill




%\end{multicols}

\end{document}